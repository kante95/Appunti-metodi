\documentclass[12pt]{book}
\usepackage{mathtools}
\usepackage[italian]{babel}
\usepackage{graphicx}
\usepackage[utf8]{inputenc}
\usepackage{float}
\usepackage{tabularx}
\usepackage{chngpage}
\usepackage{amsthm}
\usepackage{mathtools}
\usepackage{amsfonts}
\usepackage{tikz}
\usepackage{amssymb}
\usepackage{gensymb}
\usepackage{wrapfig}
\usepackage{enumerate}
\usepackage{braket}
\usepackage{pgfplots}
\usepackage{bbm}
\usepackage{fancyhdr}
\usepackage{emptypage}

\usetikzlibrary{decorations.markings}

\theoremstyle{plain}

%\numberwithin{equation}{section}
\newcommand{\R}{\mathbb{R}}
\newcommand{\F}{\mathcal{F}}
\renewcommand{\L}{\mathcal{L}}
\newcommand{\C}{\mathbb{C}}
\renewcommand{\H}{\mathcal{H}}
\newcommand{\Sum}{\sum_{n=0}^\infty}
\newcommand{\Res}[1]{\text{Res}f(z)\Big|_{#1}}
\newcommand{\vettore}[1]{\overrightarrow{#1}}
\newcommand{\p}{\mathbf{p}}
\newcommand{\x}{\mathbf{x}}


\newtheorem{thm}{Teorema}[section]
\newtheorem{prop}[thm]{Proposizione}
\newtheorem{coro}[thm]{Corollario}
\newtheorem{lem}[thm]{Lemma}
\theoremstyle{definition}
\newtheorem{dfn}[thm]{Definizione}

\theoremstyle{remark}
\newtheorem*{rmk}{Remark}
\renewcommand{\d}[2]{\frac{d #1}{d #2}} % for derivatives
\newcommand{\dd}[2]{\frac{d^2 #1}{d #2^2}} % for double derivatives
\newcommand{\pd}[2]{\frac{\partial #1}{\partial #2}} 
% for partial derivatives
\newcommand{\pdd}[2]{\frac{\partial^2 #1}{\partial #2^2}} 
% for double partial derivatives

\title{\textbf{Metodi matematici per la fisica e complementi matematici della meccanica quantistica}}
\author{Marco Canteri}
\date{2016}

\def\changemargin#1#2{\list{}{\rightmargin#2\leftmargin#1}\item[]}
\let\endchangemargin=\endlist

\newcommand{\titleGP}{\begingroup % Create the command for including the title page in the document
\centering % Center all text
\vspace*{\baselineskip} % White space at the top of the page

\rule{\textwidth}{1.6pt}\vspace*{-\baselineskip}\vspace*{2pt} % Thick horizontal line
\rule{\textwidth}{0.4pt}\\[\baselineskip] % Thin horizontal line

{\LARGE METODI MATEMATICI PER LA \\ FISICA E COMPLEMENTI MATEMATICI \\[0.4\baselineskip] DELLA MECCANICA QUANTISTICA}\\[0.2\baselineskip] % Title

\rule{\textwidth}{0.4pt}\vspace*{-\baselineskip}\vspace{3.2pt} % Thin horizontal line
\rule{\textwidth}{1.6pt}\\[\baselineskip] % Thick horizontal line

\scshape % Small caps
Un'affascinante viaggio alla scoperta dell'analisi complessa,\\ delle distribuzioni, % Tagline(s) or further description
degli spazi di Hilbert e delle loro molteplici applicazioni alla Fisica \\[\baselineskip] % Tagline(s) or further description

\vspace*{8\baselineskip} % Whitespace between location/year and editors

{\huge CANTERI MARCO\par}\vspace*{0.3\baselineskip} % Editor list
{\itshape Università degli studi di Trento\par} % Editor affiliation

\vfill % Whitespace between editor names and publisher logo


{\scshape 2016} \\[0.3\baselineskip] % Year published
{\large CANTERI PRESS}\par % Publisher

\endgroup} 


\usepackage[a4paper, inner=1.5cm, outer=3cm, top=3cm, 
bottom=3cm, bindingoffset=1cm,headheight=110pt]{geometry} 

\pagestyle{fancy}
\fancyhf{}
\fancyhead[LE]{\leftmark}
\fancyhead[RO]{\rightmark}
\fancyfoot[C]{\thepage}
\begin{document}
\pagestyle{empty} 
\titleGP
\newpage
\pagestyle{empty}
%% copyrightpage
\begingroup
\footnotesize
\parindent 0pt
%\parskip \baselineskip
Marco Canteri\\
Università degli studi di Trento\\
Finito di stampare: \today
\vspace{\fill}\\

\textcopyright{} 2016 Canteri Press \\
All rights reserved.\\
This work is subject to copyright. All rights are reserved by the Publisher, whether the whole or part
of the material is concerned, specifically the rights of translation, reprinting, reuse of illustrations,
recitation, broadcasting, reproduction on microfilms or in any other physical way, and transmission
or information storage and retrieval, electronic adaptation, computer software, or by similar or dissimilar
methodology now known or hereafter developed.
The use of general descriptive names, registered names, trademarks, service marks, etc. in this
publication does not imply, even in the absence of a specific statement, that such names are exempt from
the relevant protective laws and regulations and therefore free for general use.
The publisher, the authors and the editors are safe to assume that the advice and information in this
book are believed to be true and accurate at the date of publication. Neither the publisher nor the
authors or the editors give a warranty, express or implied, with respect to the material contained herein or
for any errors or omissions that may have been made.


\vfill
\begin{center}
\begin{tabular}{ll}
Prima edizione:  & Dicembre 2016 \\
\end{tabular}
\end{center}


Canteri Press \\
Cerro Veronese (VR) \\
https://github.com/kante95

%%%%{\LARGE\plogo}
%\vspace*{2\baselineskip}


\endgroup
\clearpage 

\tableofcontents
\pagestyle{fancy}
\chapter{Analisi complessa}
\section{Richiami sui numeri complessi}
Brevi richiami sui numeri complessi $z=x+iy \in \C$, la parte reale si indica con $\text{Re}(z) = x$ e quella immaginaria con $\text{Im}(z) = y$; $x,y\in \R$. Nella rappresentazione polare il numero complesso si può scrivere come $z=\rho e^{i\theta}$. Si definisce l'operazione di coniugazione complessa come $\overline{z} = x-iy$. Nei numeri complessi valgono le seguenti disuguaglianze
\begin{itemize}
\item $|z_1+z_2| \leq |z_1|+|z_2|\quad \forall z_1,z_2\in \C\qquad \text{(1\degree  disuguaglianza triangolare)}$
\item $|z_1-z_2| \geq \big||z_1|-|z_2|\big|\quad \forall z_1,z_2\in \C\qquad \text{(2\degree  disuguaglianza triangolare)}$
\item $\displaystyle\frac{1}{|z_1-z_2| } \leq \frac{1}{\big||z_1|-|z_2|\big|}\quad \forall z_1,z_2\in \C\qquad$
\item $|\text{Re}(z)|\leq |z|\quad \forall z\in \C$
\item $|\text{Im}(z)|\leq |z|\quad \forall z\in \C$
\end{itemize}
Una funzione $f:\C\to\C$ può essere scritta come $f(z=x+iy) = u(x,y)+iv(x,y)$ e si può rappresentare come serie di potenze
\[f(z) = \Sum a_n z^n \qquad a_n\in \C\]
Ricordiamo la serie geometrica $\Sum z^n = \frac{1}{1-z}$ che converge per $|z|<1$. Per trovare il raggio $R_0$ di convergenza di una serie esiste il criterio della radice
\[\lim_{n\to \infty} \frac{|a_{n+1}||z|^{n+1}}{|a_n||z|^n} <1\]
$f(z)$ è continua in $R$ semplicemente connesso se e solo se $u,v$ sono continue im $R$. Ricordiamo la nozione di derivabilità, $f(z)$ è differenziabile in $z_0\in R$ se e solo se $\exists \lambda\in\C, \omega(z_0,h)$ tali che 
\[f(z_0+h) = f(z_0) + \lambda h+\omega(z_0,h)\quad \bigg|\frac{\omega(z_0,h)}{h}\bigg|\xrightarrow[|h|\to 0]{} 0 \quad h\in \C\]
Se $f(z)$ è differenziabile in tutto $R$ allora $f$ è analitica in $R$ ed inoltre
\[\d{f}{z} = \lambda = \lim_{h\to 0}\frac{f(z_0+h)-f(z_0)}{h}\]
\begin{thm}
(Equazioni di Cauchy-Riemman) Se $f$ è analitica in $R$ allora valgono
\[\begin{cases}
    u_x =v_y    \\
   v_x=-u_y\\
  \end{cases} \]
\end{thm}
\begin{thm}
Se $u,v\in C^1(R)$ e soddisfano le equazioni di Cauchy-Riemman allora $f(z)$ è analitica in $R$
\end{thm}
Dimostriamo ora che una serie di potenze $f(z) =\Sum a_n z^n$ nel suo raggio di convergenza $R_0$ è analitica, infatti:
\[\omega = f(z+h)-f(z)-h\Sum na_nz^{n-1} = \Sum a_n[(z+h)^n-z^n-hnz^{n-1}]= \Sum a_n[h^2g_n]\]
dai cui il limite $\bigg|\frac{\omega(z_0,h)}{h}\bigg|$ tende a 0.
\newline
L'integrazione sul campo complesso si può ricondurre all'integrazione sul campo reale, infatti sia $f(z) = u+iv$ allora
\[\int_\gamma f(z)\,dz = \int_\gamma (u +iv)(dx+idy) = \int_\gamma[udx-vdy] +i\int_\gamma vdx+udy\] 
e utilizzando una parametrizzazione della curva $\gamma: [a,b]\to \phi(t)+i\psi(t)$ abbiamo
\[\int_\gamma f(z)\,dz = \int_a^b f(\gamma(t))(\phi'(t)+i\psi'(t))\,dt\]
\begin{lem}
(Di Darboux) Se $f(z)$ è continua su una curva orientata $\gamma$ con lunghezza $L(\gamma)$ allora
\[\bigg|\int_\gamma f(z)\,dz \bigg| \leq \underset{\gamma}{sup}|f(z)|\,L(\gamma)\]
\end{lem}
\begin{thm}
(Di Cauchy-Goursat) Sia $f(z)$ una funzione analitica su un insieme semplicemente connesso $R$ e sia $\gamma$ una curva contenuta in $R$ allora vale
\[\int_\gamma f(z)\,dz=0\]
\end{thm}
\hspace{-1.6em}\textbf{Dimostrazione:}\\
La dimostrazione di Cauchy prevedeva anche la continuità della derivata di $f$ ma Goursat dimostrò che si può indebolire il teorema con solo l'ipotesi di analiticità di $f$. Per semplicità però
assumiamo la derivata di $f$ continua, quindi dalla definizione di integrale e dal teorema di Gauss-Green
\[\int_\gamma f(z)\,dz =  \int_\gamma[udx-vdy] +i\int_\gamma vdx+udy =  \iint_D[-u_ydx-v_xdy] +i\int_D -v_ydx+u_xdy\]
Dove $\gamma = \partial D$, dalle equazioni di Cauchy-Riemman segue che gli integrandi sono nulli per cui l'integrale è nullo.\\
\begin{lem}
Se $f(z)$ è analitica in una regione semplicemente connessa $R$, siano $\gamma(z_0,z_1),\Gamma(z_0,z_1)$ due curve che uniscono gli stessi punti allora
\[\int_{\gamma(z_0,z_1)}f(z)\,dz = \int_{\Gamma(z_0,z_1)}f(z)\,dz\]
\end{lem}
\hspace{-1.6em}\textbf{Dimostrazione:}\\
La dimostrazione è una conseguenza del teorema di cauchy infatti si scelga la curva chiusa unione di $\gamma$ e $-\Gamma$ orientata quindi in senso contrario, l'integrale sul percorso chiuso è nullo, quindi la somma dell'integrale su $\gamma$ all'integrale su $\Gamma$ è nullo, cambiando orientazione e quindi di segno si ottiene la tesi.\\
\begin{thm}
(Teorema fondamentale del calcolo) Se $f(z)$ è analitica in $R$ insieme semplicemente connesso allora
\[F(z) := \int_{z_0}^z f(z')\,dz'\]
è analitica in $R$ e vale $F'(z) = f(z)$ 
\end{thm}
\hspace{-1.6em}\textbf{Dimostrazione:}\\
Valutiamo
\[\omega = F(z+h)-F(z)-hf(z) = \int_{z_0}^{z+h}f(z')\,dz' - \int_{z_0}^{z}f(z')\,dz' - \left(\int_{z}^{z+h}dz'\right)f(z) =   \]
\[\int_{z_0}^{z}f(z')\,dz' +\int_{z}^{z+h}f(z')\,dz' - \int_{z_0}^{z}f(z')\,dz' - \left(\int_{z}^{z+h}dz'\right)f(z) =\int_{z}^{z+h}[f(z')-f(z)]\,dz' \]
ora valutiamo il limite di $|\omega/h|$ per $h\to 0$ e otteniamo
\[\bigg|\frac{1}{h}\int_{z}^{z+h}[f(z')-f(z)]\,dz'\bigg|\leq\frac{1}{|h|}\bigg| \int_{z}^{z+h}[f(z')-f(z)]\,dz'\bigg|\leq \frac{|h|}{|h|} sup|f(z')-f(z)|\to 0 \]
per il lemma di Darboux e per la continuità.\\ 
\begin{thm}
(Formula integrale di Cauchy) Sia $f(z)$ analitica in $R$ semplicemente connesso e $\gamma\in R$ una curva chiusa, se $z_0\in R$ allora vale
\[f(z_0) = \frac{1}{2\pi i}\int_\gamma \frac{f(z)}{z-z_0}\, dz\]
\end{thm}
\hspace{-1.6em}\textbf{Dimostrazione:}\\
Per il principio di deformazione sui cammini abbiamo:
\[\frac{1}{2\pi i}\int_\gamma \frac{f(z)}{z-z_0}\, dz = \frac{1}{2\pi i}\int_{C_r} \frac{f(z)}{z-z_0}\, dz = \frac{1}{2\pi i}\int_{C_r} \frac{f(z)-f(z_0)+f(z_0)}{z-z_0}\, dz\]
dove $C_r$ è la circonferenza centrata in $z_0$ e di raggio $r$.
\[\frac{1}{2\pi i}\int_{C_r} \frac{f(z)-f(z_0)+f(z_0)}{z-z_0}\, dz = \frac{1}{2\pi i}\int_{C_r} \frac{f(z)-f(z_0)}{z-z_0}\, dz +\frac{1}{2\pi i}\int_{C_r} \frac{f(z_0)}{z-z_0}\, dz = \]
\[\frac{1}{2\pi i}\int_{C_r} \frac{f(z)-f(z_0)}{z-z_0}\, dz +f(z_0) \]
infatti è facile vedere parametrizzando la circonferenza e usando la definizione di integrale che $\int_{C_r}\frac{1}{z-z_0}\, dz = 2\pi i$. Valutiamo allora
\[\bigg|\frac{1}{2\pi i}\int_{\gamma} \frac{f(z)-f(z_0)}{z-z_0}\, dz - f(z_0)\bigg| =\bigg|\frac{1}{2\pi i}\int_{C_r} \frac{f(z)-f(z_0)}{z-z_0}\, dz\bigg|\leq \frac{1}{2\pi}\frac{\varepsilon}{r}2 \pi r \]
per continuità della funzione e per il lemma di Darboux. Per l'arbitrarietà di $\varepsilon$ segue la tesi.
\begin{coro}
\[f^{(n)}(z_0) = \frac{n!}{2\pi i}\int_\gamma \frac{f(z)}{(z-z_0)^{n+1}}\, dz\] 
Ovvero una funzione analitica è infinitamente volte derivabile.
\end{coro}
\hspace{-1.6em}\textbf{Dimostrazione:}\\
La dimostrazione si fa per induzione completa, mostriamo solo per $n=1$:
\[\frac{f(z+h)-f(z)}{h} = \frac{1}{h}\frac{1}{2\pi i}\int_\gamma f(z')\left[\frac{1}{z'-z-h}-\frac{1}{z'-z}\right] \, dz' = \frac{1}{2\pi i}\int_\gamma \frac{f(z')}{(z'-z-h)(z'-z)} \, dz'\]
dali limite $h\to 0$ si ha il risultato.
\begin{thm}
(Teorema di Morera) Sia $f$ continua in $R$ semplicemente connesso, se $\int_\gamma f(z)\, dz = 0$ per ogni curva $\gamma$ chiusa in $R$ allora $f$ è analitica in $R$
\end{thm}
\hspace{-1.6em}\textbf{Dimostrazione:}\\
Dato che l'integrale è nullo su ogni curva chiusa abbiamo che gli integrali non dipendono dal percorso, quindi possiamo definire $F(z) = \int_{z_0}^z f(z')\, dz'$. Mostriamo che $F(z)$ è analitica ed ha come derivata $f(z)$:
\[\omega:= F(z+h)-F(z) -hf(z)= \int_{z_0}^{z+h} f(z')\, dz' - \int_{z_0}^z f(z')\, dz' - f(z)\int_z^{z+h}\,dz'\] 
\[= \int_{z_0}^{z} f(z')\, dz' +\int_{z}^{z+h} f(z')\, dz' - \int_{z_0}^z f(z')\, dz' - f(z)\int_z^{z+h}\,dz' = \int_{z}^{z+h} f(z')-f(z)\, dz'\]
\[\left|\frac{\omega}{h}\right| = \frac{1}{|h|}\left|\int_{z}^{z+h} f(z')-f(z)\, dz'\right|\leq sup[f(z')-f(z)]\]
Che per continuità tende a 0.\\
\begin{coro}
Sia $f(z) = \Sum a_n (z-z_0)^n$ con raggio di covergenza $R$ allora $f$ è analitica $\forall z \in |z-z_0|\leq R $
\end{coro}
\hspace{-1.6em}\textbf{Dimostrazione:}\\
$f(z)$ è continua perchè la serie converge uniformemente, calcoliamo quindi  l'integrale su ogni curva chiusa
\[\int_\gamma f(z)\, dz = \int_\gamma \Sum a_n (z-z_0)^n\,dz =\Sum a_n \int_\gamma(z-z_0)^n\,dz = 0  \]
Il simbola di somma e integrazione si può scambiare in quanto la convergenza è uniforme. La tesi segue dunque dal teorema di Morera.\\
\begin{thm}
(Teorema di Liouville) Se $f(z)$ è intera e limitata: $|f(z)|\leq M$ allora $f(z)$ è costante 
\end{thm}
\hspace{-1.6em}\textbf{Dimostrazione:}\\
sfruttiamo la formula di Cauchy e scriviamo la sua derivata come integrale su un cerchio di raggio $r$ ($\gamma$)
\[|f'(z_0)| = \left|\frac{1}{2\pi i}\oint_\gamma \frac{f(z)}{(z-z_0)^2}\right|\leq \frac{1}{2\pi}\left|\oint_\gamma\frac{f(z)}{(z-z_0)^2}\right|\leq \frac{1}{2\pi}\frac{M}{r^2}2\pi r\leq \frac{M}{r}\xrightarrow[r\to\infty]{}  0\]
Il raggio può andare all'infinito in quanto la funzione è intera.\\
Dal teorema di Liouville si dimostra il teorema fondamentale dell'algebra.
\section{Singolarità e residui}
Prendiamo una regione anulare $R-\{z_0\}$ con $z_0$ singolarità isolata. Prendiamo il percorso $\gamma$ in figura
\begin{figure}[H]
\centering
\begin{tikzpicture}
%\draw (0,0)node{\color{black}$z_0$} circle (0.5cm);
%\draw (0,0)node{\color{black}$z_0$} circle (1cm);

 % draw the two circles and decorate them with arrows
    \draw[
        decoration={markings, mark=at position 0.625 with {\arrow{>}}},
        postaction={decorate}
        ]
        (0,0) circle (1);
    \draw[ 
        decoration={markings, mark=at position 0.125 with {\arrow{>}}},
        postaction={decorate}
        ]
        (0,0) circle (2);

    % draw the connecting line
    \draw[ 
        decoration={markings, mark=at position 0.5 with {\arrow{>}}, mark=at position 0.3 with {\arrow{<}}},
        postaction={decorate}
        ]
        (0,1) -- (0,2);

   \draw(0,0)node{$z_0$};
   \draw(1.3,0)node[above=1.5pt]{$C_2$};
   \draw(2,1)node[above=1pt]{$C_1$};
   \draw(-0.3,1.4)node{$l$};
   \draw(-1.3,1.1)node{$z$};
\end{tikzpicture}
\caption{$\gamma = C_1+2l+C_2$}\label{percorso}
\end{figure}
scriviamo la $f$ utilizzando la formula di Cauchy
\[f(z) = \frac{1}{2\pi i}\oint_\gamma \frac{f(z')}{z'-z}\,dz' = \frac{1}{2\pi i}\int_{C_1}\frac{f(z_1)}{z_1-z}\,dz_1 + I_l -I_l - \frac{1}{2\pi i}\int_{C_2}\frac{f(z_2)}{z_2-z}\,dz_2\]
\[f(z) =\frac{1}{2\pi i}\int_{C_1}dz_1\,f(z_1)\frac{1}{z_1-z} - \frac{1}{2\pi i}\int_{C_2}dz_2\,f(z_2)\frac{1}{z_2-z} \]
Ora consideriamo il seguente rapporto 
\[\frac{1}{z_{1,2}-z} =\frac{1}{z_{1,2}-z_0+z_0-z} = \frac{1}{(z_{1,2}-z_0)(1-\frac{z-z_0}{z_{1,2}-z_0})} = \frac{1}{z_{1,2}-z_0}\frac{1}{1-\xi_{1,2}} \]
dove è stata chiamata $\xi_{1,2} = \frac{z-z_0}{z_{1,2}-z_0}$, sfruttiamo la serie geometria e scriviamo
\[\frac{1}{1-\xi_1} = \Sum \xi_1^n \quad |\xi_1|< 1 \qquad \frac{1}{1-\xi_2} = -\sum_{n=1}^\infty \frac{1}{\xi_2^n} \quad |\xi_2|> 1 \]
Lo si vede geometricamente dalla figura \eqref{percorso}, quindi abbiamo:
\[\frac{1}{z_1-z} = \frac{1}{z_1-z_0}\Sum \frac{(z-z_0)^n}{(z_1-z_0)^{n}} =\Sum \frac{(z-z_0)^n}{(z_1-z_0)^{n+1}} \]
\[\frac{1}{z_2-z} = -\frac{1}{z_2-z_0}\sum_{n=1}\frac{(z_2-z_0)^{n}}{(z-z_0)^{n}} =-\sum_{n=1}^{\infty}\frac{(z_2-z_0)^{n-1}}{(z-z_0)^{n}} \]
la nostra funzione può essere scritta allora come:
\[f(z) =\frac{1}{2\pi i}\int_{C_1}dz_1\,f(z_1)\Sum \frac{(z-z_0)^n}{(z_1-z_0)^{n+1}} + \frac{1}{2\pi i}\int_{C_2}dz_2\,f(z_2)\sum_{n=1}^{\infty}\frac{(z_2-z_0)^{n-1}}{(z-z_0)^{n}}\]
\[f(z) =\Sum (z-z_0)^n  \underbrace{\frac{1}{2\pi i}\int_{C_1}dz_1\,\frac{f(z_1)}{(z_1-z_0)^{n+1}}}_{a_n} + \sum_{n=1}^{\infty}\frac{1}{(z-z_0)^{n}}\underbrace{\frac{1}{2\pi i}\int_{C_2}dz_2\,f(z_2)(z_2-z_0)^{n-1}}_{b_n}\]
otteniamo quindi lo sviluppo di Laurent:
\[f(z) = \Sum a_n(z-z_0)^n +\sum_{n=1}^{\infty}\frac{b_n}{(z-z_0)^{n}}\]
\begin{coro}
Se $f(z)$ è analitica in $z_0$ allora la serie di Laurent diventa la serie di Taylor della funzione.
\end{coro}
\hspace{-1.6em}\textbf{Dimostrazione:}\\
Se $f$ è analitica in $z_0$ allora $b_n=0$ per Cauchy, inoltre abbiamo:
\[a_n = \frac{n!}{n!}\frac{1}{2\pi i}\int_{C_1}dz_1\,\frac{f(z_1)}{(z_1-z_0)^{n+1}}= \frac{f^{(n)}(z)}{n!} \]
Per la forumla di Cauchy per le derivate.\\
\newline Si possono fare distinzione per i tipi di singolarità:
\begin{itemize}
\item \textbf{Polo di ordine $N$:} se $b_N \neq 0$ e $b_n=0 \, \forall n>N$ allora $z_0$ si dice polo di ordine $N$, il polo di ordine 1 si chiama anche polo semplice.
\item \textbf{Singolarità essenziale} se $b_n=0\,\forall n$ allora $z_0$ è una singolarià essenziale, nell'intorno della singolarità essenziale la funzione può assumere valori arbitrariamente grandi o piccoli.
\end{itemize}
Possiamo ora definire il residuo di una funzione:
\begin{dfn}
Si chiama residuo della funzione $f$ in $z_0$ il seguente:
\[\text{Res}f(z)\Big|_{z_0} := b_1 = \frac{1}{2\pi i}\oint_\gamma f(z)\,dz\]
\end{dfn}
Per calcolare il residuo ci sono diversi modi che dipendono dal tipo di singolarità:
\begin{itemize}
 \item  nel caso $z_0$ sia un polo semplice di ordine 1 possiamo prendere la sviluppo di Laurent e moltiplicare per $z-z_0$:
\[f(z) = \Sum a_n(z-z_0)^{n} +\frac{b_1}{(z-z_0)} \]
\[f(z)(z-z_0) = \Sum a_n(z-z_0)^{n+1} +b_1 \]
e allora evidentemente abbiamo
\[\Res{z_0} = \lim_{z\to z_0}f(z)(z-z_0)\]
\item Si può generalizzare al polo di ordine $N$ e otteniamo:
\[\Res{z_0} = \frac{1}{(N-1)!}\lim_{z\to z_0}\frac{d^{N-1}}{dz^{N-1}}[f(z)(z-z_0)^N]  \]
\item Nel caso $z_0$ sia una singolarità essenziale conviene sviluppare la funzione in serie di Laurent e prendere direttamente il coefficiente $b_1$
\end{itemize}
si può scrivere una formula nel caso di funzioni del tipo $f(z) = h(z)/g(z)$ nel caso di singolarità di poli semplici. Fattorizziamo infatti la singolarità e scriviamo $g$ come $g(z) = (z-z_0)\tilde{g}(z)$ con $\tilde{g}(z_0)\neq 0$. Ora applichiamo la formula per il residuo semplice:
\[f(z)(z-z_0) = \frac{h(z)}{\tilde{g}(z)}\xrightarrow[z\to z_0]{} \frac{h(z_0)}{\tilde{g}(z_0)} \]
Ora basta capire che cos'è $\tilde{g}(z_0)$, prendiamo la derivata di $g$
\[\d{g(z)}{z} = \tilde{g}(z) + (z-z_0)\tilde{g'}(z)\xrightarrow[z\to z_0]{} \tilde{g}(z_0)\]
ovvero $g'(z_0) = \tilde{g}(z_0)$ dai cui il residuo di $f$ risulta:
\[\Res{z_0} = \frac{h(z_0)}{g'(z_0)}\]
Si può definire quello che è un residuo all'infinito prendendo una curva orientate negativamente, quindi l'idea è di prendere come interno della curva la regione esterna di un cerchio, quindi appunto quella che contiene anche ``l'infinito''
\begin{dfn}
Si definisce il residuo all'infinito nel modo seguente:
\[\Res{\infty}=\frac{1}{2\pi i}\oint_{-\gamma}f(z)\,dz\]
\end{dfn}
Il residuo all'infinito possiamo ricondurlo ad un residuo al finito facendo una sostituzione $z\to 1/\omega$, infatti otteniamo:
\[\Res{\infty}=\frac{1}{2\pi i}\oint_{-\gamma}f(z)\,dz = \frac{1}{2\pi i}\oint_{\gamma'}-\frac{f\left(\frac{1}{\omega}\right)}{\omega^2}\,d\omega = \text{Res}\Big(-\frac{1}{\omega^2}f\left(\frac{1}{\omega}\right)\bigg|_{0} \]
Nota: la curva $-\gamma$ si trasforma in $\gamma'$ che oltre che ad essere una curva differente cambia anche verso di percorrenza, infatti prendendo per semplicità $\gamma$ come una circonferenza di raggio $r$ abbiamo che $\gamma:\, z = re^{-i\theta}$ cambiando variabile otteniamo $1/\omega = re^{-i\theta} $ da cui $\gamma':\, \omega =1/r e^{i\theta}$ che dato il segno dell'esponente è percorsa in senso antiorario.
\begin{thm}
(Teorema dei residui) Sia $f$ funzione meromorfa con singolarità in $\{z_k\}$ allora:
\[\oint_\gamma f(z)\,dz = 2\pi i \sum_k \Res{z_k}\]
\end{thm}
\hspace{-1.6em}\textbf{Dimostrazione:}\\
La dimostrazione si fa sfruttando il teorema di Cauchy per insiemi molteplicemente connessi e calcolando l'integrale su $\gamma$ come somma degli integrali su $k$ curve $\gamma_k$ che girano attorno alla singolarità $z_0$ al cui interno $z_0$ sia l'unica singolarità, allora per definizione del residuo il valore dell'integrale su $\gamma_k$ vale $2\pi i \Res{z_k}$, sommando tutti questi contributi si ottiene il teorema.\\
%\begin{wrapfigure}[100]{r}{0.25\textwidth}
%\begin{tikzpicture}[scale=.5]
%\draw [<->](-5,0) -- (5,0)node[below=1.5pt] {\color{black}$x$};
%\draw[<->](0,-5)--(0,5)node[right=1.5pt] {\color{black}$y$};
% \draw [red,thick,domain=0:180] plot ({3*cos(\x)}, {3*sin(\x)});
% \draw [blue,thick,domain=180:360] plot ({3*cos(\x)}, {3*sin(\x)});
%\draw (2.6,2.6) node{\color{black}$C_R^+$};
%\draw (2.6,-2.6) node{\color{black}$C_R^-$};
%\end{tikzpicture}
%\caption{Cammini di integrazione}
%\end{wrapfigure}\leavevmode
\begin{lem}
(Lemma di Jordan) Sia $f(z)$ una funzione tale che abbiamo $|f(z)| \leq \varepsilon(R) \xrightarrow[R\to \infty]{} 0$, allora sia $C_R^+$ la semicirconferenza positiva di raggio $R$ e centro l'origine e $C_R^-$ la contoparte negativa, abbiamo:
\[\lim_{R\to \infty}\int_{C_R^+}e^{ikz}f(z)\,dz = 0 \quad \forall k>0\]
\[\lim_{R\to \infty}\int_{C_R^-}e^{ikz}f(z)\,dz = 0 \quad \forall k<0\]
\end{lem}
\hspace{-1.6em}\textbf{Dimostrazione:}\\
Valutiamo il primo integrale parametrizzandolo con $z=Re^{i\theta}$
\[I_{R^+} = \int_{C_R^+}e^{ikz}f(z)\,dz = \int_0^\pi f(Re^{i\theta})e^{ikR(\cos\theta+i\sin\theta)} iRe^{i\theta}\,d\theta \]
\[|I_{R^+}| \leq \int_0^\pi |f(Re^{i\theta})|e^{-kR\sin\theta}R\,d\theta\leq R\varepsilon(R)\int_0^\pi e^{-kR\sin\theta}\,d\theta\leq 2R\varepsilon(R)\int_0^{\frac{\pi}{2}} e^{-kR\sin\theta}\,d\theta\]
Ora consideriamo che nell'intervallo che ci interessa $[0,\pi/2]$: $\sin\theta \geq 2\theta/\pi$, per cui $-kR\sin\theta\leq -2kR\theta/\pi$ per cui possiamo maggiorare ancora il nostro integrale con:
\[|I_{R^+}| \leq 2R\varepsilon(R)\int_0^{\frac{\pi}{2}} e^{-\frac{2kR}{\pi}\theta}\,d\theta = -\frac{\varepsilon(R)\pi}{k}(e^{-kr}-1) \xrightarrow[R\to \infty]{} 0\]
La dimostrazione del secondo segue la prima.\\
\newline
Alcuni integrali con integrandi nella forma $f(x)/(x-x_0)$ normalmente divergerebbero se integrati in un intervallo che include $x_0$, è possibile però associare una regolarizzazione in modo che l'integrale converga, questo ci porta alla seguente definizione
\begin{dfn}
Se il limite esiste si chiama parte principale secondo Cauchy dell'integrale convergente la seguente regolarizzazione:
\[P\int_a^b\frac{f(x)}{x-x_0}\,dx := \lim_{\varepsilon\to0}\left(\int_a^{x_0-\varepsilon}+\int_{x_0+\varepsilon}^b\right)\frac{f(x)}{x-x_0}\,dx\]
\end{dfn}
\begin{lem}
Se $f(z)$ è analitica in $\text{Im}(z)>0$ e se ivi $|f|\xrightarrow[|z|\to \infty]{} 0$ allora vale:
\[P\int_{-\infty}^{+\infty}\frac{f(x)}{x-x_0}\, dx = i\pi f(x_0)\]
\end{lem}
\hspace{-1.6em}\textbf{Dimostrazione:}\\
consideriamo la funzione $g(z) = f(z)/(z-z_0)$ e integriamola lungo la solito semicirconferenza positiva. L'integrale è nullo perchè all'interno della curva la funzione è analitica, otteniamo quindi:
\[\int_\gamma g(z) = I_{C_{R^+}} + \int_{-R}^{x_0-\varepsilon}g(x)\, dx + \int_{x_0+\varepsilon}^{R}g(x) + I_\varepsilon= 0\]
\begin{figure}[H]
\centering
\begin{tikzpicture}
\draw [<->](-5,0) -- (5,0)node[below=1.5pt] {\color{black}$x$};
\draw[->](0,0)--(0,4)node[right=1.5pt] {\color{black}$y$};
\draw [red,thick,domain=0:180] plot ({3*cos(\x)}, {3*sin(\x)});
\draw [red,thick,domain=0:180] plot ({1.4+0.3*cos(\x)}, {0.3*sin(\x)});
\draw[red] (-3,0)--(1.1,0);
\draw[red] (1.7,0)--(3,0);
\draw[fill] (1.4,0) node[below=1.5pt] {\color{black}$x_0$} circle (1pt);
\draw (-3,0) node[below=1.5pt] {\color{black}$-R$};
\draw (3,0) node[below=1.5pt] {\color{black}$R$};
\draw (1.4,0.5) node {\color{black}$I_\varepsilon$};
\end{tikzpicture}
\caption{percorso di integrazione}
\end{figure}
Nel limite $R\to \infty$, $I_{C_{R^+}}$ si annulla, $I_\varepsilon$ nel limite $\varepsilon\to 0$ diventa:
\[\int_{I_\varepsilon} \frac{f(z)}{z-z_0}\,dz = \int_{\pi}^0 \frac{f(x_0+\varepsilon e^{i\theta})}{\varepsilon e^{i\theta}}i\varepsilon e^{i\theta}\,d\theta = i\int_\pi^0f(x_0+\varepsilon e^{i\theta})\,d\theta \xrightarrow[\varepsilon\to 0]{} -i\pi f(x_0)\]
Otteniamo quindi nel limite $R\to \infty $ e $\varepsilon\to0$, regolarizzando la funzione $g$ con la parte principale:
\[P\int_{-\infty}^{+\infty}\frac{f(x)}{x-x_0}\, dx - i\pi f(x_0) = 0\]
\newline
a questo punto se consideriamo la funzione complessa: $f(x) = \text{Re}f(x) + i\text{Im}f(x)$ utilizzando questo teorema ricaviamo:
\[P\int_{-\infty}^{+\infty}\frac{\text{Re}f(x) + i\text{Im}f(x)}{x-x_0}\, dx = i\pi(\text{Re}f(x_0) + i\text{Im}f(x_0))\]
\[P\int_{-\infty}^{+\infty}\frac{\text{Re}f(x)}{x-x_0}\, dx + iP\int_{-\infty}^{+\infty}\frac{\text{Im}f(x)}{x-x_0}\, dx = i\pi\text{Re}f(x_0) -\pi\text{Im}f(x_0)\]
\\da cui le \textbf{\emph{relazioni di dispersione}} o \textbf{\emph{Trasformate di Hilbert}}:
\[\begin{cases}
    \displaystyle\frac{P}{\pi}\int_{-\infty}^{+\infty}\frac{\text{Re}f(x)}{x-x_0}\, dx =  -\text{Im}f(x_0)   \\ \\
   \displaystyle\frac{P}{\pi}\int_{-\infty}^{+\infty}\frac{\text{Im}f(x)}{x-x_0}\, dx = \text{Re}f(x_0)\\
  \end{cases} \]
\section{Sviluppo di Mittag-Leffler e serie numeriche}
Consideriamo un funzione $f(z)$ meromorfa in $\C$ con infiniti poli semplici $z_n$ con $|z_1|<|z_2|<\dots$, inoltre la funzione è limitata sulle circonferenza di centro origine e raggio $|z_n|$: $|f(z)|\leq M\, \forall z\in C_{z_n}$. Siano i residui della funzione $\Res{z_n} = b_n$, allora se la funzione è regolare in 0 possiamo scrivere lo sviluppo di Mittag-Leffler per la funzione:
\[f(z) = f(0) + \sum_n b_n \left(\frac{1}{z-z_n} + \frac{1}{z_n}\right)\]
la dimostrazione procede considerando la funzione $g(z') = f(z')/z'(z'-z)$ e integrandola su una circonferenza $C_N$ che contenga $N$ poli, utilizzando il teorema dei residui abbiamo il residuo in 0, il residuo in $z$ e i restanti residui nei poli di $f$, abbiamo quindi:
\[I_N = \frac{1}{2\pi i}\oint_{C_N} g(z')\,dz' = -\frac{f(0)}{z} + \frac{f(z)}{z} + \sum_j^N \frac{b_j}{z_j(z_j-z)}\]
Maggioriamo l'integrale con il lemma di darboux:
\[|I_N| \leq \frac{1}{2\pi}2\pi R\, \underset{C_N}{\text{sup}}\left|\frac{f(z')}{z'(z'-z)}\right|\leq \frac{RM}{R(R-z)}\xrightarrow[R\to \infty]{} 0  \]
nel limite includiamo tutti i residui della funzione e ricaviamo dunque:
\[0 = -\frac{f(0)}{z} + \frac{f(z)}{z} + \sum_n \frac{b_n}{z_n(z_n-z)} \]
\[f(z) = f(0) + \sum_n \frac{b_nz}{z_n(z-z_n)} = f(0) + \sum_n\left(\frac{b_n}{z-z_n} + \frac{b_n}{z_n}\right) \]
questo sviluppo di permette di ricostruire la funzione conoscendo il suo comportamente nei poli.\\
\newline
Sempre utilizzando il metodo dei residui possiamo valutare la somma di serie del tipo:
\[\sum_{-\infty}^{+\infty}f(n)\qquad \sum_{-\infty}^{+\infty}(-1)^nf(n)\]
con $f(z)$ meromorfa, con un numero finito di poli $z_i$ e tale che $f(z) \xrightarrow[|z|\to \infty]{} |z|^{-1-\alpha}$ con $\alpha>0$.
Abbiamo allora:
\begin{enumerate}[a)]
\centering
\item$\displaystyle\sum_{-\infty}^{+\infty}f(n) = -\pi \displaystyle\sum_i \text{Res}(f(z)\text{cotan}(\pi z)\Big|_{z_i}$
\item $\displaystyle\sum_{-\infty}^{+\infty}(-1)^nf(n) = -\pi \sum_i \text{Res}\bigg(\frac{f(z)}{\sin(\pi z)}\bigg|_{z_i}$
\end{enumerate}
nel caso in cui $f(z)$ non sia definita in 0, la sommatoria esclude il punto $n=0$ e ovviamente nei residui ci sarà anche quello della funzione in 0.\\
\newline\textbf{Dimostrazione (a):}\\
Consideriamo la funzione $F(z):= f(z)\text{cotan}(\pi z)$, essa ha come singolarità i poli di $f$ e i poli della contangente in $z=k=0,\pm1,\pm2,\dots$ i residui della cotangente si possono calcolare considerando il suo sviluppo di Laurent:
\[\text{cotan} \simeq \frac{1}{\pi z} + \frac{\pi z}{3} + \dots \implies \text{Res}\Big(\text{cotan}(\pi z)\Big|_{k} = \frac{1}{\pi}\]
Prendiamo allora un quadrato $Q_N$ sul piano complesso centrato in 0 e lato lungo $2(N+1)$, integrando la funzione $F$ su tale quadrato e utilizzando il teorema dei residui otteniamo:
\[I_N = \frac{1}{2\pi i} \int_{Q_N} F(z)\,dz = \frac{1}{\pi}\sum_{n = -N}^N f(n) + \sum_i \text{Res}(f(z)\text{cotan}(\pi z)\Big|_{z_i} \]
Tentiamo ora come al solito di maggiorare l'integrale con il lemma di darboux:
\[|I_N|\leq \frac{L(Q_N)}{2\pi}\underset{|z|\in Q_n}{\text{sup}}|f(z)\text{cotan}(\pi z)|\simeq \frac{4N}{2\pi} N^{-1-\alpha} \underset{|z|\in Q_n}{\text{sup}}|\text{cotan}(\pi z)|\]
Ora basta mostrare che $|\text{cotan}(\pi z)|$ è un oggetto limitato indipendentemente da $N$ e quindi per $N\to \infty$ l'integrale si annulla.
\footnotetext[1]{si trovano sulla dispensa di Cognola però}
\[|\text{cotan}(\pi z)| = \left|\frac{e^{i\pi z} + e^{-i\pi z}}{e^{i\pi z} -e^{-i\pi z}}\right| \leq \text{conti che zerbini non fa\footnotemark} \leq A\]
Abbiamo quindi nel limite che $N$ vada al'infinito
\[0 = \sum_{n = -N}^N f(n) + \pi \sum_i \text{Res}(f(z)\text{cotan}(\pi z)\Big|_{z_i}\]
da cui banalmente il risultato.\\
\newline La dimostrazione del (b) segue la stessa dimostrazione prendendo però come funzione $F=f(z)/\sin(\pi z)$ infatti il reciproco del seno ha come residui $(-1)^n/(\pi)$ e la funzione $1/\sin(\pi z)$ è sempre limitata sul quadrato $Q_N$
\section{Prolungamento analitico}
Sia $f_1$ una funzione analitica in $R_1$ e $f_2$ un altra funzione analitica in una regione $R_2$, possiamo dare la seguente definizione
\begin{dfn}
Se $f_1 = f_2$ in $R_1 \cap R_2$ allora si dice che $f_1$ è la continuazione analitica di $f_2$ o equivalentemente che $f_2$ è la continuazione analitica di $f_1$
\end{dfn}
si può dimostrare che se la continuazione analitica esiste allora è unica. Ma occupiamoci di una condizione sufficiente per l'analiticità:
\begin{thm}
Sia $g(z,t)$ continua in $z$ e analitica in $t$ per ogni $z\in R$, se la sua rappresentazione integrale
\[f(z) = \int_a^bg(z,t)\,dt\]
converge uniformenente allora $f(z)$ è analitica in $R$
\end{thm}
\hspace{-1.6em}\textbf{Dimostrazione:}\\
Dato che la convergenza è uniforme allora la funzione $f$ è continua, inoltre prendiamo una curva chiusa $\gamma$ e integriamoci $f(z)$:
\[\oint_\gamma f(z)\,dz = \oint_\gamma\int_a^bg(z,t)\,dtdz = \int_a^bdt\oint_\gamma g(z,t)\,dz = 0\]
Per il teorema di Cauchy, di conseguenza per il teorema di Morera $f(z)$ è continua.\\
\newline
grazie a questo teorema possiamo costruire un sacco di funzioni interessanti
\subsection{Funzione $\Gamma$ di Eulero}
Si definisce la funzione $\Gamma$ di Eulero come
\[\Gamma(z) =\int_0^\infty t^{z-1}e^{-t}\,dt\]
l'integrale converge per $\text{Re}(z)>0$ infatti 
\[t|t^{z-1}e^{-t}| = t^{\text{Re(z)}}e^{-t} \begin{array}{l@{\ }l}
    \raisebox{-1ex}{$\nearrow$} & 0 \quad  t\to \infty \quad \forall z \\
    \raisebox{1ex}{$\searrow$}  & 0\quad t\to 0\quad\text{Re}(z)>0
  \end{array}
\]
Possiamo trovare quindi il suo prolungamento analitico ed estendere la funzione a tutto il piano complesso, per farlo spezziamo l'integrale negli intervalli $[0,1]$ e $[1,+\infty)$, questo perchè il problema della convergenza è in $t=0$ , conviene quindi spezzare la funzione in una parte più problematica e in un pezzo intero.
\[\Gamma(z) =\int_0^\infty t^{z-1}e^{-t}\,dt = \int_0^1 t^{z-1}e^{-t}\,dt+\int_1^\infty t^{z-1}e^{-t}\,dt= \int_0^1 t^{z-1}e^{-t}\,dt+\Gamma_1(z)\]
con $\Gamma_1(z)$ funzione intera. Possiamo concentrarci ora solo sul primo pezzo sviluppando con Taylor l'esponenziale
\[\int_0^1 t^{z-1}e^{-t}\,dt = \int_0^1 t^{z-1}\sum_{k=0}^\infty \frac{(-1)^k}{k!}t^k\,dt = \sum_{k=0}^\infty \frac{(-1)^k}{k!}\int_0^1 t^{z+k-1}\,dt = \sum_{k=0}^\infty \frac{(-1)^k}{k!} \frac{1}{z+k} \]
che vale per ogni $z$, a parte nei poli della funzione in $z=-k=-1,-2,\dots$ con $k$ intero positivo ed evidentemente abbiamo che
\[\text{Res}\,\Gamma(z)\Big|_{-k} = \frac{(-1)^k}{k!}\]
in 0 la funzione diverge, ma ci torneremo tra un poco. Calcoliamo invece la funzione in $z=1$, l'integrale qui è convergente quindi non c'è bisogno di scomodare il prolungamento:
\[\Gamma(1) = \int_0^\infty e^{-t}\, dt = 1\]
più interessante è calcolarla in $z+1$
\[\Gamma(z+1) = \int_0^\infty t^{z}e^{-t}\,dt = -t^ze^{-t}\Big|_{0}^\infty + z\int_0^\infty t^{z-1}e^{-t}\, dt = z\Gamma(z)\]
nota come \emph{\textbf{prima relazione funzionale:}}
\[z\Gamma(z) = \Gamma(z+1)\]
possiamo sfruttare questa relazione per calcolare $\Gamma(-\frac{1}{2})$:
\[-\frac{1}{2}\Gamma\left(-\frac{1}{2}\right) = \Gamma\left(\frac{1}{2}\right)\]
ci conviene prima però effettuare un cambiamento di variabile nella definizione della funzione e porre $t=x^2$, $dt=2xdx$
\begin{equation}\label{eq:gamma}\Gamma(z) = \int_0^\infty t^{z-1}e^{-t}\,dt = 2\int_0^\infty x^{2z-1}e^{-x^2}\,dt\end{equation}
allora  si ha che
\[\Gamma\left(\frac{1}{2}\right) = 2\int_0^\infty e^{-x^2}\,dt = \sqrt{\pi}\]
da cui $\Gamma(-\frac{1}{2}) = -2\sqrt{\pi}$.\\
Molto interessante è il caso della funzione $\Gamma$ valutata in $n+1$, com $n\in\mathbb{N}$, infatti otteniamo:
\[\Gamma(n+1) = n\Gamma(n) = n(n-1)\Gamma(n-1) =n(n-1)(n-2)\dots1\,\Gamma(1) = n! \]
Valutiamo ora la funzione nell'origine, per farlo sfruttiamo una funzione ausiliaria:
\[g(\varepsilon) := \frac{1}{\Gamma(\varepsilon)} = \frac{\varepsilon}{\varepsilon\Gamma(\varepsilon)} =\frac{\varepsilon}{\Gamma(\varepsilon+1)} \xrightarrow[\varepsilon\to 0]{} 0\]
dai cui quindi la funzione $\Gamma$ è divergente nell'origine.\\
\newline Dimostriamo a questo punto la \emph{\textbf{formula di Stirling:}}
\[n! = \Gamma(n+1) \simeq \sqrt{2\pi n}n^n e^{-n}\quad n\to \infty \]
utlizziamo la funzione $\Gamma$:
\[\Gamma(x+1) = \int_0^\infty t^{x}e^{-t}\,dt = \int_0^\infty e^{-t+x\ln t}\,dt\]
a questo punto notiamo che la funzione $-t+x\ln t $ ha un massimo in $t=x$, possiamo quindi provare a approssimare questa funzione con il suo secondo ordine di Taylor in quel punto, questo possiamo farlo perchè il contributo maggiore all'integrale è dato dal massimo, in quanto l'esponenziale va rapidamente a 0.
\[-t+x\ln t \simeq -x + x\ln x - \frac{1}{2}\frac{(t-x)^2}{x}\]
\begin{figure}[H]
\centering
\begin{tikzpicture}
\draw [->](0,0) -- (10,0)node[below=1.5pt] {\color{black}$t$};
\draw[->](0,0)--(0,4)node[right=1.5pt] {\color{black}$y$};
\draw [red,domain=0.01:10,samples = 100] plot ({\x}, {2*2.718^(-\x+3*ln(\x))});
\draw (3,0) node[below=1.5pt] {$x$}; 
\draw[dotted] (3,0) -- (3,2.7);
\draw [blue,thick,domain=0.01:10,samples = 100] plot ({\x}, {2*2.718^(-3+3*ln(3)-((\x-3)^2)/6  )});
\end{tikzpicture}
\caption{In rosso la funzione $e^{-t+x\ln t}$, in blu la sua approssimazione con il secondo ordine di Taylor $e^{-x+x\ln x -\frac{1}{2}\frac{(t-x)^2}{x}}$, l'approssimazione diventa sempre migliore più grande è $x$}
\end{figure}
l'integrale diventa quindi:
\[\int_0^\infty e^{-t+x\ln t}\,dt = \int_0^\infty e^{-x+x\ln x -\frac{1}{2}\frac{(t-x)^2}{x}}\,dt = e^{-x+x\ln x}\int_0^\infty e^{-\frac{1}{2}\frac{(t-x)^2}{x}}\,dt \]
cambiamo di variabile $y = \frac{t-x}{\sqrt{2x}}$, $dt = dy\sqrt{2x}$ arriviamo a:
\[e^{-x+x\ln x}\sqrt{2x}\int_{-\frac{x}{\sqrt{2x}}}^\infty e^{-y^2}\, dy \xrightarrow[x\to \infty]{} e^{-x}x^x\sqrt{2x}\int_{-\infty}^\infty e^{-y^2}\, dy = e^{-x}x^x\sqrt{2\pi x}  \]
\subsection{Funzione $\beta$ di Eulero}
Un altra funzione definita tramite un integrale è la $\beta$ di Eulero strettamente imparentata con la $\Gamma$ di Eulero, inziamo a definire la funzione:
\[\beta(a,b) = \int_0^1t^{a-1}(1-t)^{b-1}\, dt \quad a,b\in \C\]
l'integrale è covergente per $\text{Re}(a)>0,\text{Re}(b)>0$. Operando un cambiamento di variabile $t = \sin^2\theta$ all'interno dell'integrale possiamo scrivere la funzione $\beta$
come:
\[ \int_0^1t^{a-1}(1-t)^{b-1} =  \int_0^{\frac{\pi}{2}} (\sin\theta)^{2a-2}(\cos\theta)^{2b-2} 2\sin\theta\cos\theta\,d\theta = 2\int_0^{\frac{\pi}{2}} (\sin\theta)^{2a-1}(\cos\theta)^{2b-1}\,d\theta \]
dimostriamo ora la relazione tra la $\beta$ e la $\Gamma$, infatti vale:
\begin{equation}\label{eq:beta}\beta(a,b) = \frac{\Gamma(a)\Gamma(b)}{\Gamma(a+b)}
\end{equation}
per dimostrarlo valutiamo la seguente utilizzando la definizione di $\Gamma$ data da \eqref{eq:gamma}:
\[\Gamma(a)\Gamma(b) = 4\int_0^\infty y^{2a-1}e^{-y^2}\,dy\int_0^\infty x^{2b-1}e^{-x^2}\,dx = 4\int_0^\infty\int_0^\infty e^{-(x^2+y^2)} x^{2b-1}y^{2a-1}\,dxdy \]
passando in coordinate polari:
\[\begin{cases}
    x =\rho \cos\theta   \\
    y=\rho \sin\theta\\
  \end{cases}\]
 arriviamo a:
 \[4\int_0^{\frac{\pi}{2}}d\theta\int_0^\infty d\rho\,\rho e^{-\rho^2}\rho^{2b-1+2a-1}(\cos\theta)^{2b-1}(\sin\theta)^{2a-1} = 4\int_0^{\frac{\pi}{2}}d\theta\,(\cos\theta)^{2b-1}(\sin\theta)^{2a-1}\int_0^\infty d\rho\,e^{-\rho^2}\rho^{2b+2a-1} \]
 \[= \underbrace{2\int_0^{\frac{\pi}{2}}d\theta\,(\cos\theta)^{2b-1}(\sin\theta)^{2a-1}}_{\beta(a,b)}\,\underbrace{2\int_0^\infty d\rho\,e^{-\rho^2}\rho^{2b+2a-1}}_{\Gamma(a+b)}\]
ovvero:
\[\Gamma(a)\Gamma(b) =\beta(a,b)\Gamma(a+b)\]
questa relazione è utile per arrivare ad un altra relazione funzionale per la $\Gamma$, prendiamo infatti il caso $a=z,b=1-z$, quindi $a+b=1$ e utilizziamo la relazione \eqref{eq:beta}:
\[\frac{\Gamma(z)\Gamma(1-z)}{1} = \beta(z,1-z) = 2\int_0^{\frac{\pi}{2}}(\cos\theta)^{-2z+1}(\sin\theta)^{2z-1}\, d\theta = 2\int_0^{\frac{\pi}{2}}(\tan\theta)^{2z-1}\,d\theta \]
cambiamo variabile con $\sqrt{x} = \tan\theta$, quindi $\theta = \text{arctan}\sqrt{x}$ da cui $d\theta = \frac{1}{1+x}\frac{1}{2\sqrt{x}}dx$
\[2\int_0^\infty x^{z-\frac{1}{2}}\frac{1}{1+x}\frac{1}{2\sqrt{x}}\,dx = \int_0^\infty \frac{x^{z-1}}{x+1}\,dx\]
l'integrale si può calcolare con il metodo dei residui, infatti sfruttando il fatto che $x^{z-1} = e^{(z-1)\ln x}$ con il teorema dei residui otteniamo:
\[I = \int_0^\infty \frac{x^{z-1}}{x+1}\,dx = \int_0^\infty \frac{e^{(z-1)\ln x}}{x+1}\,dx = 2\pi i(e^{(z-1)\ln(-1)}) = 2\pi i (e^{i\pi z} e^{-\pi i})\]
sfruttando il taglio del logaritmo:
\[(1-e^{(z-1)2\pi i}) I = 2\pi i (e^{i\pi z} e^{-\pi i}) = - 2\pi ie^{i\pi z} \]
da cui:
\[2\pi i = I(1-e^{(z-1)2\pi i}) (-e^{-i\pi z }) = I(-e^{-i\pi z }+e^{(z-1)2\pi i-i\pi z}) = I(-e^{-i\pi z }+e^{i\pi z}) \]
\[I = \frac{\pi}{\sin(\pi z)}\]
ovvero per concludere
\[\Gamma(z)\Gamma(1-z) = \frac{\pi}{\sin(\pi z)}\]
\subsection{Funzione $\zeta$ di Riemann}
Si definisce la funzione $\zeta$ mediante una serie:
\[\zeta(z) = \sum_{n=1}^\infty n^{-z} \qquad \text{Re}(z)>1\]
si può sfruttare la $\Gamma$ per prolungarla analiticamente, infatti, sfruttando un cambiamento di varibili $t = nx$,$dt=ndx$ possiamo scrivere la funzione come:
\[\Gamma(z) = \int_0^\infty t^{z-1}e^{-t}\, dt =\int_0^\infty (nx)^{z-1}e^{-nx}\,n dx = n^z\int_0^\infty x^{z-1}e^{-nx}\,dx\]
da cui
\[n^{-z} =\frac{1}{\Gamma(z)} \int_0^\infty x^{z-1}e^{-nx}\,dx\]
la funzione $\zeta$ può essere riscritta sfruttando questa relazione:
\[\zeta(z) = \sum_{n=1}^\infty \frac{1}{\Gamma(z)} \int_0^\infty x^{z-1}e^{-nx}\,dx = \frac{1}{\Gamma(z)} \int_0^\infty x^{z-1} \sum_{n=1}^\infty e^{-nx}\,dx\]
dato che $e^{-x}<1$ per ogni $x>0$ abbiamo una serie geometrica tolto il valore in $n=0$
\[\frac{1}{\Gamma(z)} \int_0^\infty x^{z-1} \sum_{n=1}^\infty e^{-nx}\,dx  = \frac{1}{\Gamma(z)} \int_0^\infty x^{z-1} \left(\frac{1}{1-e^{-x}}-1\right)\,dx \]
con semplici manipolazioni algebriche:
\[\frac{1}{1-e^{-x}}-1 = \frac{1-1+e^{-x}}{1-e^{-x}} = \frac{e^{-x}}{1-e^{-x}} \frac{e^x}{e^x} = \frac{1}{e^x-1}\]
arriviamo a:
\[\zeta(z) = \frac{1}{\Gamma(z)}\int_0^\infty \frac{t^{z-1}}{e^t-1}\,dt\]
Grazie alla funzione $\zeta$ possiamo dare un senso alle serie divergenti come $\sum_{n=1}^\infty n = 1+2+3+\dots = \zeta(-1)$, ovvero facciamo convergere serie divergenti\footnote{Queste serie vengono chiamate somme di Ramanujan}. Sembra pazzo ma a quanto pare viene sfruttato nella teoria bosonica delle stringhe per calcolare l'energia del vuoto.\\
Tentiamo allora di calcolare il valore di $\zeta(-1)$, sfruttiamo la definizione appena data della funzione e scriviamola in questo modo:
\[\zeta(z) = \frac{1}{\Gamma(z)}\int_0^\infty t^{z-2} \frac{t}{e^t-1}\,dt\]
definiamo la funzione $B(t) = \frac{t}{e^t-1}$ e sviluppiamola in serie:
\[B(t) = \frac{t}{e^t-1} \simeq \frac{t}{t+\frac{t^2}{2}+\frac{t^3}{6}+\dots} \simeq \frac{1}{1+\frac{t}{2}+\frac{t^2}{6}+\dots} \simeq 1-\frac{t}{2}-\frac{t^2}{6}+ \dots + \left(\frac{t}{2}-\frac{t^2}{6}+ \dots\right)^2 \]
\[\simeq 1-\frac{t}{2}+\frac{t^2}{12}-\frac{t^4}{720}+\dots\]
Definiamo adesso $\hat{B}(t) = \frac{t}{2} + B(t)$, osserviamo che $\hat{B}(0) = 1$ e che $\hat{B}(t)$ è una funzione pari quindi scriviamo per comodità:
\[\hat{B}(t) = 1+ \sum_{k=1}^\infty B_{2k}t^{2k}\]
con $B_{2k}$ numeri di Bernoulli: $B_2 =1/12,B_4 = -1/720$.
possiamo esprimere quindi la funzione come:
\[\zeta(z) =  \frac{1}{\Gamma(z)}\int_0^1 t^{z-2} B(t)\,dt + \frac{1}{\Gamma(z)}\int_1^\infty t^{z-2} B(t)\,dt = \frac{1}{\Gamma(z)}\int_0^1 t^{z-2} B(t)\,dt + \frac{J_1(z)}{\Gamma(z)}\]
con $\frac{J_1(z)}{\Gamma(z)}$ funzione intera. Utilizziamo adesso la funzione $\hat{B}$ da cui $B(t) = \hat{B}(t)-\frac{t}{2}$ e abbiamo:
\[\zeta(z) =  \frac{1}{\Gamma(z)}\int_0^1 t^{z-2}( 1 -\frac{t}{2}+ \sum_{k=1}^\infty B_{2k}t^{2k} ) \,dt + \frac{J_1(z)}{\Gamma(z)}\]
l'integrale a questo punto diventa immediato e otteniamo:
\[\zeta(z)  =  \frac{1}{\Gamma(z)} \left(\frac{t^{z-1}}{z-1}\bigg|_0^1 -\frac{1}{2}\frac{t}{z}\bigg|_0^1 + \sum_{k=1}^\infty B_{2k} \frac{t^{2k+z-1}}{2k+z-1}\bigg|_0^1 \right)  + \frac{J_1(z)}{\Gamma(z)}\]
\[= \frac{1}{\Gamma(z)} \left(\frac{1}{z-1}-\frac{1}{2z} + \sum_{k=1}^\infty \frac{B_{2k}}{2k+z-1}\right)  + \frac{J_1(z)}{\Gamma(z)}\]
Adesso ci siamo, prendiamo $\varepsilon$ e calcoliamo quindi il limite di $\displaystyle\lim_{\varepsilon\to0}\zeta(-1+\varepsilon)$:
\[\zeta(-1+\varepsilon) = \frac{1}{\Gamma(-1+\varepsilon)} \left(\frac{1}{-2+\varepsilon}-\frac{1}{-2+2\varepsilon} + \sum_{k=1}^\infty \frac{B_{2k}}{2k-2+\varepsilon}\right)  + \frac{J_1(-1+\varepsilon)}{\Gamma(-1+\varepsilon)}\]
i primi due termini all'interno della parentesi con $\varepsilon$ che tende a 0 si semplificano quindi non sono un problema, consideriamo il termine $\frac{J_1}{\Gamma}$:
\[\frac{J_1(-1+\varepsilon)}{\Gamma(-1+\varepsilon)}\frac{(-1+\varepsilon)}{(-1+\varepsilon)} =\frac{J_1(-1+\varepsilon)(-1+\varepsilon)}{\Gamma(\varepsilon)}\frac{\varepsilon}{\varepsilon} = \frac{J_1(-1+\varepsilon)(-1+\varepsilon)\varepsilon}{\Gamma(\varepsilon +1)} \xrightarrow[\varepsilon\to 0]{} 0\]
dato che $J_1(z)$ è un integrale convergente per ogni $z.$ Consideriamo quindi il termine con la sommatoria facendo lo stesso trucco:
\[\sum_{k=1}^\infty \frac{B_{2k}}{\Gamma(-1+\varepsilon)(2k-2+\varepsilon)} = \sum_{k=1}^\infty \frac{B_{2k}(-1+\varepsilon)}{(-1+\varepsilon)\Gamma(-1+\varepsilon)(2k-2+\varepsilon)} =  \sum_{k=1}^\infty \frac{B_{2k}(-1+\varepsilon)}{\Gamma(\varepsilon)(2k-2+\varepsilon)}\frac{\varepsilon}{\varepsilon}\]\[ = \sum_{k=1}^\infty \frac{B_{2k}(-1+\varepsilon)\varepsilon}{\Gamma(\varepsilon+1)(2k-2+\varepsilon)}\]
adesso è facile vedere che se $k=1$, l'$\varepsilon$ al denominatore si semplifica con quello al numeratore, mentre per tutti i restanti $k$ questo non succede e quindi nel limite di $\varepsilon$ che tende a 0, sopravvive e da l'unico contributo il termine con $k=1$ pari a $-B_2$, da cui infine:
\[\zeta(-1+\varepsilon)\xrightarrow[\varepsilon\to 0]{} -B_2 = -\frac{1}{12} = \sum_{n=1}^\infty n\]
\section{Trasformate}
La spazio $\R^3$ è uno spazio vettoriale funzionale in cui abbiamo a disposizione una base vettoriale data da: $\vettore{e_1},\vettore{e_2},\vettore{e_3}$. E naturalmente vale $\vettore{e_i}\cdot \vettore{e_2} = \delta_j^i$. È definito anche un prodotto scalare tra due vettori $\vettore{a}\cdot\vettore{b}$ e infine possiamo rappresentare ogni vettore dello spazio con le sue componenti $\vettore{x} = x^k \vettore{e_k}$ (con il simbolo di sommatoria implicito secondo la convenzione di Einstein). Tutto questo può essere fatto in perfetta analogia con le funzioni, prendiamo in particolare una classe di funzioni periodica ovvero che vale
\[f(x) = f(x+2l)\]
con $2l$ il periodo della funzione. Definiamo quindi il prodotto scalare:
\[f\cdot g := (f,g) = \int_l^l \overline{f(x)}g(x)\,dx\] 
come sistema ortonormale quindi come base possiamo prendere:
\[\vettore{e_i} = e_n(x) = \frac{e^{i\frac{\pi}{l}nx}}{\sqrt{2l}}\]
infatti se facciamo il prodotto scalare tra due componenti della base:
\[(e_n,e_m) = \frac{1}{2l}\int_l^l e^{-\frac{\pi}{l}(n-m)xi}\,dx \begin{array}{l@{\ }l}
    \raisebox{-1ex}{$\nearrow$} & 1 \quad  n=m\\
    \raisebox{1ex}{$\searrow$}  & 0\quad n\neq m\\
  \end{array}\]
definiamo quindi i coefficienti di Fourier e la sua serie:
\[f_n = (e_n,f) = \frac{1}{\sqrt{2\pi}}\int_l^l e^{-i\frac{\pi}{l}nx}f(x)\,dx\qquad \sum_n f_n e_n(x)\]
nel caso $l = \pi$ abbiamo la classica serie di Fourier. Tenendo conto che vale:
\[f(x) = \sum_{-\infty}^\infty f_n \frac{e^{i\frac{\pi}{l}nx}}{\sqrt{2l}} \qquad f_n = \frac{1}{\sqrt{2\pi}}\int_l^l e^{-i\frac{\pi}{l}nx}f(x)\,dx\]
Studiamo il problema della convergenza della serie, ricordiamo le seguenti definizioni e risultati:
\begin{itemize}
\item \textbf{Convergenza puntuale:} $|f_n(x)-f(x)|<\varepsilon \quad \forall n(\varepsilon,x)$
\item \textbf{Convergenza uniforme:} $|f_n(x)-f(x)|<\varepsilon \quad \forall n$
\end{itemize}
\begin{thm}
(Teorema di Weierstrass) se $|g_n(x)|\leq a_n$ e $\sum_n a_n$ converge allora $g_n$ converge uniformemente.
\end{thm}
\begin{thm}
Se $f(x)$ è limitata, periodi di periodo $2l$ e continua a tratti allora:
\[f(x) = \sum_{-\infty}^\infty f_n \frac{e^{i\frac{\pi}{l}nx}}{\sqrt{2l}}\]
nei punti $x$ in cui la funzione è continua, invece nei punti in cui la funzione è discontinua la sommatoria converge alla semisomma
\[\frac{f(x_0+0^+)+f(x_0+0^-)}{2}\]
\end{thm}
\begin{thm}
Se $f$ periodica è continua, $f'$ continua a tratti, allora la serie di Fourier converge uniformemente.
\end{thm}
\subsection{Trasformata di Fourier}
Si può passare dalla serie di Fourier alla trasformata di Fourier mandando il periodo $l$ all'infinito in modo che possiamo trasformare qualsiasi funzione non necessariamente periodica. Facendo i limiti si ottiene:
\[k_n = \frac{\pi}{l}n \xrightarrow[l\to \infty]{} k\in\R\quad\Delta k =\frac{\pi}{l} \xrightarrow[l\to \infty]{} dk  \quad\sum_n \Delta k \xrightarrow[l\to \infty]{}\int_{-\infty}^\infty dk\]
per cui la serie diventa
\[f(x) = \sum_n \frac{e^{ik_nx}}{\sqrt{2l}}\frac{l}{\pi}\Delta k = \sum_n \Delta k e^{ik_nx}\frac{\tilde{f_n}}{2\pi}\]
con $\tilde{f_n} = f_n\sqrt{2l}$ e vale:
\[\tilde{f_n} = \int_{-l}^l e^{-ik_nx}f(x)\,dx\]
che prendendone il limite all'infinito otteniamo la trasformata di Fourier!
\[\tilde{f_n}\xrightarrow[l\to \infty]{} \tilde{f}(k) = \int_{-\infty}^\infty e^{-ikx}f(x)\,dx\]
e con trasformata inversa data da:
\[f(x) = \frac{1}{2\pi}\int_{-\infty}^\infty e^{ikx}\tilde{f}(x)\,dx\]
\begin{thm}
Se $f(x)\in L_1(\R)$ ovvero è convergente (nel senso di Lebesque) l'integrale:
\[\int_{-\infty}^\infty|f(x)|\,dx<+\infty\] 
allora la trasformata di Fourier $\mathcal{F}(f) = \tilde{f}(k)$ esiste ed è continua.
\end{thm}
\hspace{-1.6em}\textbf{Dimostrazione:}\\
dimostriamo che esiste ovvero che l'integrale di Fourier è convergente:
\[| \tilde{f}(k)| = \left|\int_{-\infty}^\infty e^{-ikx}f(x)\,dx \right| = \int_{-\infty}^\infty |f(x)|\,dx <+\infty \]
per dimostrare la continuità prendiamo il limite di $k\to k_0$
\[\lim_{k\to k_0} \tilde{f}(k) = \int_{-\infty}^\infty \lim_{k\to k_0}  e^{-ikx}f(x)\,dx =  \int_{-\infty}^\infty e^{-ik_0x}f(x)\,dx = \tilde{f}(k_0)\]
per il teorema della convergena dominata.
\begin{lem}
(Lemma di Riemann- Lebesque) se $f(x)\in L_1(\R)$ allora:
\[\lim_{|k|\to +\infty} \tilde{f}(k) = 0\]
\end{lem}
\hspace{-1.6em}\textbf{Dimostrazione:}\\
Per dimostrarlo è sufficiente dimostrare il lemma per le funzioni caratteristiche in un generico intervallo $[a_j,b_j]$:
\[\chi_j =\begin{cases}
    1 \qquad a_j<x<b_J  \\
   0\qquad \text{altrove}\\
  \end{cases} \]
infatti una funzione sommabile può essere vista come una somma di funzioni caratteristiche per modulate da un fattore $f = \sum_j f_j \chi_j$
\[\tilde{\chi_j}(k) = \int_{-\infty}^\infty \chi_j e^{-ikx}\,dx = \int_{a_j}^{b_j}e^{-ikx}\,dx = -\frac{e^{-ikx}}{ik}\bigg|_{a_j}^{b_j} \xrightarrow[k\to \infty]{} 0\]
\\
\newline
Ricaviamo le proprietà della trasformata di Fourier, oltre a quella ovvia di linearità.
\begin{enumerate}[a)]
\item se $f,f'\in L_1(\R)$ allora:  $\mathcal{F}(\d{f}{x}) = ik\mathcal{F}(f)$
\item $\d{\mathcal{F}(f)}{k} = \mathcal{F}(-ixf)$
\end{enumerate}
\textbf{Dimostrazione (a):}\\
\[\mathcal{F}\left(\d{f}{x}\right)= \int_{-\infty}^\infty e^{-ikx}\d{f}{x}\,dx =fe^{-ikx}\Big|_{-\infty}^\infty+ik \int_{-\infty}^\infty e^{-ikx}f(x)\,dx =  ik\mathcal{F}(f) \]
\textbf{Dimostrazione (b):}\\
\[\d{\mathcal{F}(f)}{k} =  \frac{d}{dk}\int_{-\infty}^\infty e^{-ikx}f(x)\,dx = -ix \int_{-\infty}^\infty e^{-ikx}f(x)\,dx  =  \mathcal{F}(-ixf)\]
\\
Verifichiamo la formula di inversione della trasformata di Fourier:
\[f(x) = \frac{1}{2\pi}\int_{-\infty}^\infty e^{ikx}\tilde{f}(k)\,dk\]
per farlo introduciamo $f_N(x)$
\[f_N(x) = \frac{1}{2\pi}\int_{-N}^N e^{ikx}\tilde{f}(k)\,dk\]
e vediamo cosa succede per $N$ che tende all'infinito, per definizione di trasformata abbiamo:
\[f_N(x) = \frac{1}{2\pi}\int_{-N}^N e^{ikx}\int_{-\infty}^\infty e^{-iky}f(y)\,dydk = \frac{1}{2\pi}\int_{-\infty}^\infty f(y)\,dy\int_{-N}^N e^{ikx-iky}\,dk = \]
\[\frac{1}{2\pi}\int_{-\infty}^\infty f(y) \frac{e^{ik(x-y)}}{i(x-y)}\bigg|_{-N}^N \,dy =  \frac{1}{2\pi}\int_{-\infty}^\infty f(y) \frac{e^{iN(x-y)}-e^{-iN(x-y)}}{i(x-y)}\,dy \]
Definiamo a questo punto la seguente funzione:
\[\delta_N(x) = \frac{\sin(Nx)}{\pi x} = \frac{e^{iNx}-e^{-iNx}}{2\pi ix}\]
il nostro integrale diventa semplicemente:
\[\int_{-\infty}^\infty f(y) \delta_N(x-y)\,dy = \int_{-\infty}^\infty (f(y)-f(x))\delta_N(x-y)\,dy + \int_{-\infty}^\infty f(x) \delta_N(x-y)\,dy \]
vediamo meglio come si comporta questa funzione $\delta$ al tendere di $N$ all'infinto, abbiamo che con un cambio di variabile:
\[\int_{-\infty}^\infty \delta_N(x)\,dx =\int_{-\infty}^\infty \frac{\sin(Nx)}{\pi x}\,dx  = \int_{-\infty}^\infty \frac{\sin(y)}{\pi y}\,dy = 1\]
quindi otteniamo che:
\[\int_{-\infty}^\infty (f(y)-f(x))\delta_N(x-y)\,dy +  f(x) \int_{-\infty}^\infty\delta_N(x-y)\,dy  =  f(x) + \int_{-\infty}^\infty (f(y)-f(x))\delta_N(x-y)\,dy \]
usiamo il teorema di Lagrange e scriviamo:
\[\int_{-\infty}^\infty (f(y)-f(x))\frac{\sin(N(x-y))}{\pi (x-y)}\,dy =\int_{-\infty}^\infty f'(\xi) \frac{e^{iN(x-y)}-e^{-iN(x-y)}}{2\pi i}\,dy \xrightarrow[N\to \infty]{} 0 \]
\newline A questo punto definiamo meglio uno spazio di funzioni in cui l'operatore di trasformata è definito.
\begin{dfn}
Si chiama lo spazio dellle funzioni a decrescita rapita il seguente:
\[\mathcal{S} = \{f(x)\in C^\infty(\R):\, |x^\alpha f^{(\beta)}(x)|\xrightarrow[|x|\to \infty]{} 0 \;\forall\alpha,\beta  \}  \]
\end{dfn}
allora l'operatore di trasformata è un operazioni chiusa su questo insieme, $\mathcal{F}:\mathcal{S}\to\mathcal{S}$. Possiamo allora prendere due funzioni $f,g\in \mathcal{S}$ è definire il prodotto di convuluzione:
\[(f*g)(x) = \int_{-\infty}^\infty f(x-y)g(y)\,dy = g*f\]
questa operazione è di particolare utilità perchè vale il seguente teorema
\begin{thm}
Siano $f,g\in \mathcal{S}$ allora vale:
\[\mathcal{F}(f*g) =\mathcal{F}(f)\mathcal{F}(g) \]
\end{thm}
\hspace{-1.6em}\textbf{Dimostrazione:}\\
\[\mathcal{F}(f*g) = \int_{-\infty}^\infty dy\,e^{-iky}\int_{-\infty}^\infty f(y-t)g(t)\,dt = \]
con un cambiamento di variabile $y-t = x$,$dy=dx$:
\[\int_{-\infty}^\infty e^{-ikx}e^{-ikt}\, dx \int_{-\infty}^\infty f(x)g(t)\,dt = \int_{-\infty}^\infty e^{-ikx} f(x)\, dx \int_{-\infty}^\infty e^{-ikt} g(t)\,dt =  \mathcal{F}(f)\mathcal{F}(g)\]
\begin{thm}
(Teorema di Plancherel) Siano $f,g\in \mathcal{S}$ e sia il loro prodotto scalere
\[(f,g) = \int_{-\infty}^\infty\overline{f}(x)g(x)\,dx\]
allora abbiamo che:
\[(\tilde{f},\tilde{g}) = 2\pi(f,g)\]
\end{thm}
\hspace{-1.6em}\textbf{Dimostrazione (a):}\\
prendiamo come nella formula di inversione $N$ finito e facciamone poi il limite:
\[\int_{-N}^N \overline{\tilde{f}}(k) \tilde{g}(k)\,dk = \int_{-N}^N\,dk \int_{-\infty}^\infty e^{ikx} \overline{f}(x)\, dx  \int_{-\infty}^\infty e^{-iky} g(y)\,dy = \int_{-\infty}^\infty \overline{f}(x)\, dx \int_{-\infty}^\infty g(y)\,dy \int_{-N}^Ndk \,e^{ik(x-y)}\]
\[\int_{-\infty}^\infty \overline{f}(x)\, dx \int_{-\infty}^\infty g(y)\,dy \frac{e^{iN(x-y)} - e^{-iN(x-y)}}{i(x-y)} = 2\pi\int_{-\infty}^\infty \overline{f}(x)\, dx \int_{-\infty}^\infty g(y)\delta_N(x-y)\,dy\]
che nel limite $N\to\infty$ passando sotto il segno di integrale:
\[\lim_{N\to\infty}2\pi\int_{-\infty}^\infty \overline{f}(x)\, dx \int_{-\infty}^\infty g(y)\delta_N(x-y)\,dy =  2\pi\int_{-\infty}^\infty \overline{f}(x)g(x)\, dx = (f,g) \]
\\
siccome ai fisici non piace molto quel fattore $2\pi$ allora si preferisce usare un altra convenzione in modo che quando si passa nello spazio delle trasformate di Fourier il prodotto scalare viene preservato senza quindi avere quel fastidioso fattore $2\pi$ 
\[\tilde{f}(k) = \frac{1}{\sqrt{2\pi}} \int_{-\infty}^\infty e^{-ikx}f(x)\, dx \quad f(x) = \frac{1}{\sqrt{2\pi}} \int_{-\infty}^\infty e^{ikx}\tilde{f}(k)\, dk \]
La trasformata di Fourier si può generalizzare in spazi del tipo $\R^n$ in modo da poter applicarle alle equazioni alle derivate parziali, brevemente abbiamo:
\[\tilde{f}(\vettore{k}) =  \int_{\R^n} e^{-i\vettore{k}\cdot\vettore{x}}f(\vettore{x})\, d\vettore{x} \qquad f(\vettore{x}) =  \frac{1}{(2\pi)^n}\int_{\R^n} e^{i\vettore{k}\cdot\vettore{x}}\tilde{f}(\vettore{k})\, d\vettore{k} \]
valgono i seguenti risultati:
\begin{itemize}
\item $\mathcal{F}(f*g)=\mathcal{F}(f)\mathcal{F}(g) $
\item $(\tilde{f},\tilde{g}) = (2\pi)^n(f,g)$
\item $\mathcal{F}(\vettore{\nabla}f) = i\vettore{k}\mathcal{F}(f)$
\item $\mathcal{F}(\nabla^2f) = -k^2\mathcal{F}(f)$
\end{itemize}
Dimostriamo la terza formula considerando che:
\[\vettore{\nabla}(e^{-i\vettore{k}\cdot\vettore{x}}f) = e^{-i\vettore{k}\cdot\vettore{x}}\vettore{\nabla}f + fe^{-i\vettore{k}\cdot\vettore{x}}\vettore{\nabla}(-i\vettore{k}\cdot\vettore{x}) =   e^{-i\vettore{k}\cdot\vettore{x}}\vettore{\nabla}f -i\vettore{k} fe^{-i\vettore{k}\cdot\vettore{x}} \]
da cui l'integrale:
\[\int_{\R^n} e^{-i\vettore{k}\cdot\vettore{x}}\vettore{\nabla}f(\vettore{x})\, d\vettore{x}  = \int_{\R^n} [i\vettore{k} fe^{-i\vettore{k}\cdot\vettore{x}} + \vettore{\nabla}(e^{-i\vettore{k}\cdot\vettore{x}}f)]d\vettore{x} = i\vettore{k}\mathcal{F}(f)\]
in quanto l'integrale del secondo termine è nullo.
\subsection{Trasformata di Laplace unilatera}
La trasformata di Fourier può essere generalizzata, consideriamo il fattore $e^{-ikt}$ e prendiamo un numero complesso $p = \eta +ik$ il termine esponenziale di Fourier diventa quindi $e^{-pt}$. Definiamo quindi la trasformata di Laplace unilatera di una funzione per cui vale $f(t) = 0 $ per $t<0$ come:
\[\mathcal{L}[f](p) = \hat{f}(p) := \int_0^\infty e^{-pt} f(t)\,dt \]
\begin{thm}
Se $|f(t)|\leq A e^{\gamma_0t}$ allora la sua trasformata di Laplace unilatera esiste ed è analitica per $\text{Re}(p)>\gamma_0$
\end{thm} 
\hspace{-1.6em}\textbf{Dimostrazione:}\\
\[\hat{f}(p)| = \int_0^\infty |e^{-pt}|\, |f(t)|\,dt\leq \int_0^\infty Ae^{\gamma_0t} e^{-\eta t}\,dt = \int_0^\infty Ae^{(\gamma_0-\eta)t}\,dt< \infty \]
L'analiticità si ricava facendo l'integrale su una curva chiusa e osservando che fa 0, quindi per il teorema di Morera è analitica.\\
\newline Per la trasformata di Laplace valgono le seguenti proprietà oltre alla ovvia linearità, se $f,f'$ ed $f''$ sono $\mathcal{L}$-trasformabili allora:

\begin{enumerate}[a)]
\item $\mathcal{L}[\d{f}{t}] = p\mathcal{L}[f]-f(0)$
\item $\mathcal{L}[\dd{f}{t}] = p^2\mathcal{L}[f]-pf(0)-f'(0)$
\item $\mathcal{L}[\int_0^t f(t')\,dt'] = \frac{1}{p}\mathcal{L}[f]$
\end{enumerate}
\textbf{Dimostrazione (a):}\\
dalla definizione si integra per parti e si ottiene:
\[\hat{\d{f}{t}}(p) := \int_0^\infty e^{-pt}\d{f}{t}\,dt =  f(t)e^{-pt}\Big|_0^\infty + p\int_0^\infty e^{-pt} f(t)\,dt = p\mathcal{L}[f]-f(0)  \]
\textbf{Dimostrazione (c):}\\
Sia $F(t) = \int_0^t f(t')\,dt'$, notiamo che $F(0) = 0$ e che $F'(t) = f(t)$ usiamo quindi la proprietà (a):
\[\mathcal{L}[f] = \mathcal{L}\left[\d{F}{t}\right] = p\mathcal{L}[F] = \mathcal{L}\left[\int_0^t f(t')\,dt'\right]    \]
\begin{thm}
(Teorema dell'attenuazione) Sia $h\in\C$ se $\mathcal{L}[e^{ht}f]$ esiste allora:
\[\mathcal{L}[e^{ht}f] = \mathcal{L}[f](p-h)  \]
\end{thm}
\hspace{-1.6em}\textbf{Dimostrazione:}\\
Autoevidente quando metti $e^{ht}f$ nella definizione di trasformata di Laplace
\begin{thm}
(Teorema di convoluzione) 
\[\L[f*g] = \L[f]\L[g]\]
\end{thm}
\hspace{-1.6em}\textbf{Dimostrazione:}\\
prendiamo $N$ finito e facciamone il limite che va all'infinito:
\[\L[f*g] = \lim_{N\to\infty}\int_0^N e^{-pt}f*g\,dt = \int_0^N e^{-pt}\,dt\int_0^t f(t-y)g(y)\,dy\]
per il teorema di Fubini vale che:
\[\int_0^Ndt\int_0^tdy = \int_0^Ndy\int_y^Ndt\]
quindi possiamo scrivere il nostro integrale come
\[\int_0^N e^{-pt}\,dt\int_0^t f(t-y)g(y)\,dy = \int_0^N g(y)\,dy\int_y^N f(t-y)e^{-pt}\,dt\]
cambiamo variabile $\tau = t-y$:
\[ \int_0^N g(y)\,dy\int_0^{N-y} f(\tau)e^{-p\tau}e^{-py}\,d\tau = \int_0^N g(y)e^{-py}\,dy\int_0^{N-y} f(\tau)e^{-p\tau}\,d\tau \xrightarrow[N\to \infty]{} \L[f]\L[g] \]
\\
sarebbe interessante anche avere una formula di inversione per la trasformata di Laplace, ci si può condurre in qualche modo a quella di Fourier, infatti:
\[ \hat{f}(p) := \int_0^\infty e^{-pt} f(t)\,dt =  \int_0^\infty e^{-\eta t}e^{-ikt} f(t)\,dt = \mathcal{F}(e^{-\eta t}f(t))  \]
possiamo ora sfruttare l'antitrasformata di Fourier e scrivere:
\[e^{-\eta t}f(t) = \mathcal{F}^{-1}(\hat{f}(p)) = \frac{1}{2\pi}\int_{-\infty}^\infty e^{ikx}\hat{f}(p)\,dk\]
dai cui:
\[f(t) = \frac{1}{2\pi}\int_{-\infty}^\infty e^{ikt}e^{\eta t}\hat{f}(p)\,dk = \frac{1}{2\pi}\int_{-\infty}^\infty e^{pt}\hat{f}(p)\,dk\]
cambiamo di variabile tenendo conto ch e $p = \eta + ik$,$dp = idk$:
\[f(t) = \frac{1}{2\pi i }\int_{\eta-i\infty}^{\eta+i\infty} e^{pt}\hat{f}(p)\,dp\]

\section{Equazione differenziale del secondo ordine}
Consideriamo la seguente equazione differenziale:
\[(a_2+xb_2)y'' + (a_1+xb_1)y'+(a_0+xb_0)y =0\]
Possiamo risolverla con il metodo di Laplace. Esso ci dice che proviamo a prendere:
\[y(x) = \int_{C_{ab}}e^{sx}Z(z)\,ds \quad y'(x) = \int_{C_{ab}}se^{sx}Z(z)\,ds \quad y''(x) = \int_{C_{ab}}s^2e^{sx}Z(z)\,ds\]
dove $C_{ab}$ è una generica curva sul piano complesso con estremi $a$ e $b$. Mettiamo nell'equazione differenziale e otteniamo:
\[0 = \int_{C_{ab}}[((a_0+xb_0)+(a_1+xb_1)s+(a_2+xb_2)s^2)e^{sx}Z(z)]\,ds = \int_{C_{ab}}[(A(s)+xB(s))e^{sx}Z(z)]\,ds\]
dove abbiamo definito:
\[ \begin{cases}
   A(s) := a_0+sa_1+s^2a_2  \\
   B(s) := b_0+sb_1+s^2b_2\\
  \end{cases} \]
cerchiamo ora di integrare per parti il termine $xB(s)Z(s)e^{sx}$:
\[\frac{d}{ds}(B(s)Z(s)e^{sx}) = xB(s)Z(s)e^{sx} + e^{sx}\frac{d}{ds}(B(s)Z(s))\]
l'integrale lo facciamo diventare allora:
\[\int_{C_{ab}}xB(s)e^{sx}Z(z)\,ds = B(s)Z(s)e^{sx}\Big|_a^b - \int_{C_{ab}}e^{sx}\frac{d}{ds}(B(s)Z(s))ds \]
aggiungendo il termine con la $A(s)$:
\[\int_{C_{ab}}[(A(s)+xB(s))e^{sx}Z(z)]\,ds =  B(s)Z(s)e^{sx}\Big|_a^b + \int_{C_{ab}}e^{sx}[A(s)Z(s)-\frac{d}{ds}(B(s)Z(s))]ds = 0 \]
se vogliamo che si annulli ci vogliono 2 condizioni, la prima ci fornisce le condizioni al contorno:
\[B(s)Z(s)e^{sx}\Big|_a^b = 0 \quad B(b)Z(b)e^{bx}= B(a)Z(a)e^{ax}\]
la quale ci permette di scegliere i due estremi $a,b$. La seconda condizioni è che si annulli l'integrale, ovvero si annulli l'integrando:
\[A(s)Z(s) =\frac{d}{ds}(B(s)Z(s))  \]
da cui integrando per parti:
\[\frac{A(s)}{B(s)}ds =\frac{dB(s)Z(s)}{B(s)Z(s)} \quad \int\frac{A(s)}{B(s)}ds =\int\frac{dB(s)Z(s)}{B(s)Z(s)} \quad \int\frac{A(s)}{B(s)}ds =\ln(B(s)Z(s))\]
ovvero:
\[Z(s) = \frac{e^{\int\frac{A(s)}{B(s)}}}{B(s)}\]
\section{Equazione di propagazione in una dimensione}
Si tratta di risolvere la seguente equazione differenziale
\begin{equation}
\label{eq}
\begin{cases}
    \pdd{u(x,t)}{t}-c^2\pdd{u(x,t)}{x}=0  \\
   u(x,0) = 0\\
   \pd{u}{t}(x,0) = g(x)
  \end{cases} 
\end{equation}
Applichiamo la trasformata di Fourier all'equazione per lo spazio e otteniamo
\[\mathcal{F}\left(\pdd{u(x,t)}{t}-c^2\pdd{u(x,t)}{x}\right) = 0\]
\[\pdd{\mathcal{F}(u(k,t)}{t}+c^2k^2\mathcal{F}(u(k,t)) =0\]
infatti la derivata rispetto al tempo passa fuori dalla trasformata di Fourier in quanto fatta solo rispetto allo spazio e la derivata rispetto allo spazio per proprietà della trasformata diventa un fattore $-k^2$ davanti. Scriviamo la trasformata cosi: $\mathcal{F}(u(k,t)) \equiv \tilde{u}(k,t)$ e otteniamo l'equazione
\[\pdd{\tilde{u}(k,t)}{t}+c^2k^2\tilde{u}(k,t)=0\]
definito $\omega = kc$ abbiamo
\[\pdd{\tilde{u}(k,t)}{t}+\omega^2\tilde{u}(k,t)=0\]
che è l'equazione armonica con soluzione nota:
\begin{equation}
\label{sol}
\tilde{u}(k,t) = A(k)\sin(\omega t+\phi)
\end{equation}
con $A(k)$ da determinare secondo le condizioni iniziali e $\phi=0$ per la prima condizione inziale. Per trovare $A(k)$ prendiamo la seconda condizione dell'equazione \eqref{eq} e applichiamo la trasformata di Fourier in modo da avere la condizione iniziale nello spazio di Fourier
\[\mathcal{F}\left(\pd{u}{t}(k,0)\right) = \mathcal{F}(g(k)) \qquad \pd{\tilde{u}}{t}(k,0) = \tilde{g}(k)\]
quindi prendiamo la soluzione \eqref{sol} deriviamola rispetto al tempo e imponiamo la condizione in $t=0$
\[\pd{\tilde{u}}{t}(k,t) = A(k)\omega \cos(\omega t)\]
\[\pd{\tilde{u}}{t}(k,0) =A(k)\omega = \tilde{g}(k) \]
da cui $A(k) = \tilde{g}(k)/\omega$
La nostra soluzione finale quindi è:
\[\tilde{u}(k,t) = \frac{\tilde{g}(k)}{\omega}\sin(\omega t)\]
Ora l'unica cosa che ci rimane è antitrasformare la $\tilde{u}(k,t)$ in modo da avere l'originale $u(x,t)$
\begin{equation}\label{f}
\mathcal{F}^{-1}(\tilde{u}(k,t)) = u(x,t) = \mathcal{F}^{-1}\left(\frac{\tilde{g}(k)}{\omega}\sin(\omega t)\right)
\end{equation}
Per il secondo membro utilizziamo il teorema della convoluzione, ovvero:
\[\mathcal{F}(f*\chi) = \mathcal{F}(f)\mathcal{F}(\chi)\]
infatti se nel nostro caso troviamo una $f$ e una $\chi$ in modo che: $\mathcal{F}(f) \equiv \tilde{g}(k)$ e $\mathcal{F}(\chi) \equiv \sin(\omega t)/\omega$, possiamo scrivere l'equazione \eqref{f} come:
\[u(x,t) = \mathcal{F}^{-1}\left(\mathcal{F}(f*\chi)\right) = f*\chi\]
trovare $\mathcal{F}(f) \equiv \tilde{g}(k)$ è facile, per definizione di trasformata di Fourier $f\equiv g$. Dobbiamo ora trovare una $\chi$ tale che 
\[\mathcal{F}(\chi)=\tilde{\chi} = \frac{\sin(\omega t)}{\omega}\]
Facciamo che magicamente sappiamo che è: \[\chi = \frac{1}{2c}\theta(ct-|x|)\] e verifichiamo che effettivamente la sua trasformata sia $\sin(\omega t)/\omega$
Consideriamo che la funzione $\theta(x)$ è definita come:
\[\theta(x) =\begin{cases}
    1 \qquad x>0  \\
   0\qquad x<0\\
  \end{cases}\]
quindi $\theta(ct-|x|) = 1$ quando $ct-|x|>0$ e $\theta(ct-|x|) = 0$ quando $ct-|x|<0$. Abbiamo quindi:
\[\theta(ct-|x|) = 1 \implies ct-|x|>0 \implies |x|<ct \implies -ct<x<ct\]
ovviamente allora:
\[\theta(ct-|x|) = 0 \implies x<-ct \quad\&\quad x>ct\]
Graficamente abbiamo quindi:
\begin{figure}[H]
\centering
\begin{tikzpicture}
\draw [->](-5,0) -- (5,0)node[below=1.5pt] {\color{black}$x$};
\draw[->](0,0)--(0,5)node[right=1.5pt] {\color{black}$\theta(ct-|x|)$};
\draw[blue](-5,0)--(-2.5,0)node[below=1.5pt] {\color{black}$-ct$};
\draw[blue](2.5,0)node[below=1.5pt,yshift=-3pt] {\color{black}$ct$}--(5,0);
\draw[blue](-2.5,0)--(-2.5,2.5);
\draw[blue](2.5,0)--(2.5,2.5);
\draw[blue](-2.5,2.5)--(0,2.5)node[above=1.5pt,xshift=-3pt] {\color{black}$1$}--(2.5,2.5);
\end{tikzpicture}
\end{figure}
Per definizione di trasformata abbiamo:
\[\tilde{\chi}(k) = \int_{-\infty}^{+\infty}e^{-ikx}\chi(x)\,dx =\frac{1}{2c}\int_{-\infty}^{+\infty}e^{-ikx}\theta(ct-|x|)\,dx \]
ma per come abbiamo appena mostrato essere $\theta$, l'integrando è nullo per $x<-ct$ e $x>ct$, mentre vale $e^{-ikx}$ altrimenti, ci riduciamo quindi a:
\[\frac{1}{2c}\int_{-\infty}^{+\infty}e^{-ikx}\theta(ct-|x|)\,dx = \frac{1}{2c}\int_{-ct}^{+ct}e^{-ikx}\,dx = \frac{1}{c}\left(\frac{e^{-ikx}}{-ik}\right)\bigg\vert_{-ct}^{ct}\]
\[= \frac{1}{2c}\left(\frac{e^{-ikct}-e^{ikct}}{-ik} \right) =\frac{1}{ck}\left(\frac{e^{ikct}-e^{-ikct}}{2i} \right) \]
tenendo conto che:
\[\sin(x) = \frac{e^{ix}-e^{-ix}}{2i}\]
sapendo che $\omega=ct$ otteniamo come ci aspettavamo:
\[\tilde{\chi}(k) =\frac{\sin(\omega t)} {\omega}\]
Torniamo all'equazione \eqref{f} abbiamo ottenuto che:
\[u(x,t) = g*\chi\]
che per definizione di prodotto di convuluzione:
\[u(x,t) = \int_{-\infty}^{+\infty}g(x-y)\chi(y)\,dy = \frac{1}{2c}\int_{-\infty}^{+\infty}g(x-y)\theta(ct-|y|)\,dy\]
che per le considerazioni già fatte prima:
\[\frac{1}{2c}\int_{-\infty}^{+\infty}g(x-y)\theta(ct-|y|)\,dy=\frac{1}{2c}\int_{-ct}^{+ct}g(x-y)\,dy = u(x,t)\]
\chapter{Teoria delle distribuzioni}
\section{Spazi vettoriali topologici}
In uno spazio vettoriale $V$ con campo $\C$ sono definite le operazioni di somma e prodotto per scalare:
\begin{itemize}
\item $v+w:V\times V\to V$
\item $\lambda V:V\to V$ con $\lambda\in C$ 
\end{itemize}
un esempio di spazio vettoriale a dimensione infinita è l'insieme delle funzioni continue in un intervallo $[a,b]$
\begin{dfn}
Si dice spazio metrico $E$ uno spazio vettoriale in cui è definita la distanza $d:E\times E\to E$ che soddisfa:
\begin{enumerate}[i.]
\item $d(f,g) = d(g,f)$
\item $d(f,g) = 0 \iff f=g$
\item $d(f,g)\leq d(f,h) + d(f,g)$
\end{enumerate}
\end{dfn}
Un esempio di metrica poterebbe essere in $\C$ il modulo.\\ Vediamo cosa significa una successione convergente in $E$.
Prendiamo una successione $\{f_n\}\in E$ allora si dice che questa successione converge a $f$ in $E$ se:
\[f_n \xrightarrow[n\to \infty]{E} f \iff d(f_n,f) \xrightarrow[n\to \infty]{}0 \]
si dice inoltre che una successione è di Cauchy se:
\[d(f_n,f_m) \xrightarrow[n,m\to \infty]{}0 \]
ogni successione convergente è anche di Cauchy, infatti usando la terza proprietà della metrica:
\[d(f_n,f_m) \leq d(f_n,f)+d(f_m,f)\to 0\]
non è vero il contrario però, e se succede, quindi se in uno spazio ogni successione di Cauchy è convergente allora si dice che lo spazio è completo.
\begin{thm}
(Teorema del completamento) Ogni spazio metrico $E$ non completo si può completare.
\end{thm}
\hspace{-1.4em}Un applicazione $A:E_1\to E_2$ è continua quando:
\[\forall \{f_n\}\quad f_n \xrightarrow[n\to \infty]{E_1} f \implies A(f_n) \xrightarrow[n\to \infty]{E_2} A(f) \]
ovvero se si può scambiare il limite con la funzione $A$: $\displaystyle\lim_{n\to\infty}A(f_n) = A(\displaystyle\lim_{n\to\infty} f_n)$
nel caso in cui $E_1=E_2=E$ e per l'applicazione $A$ vale:
\[d(A(f_1),A(f_2))\leq\alpha d(f_1,f_2)\]
con $\alpha\in(0,1)$ allora si dice che $A$ è una contrazione. È immediato che ogni contrazione è anche continua, infatti presa una successione $\{f_n\}$
\[d(f_n,f)\to0 \qquad d(A(f_n),A(f))\leq\alpha d(f_n,f)\]
\begin{thm}
(Teorema di Banach-Caccioppoli) Se $A:E\to E$ è completo allora esiste $f^*$ unico tale che $A(f^*) = f^*$
\end{thm}
\begin{dfn}
Si dice spazio vettoriale normato uno spazio vettoriale $V$ su $\C$ dotato di una norma $\|\cdot \|:V\to\R^+$
per cui valgono le seguenti
\begin{enumerate}[i.]
\item $\|f\| = 0\iff f=0$
\item $\|\lambda f\|= |\lambda| \|f\|$
\item $\|f+g\|\leq \|f\|+\|g\|$
\end{enumerate}
\end{dfn}
è possibile ammorbidire la prima condizione mettendo solo l'implicazione ovvero:\\ ib. $f=0\implies \|f\|=0$\\
in questo caso si parla allora di seminorma e di spazio seminormato.
\begin{thm}
per ogni $V$ normato, allora è anche uno spazio metrico con distanza
\[d(f,g) = \|f-g\|\]
\end{thm}
\hspace{-1.6em}\textbf{Dimostrazione:}\\
le prime due condizioni della distanza sono ovviamente verificate, vediamo la terza:
\[d(f,g) =  \|f-g\| =  \|f-h+h-g\|\leq  \|f-h\|+ \|h-g\| = d(f,h) + d(g,h)\]
\\
vediamo il significato di convergenza in uno spazio metrico, prediamo una successione $\{f_n\}$ si dice che converge a $f$ in $V$ se:
\[f_n  \xrightarrow[n\to \infty]{V} f\iff \|f_n-f\| \xrightarrow[n\to \infty]{} 0\]
inoltre si dice che una successione è di cauchy se:
\[ \|f_n-f_m\| \xrightarrow[n,m\to \infty]{} 0\]
di nuovo se tutte le successioni di Cauchy convergono allora si dice che lo spazio è completo, uno spazio normato completo si chiama \emph{Spazio di Banach}.\\
La norma è un applicazione continua, infatti:
\[|\,\|f_n\|-\|f\|\,|\leq \|f_n-f\|\xrightarrow[n\to \infty]{} 0 \implies \|f_n\|\xrightarrow[n\to \infty]{} \|f\|\]
un esempio di spazio di Banach è l'insieme $C([a,b])$ delle funzioni continue in $[a,b]$ dotato con la norma:
\[\|f\| = \underset{x\in[a,b]}{max}|f(x)|\]
si può definire una convergenza uniforme su tale insieme come:
\[f_n \rightrightarrows f\iff \underset{x\in[a,b]}{max}|f_n-f|\xrightarrow[n\to \infty]{} 0 \]
\begin{dfn}
Sia un applicazione lineare $A:V_1\to V_2$ con $V_1,V_2$ spazi di Banach allora si dice:
\begin{enumerate}
\item se $V_1=V_2=V$ allora $A$ si chiama operatore
\item se $V_1 = V$ e $V_2=\C$ allora $A$ si chiama funzionale lineare
\end{enumerate}
\end{dfn}
la continuità su $A$ è definiti come per gli spazi metrici ovvero $A$ è continua se presa una successione $f_n\xrightarrow[n\to \infty]{V_1} f$ allora vale:
\[A(f_n) \xrightarrow[n\to \infty]{V_2} A(f)\]
\begin{dfn}
$A$ si dice limitata se
\[\|Af\|_{V_2}\leq C_A\|f\|_{V_1}\quad \forall f\in D\subseteq V_1\]
\end{dfn}
\begin{thm}
Sia $A$ lineare $A:V_1\to V_2$  allora $A$ è continua se e solo se $A$ è limitata
\end{thm}
\hspace{-1.6em}\textbf{Dimostrazione:}\\
Iniziamo a mostrare l'implicazione diretta. Se $A$ è limitata allora
 \[\|A(f_n)-A(f)\|_{V_2} = \|A(f_n-f)\|_{V_2} \leq C_A \|f_n-f\|_{V_1}\xrightarrow[n\to \infty]{} 0 \]
L'implicazione inversa invece, se $A$ è continua poniamo per assurdo che non è limitata allora esiste $\{g_n\}$ tale che $\|A(g_n)\|_{V_2}\geq \eta\|g_n\|_{V_1}$
definiamo una nuova successione $\hat{g_n} = g_n/\|g_n\|$ la cui norma è unitaria. Definiamo un altra successione $h_n = \hat{g_n}/n$, $\|h_n\| = 1/n \xrightarrow[n\to \infty]{} 0$. A questo punto allora:
\[\|A(h_n)\|_{V_2} = \|A\left(\frac{\hat{g_n}}{n}\right)\|_{V_2} = \frac{1}{n}\|A(\hat{g_n})\|_{V_2}\geq \|g_n\|_{V_1}\xrightarrow[n\to \infty]{} \neq0\]

\section{Spazio delle distribuzioni}
Preso uno spazio vettoriale $F$ possiamo assegnarli una famiglia numerabile di seminorme
\[\|\cdot\|_\alpha:F\to \R^+\]
con la seguente prescrizione di convergenza:
\[\varphi_n \xrightarrow[n\to \infty]{F} \varphi \iff \|\varphi_n-\varphi\|_\alpha \xrightarrow[n\to \infty]{} 0 \quad \forall\alpha\]
Allora diremo che $F$ è uno \emph{spazio vettoriale numerabilmente seminormato} (SVNS). Nel caso in cui uno spazio SVNS sia anche completo allora viene detto \emph{spazio di Frechet}. Il duale di una spazio di Frechet è lo \emph{spazio delle distribuzioni}. 
\begin{dfn}
Sia $f\in F$ e $\varphi\in F$ allora $f$ si dice limitato se esiste un $\beta$ tale che
\[|f(\varphi)|\leq C_f \|\varphi\|_\beta\]
\end{dfn}
Su $F$ vale ancora che per ogni funzionale lineare vale che la continuità implica la limitatezza su $F$ e viceversa.\\
Prendiamo ora come spazio l'insieme delle funzioni infinitamente differenziabili a supporto compatto:
\[\mathcal{D} = \{\varphi(x),\varphi(x)\in C_0^\infty(\R) : \;\varphi(x)\,\text{a supporto compatto} \}\]
dotiamolo delle seguenti seminorme
\[\|\varphi\|_\alpha = \underset{k\leq\alpha}{sup}|f^{(k)}(x)|\]
dotato della seguente convergenza:
\[\varphi_n \xrightarrow[n\to \infty]{\mathcal{D}} \varphi \iff \varphi_n^{(k)}  \rightrightarrows \varphi^{(k)}\quad \forall k\]
Allora $\mathcal{D}$ è uno spazio di Frechet, il suo duale è lo spazio delle distribuzioni.
Definiamo la convergenza in $\mathcal{D}'$ come:
\[f_n \xrightarrow[n\to \infty]{\mathcal{D}'} f \iff (f_n,\varphi) \xrightarrow[n\to \infty]{} (f,\varphi)\]
Nel caso una funzione sia localmente sommabile ovvero $\int_K |f(x)|\,dx<\infty$ allora $f(x)\in\mathcal{D}'$ e vale:
\[(f,\varphi) = \int_{-\infty}^\infty f(x)\varphi(x)\,dx\]
in questo caso si dice che $f$ è un funzionale regolare. Un esempio invece di funzionale singolare è:
\begin{dfn}
Si definisce la delta di Dirac come il funzionale tale che:
\[(\delta,\varphi) = \varphi(0)\]
\end{dfn}
Banalmente si vede che la $\delta$ è limitata. Un altra distribuzione singolare è la parte principale definita come:
\[\left(P\frac{1}{x},\varphi\right) = P\int_{-\infty}^\infty \frac{\varphi(x)}{x}\, dx = P\int_{-N}^N \frac{\varphi(x)}{x}\, dx = P\int_{-N}^N \frac{\varphi(0)+x\varphi'(\xi)}{x}\, dx \]\[ =\varphi(0)P\int_{-N}^N \frac{dx}{x} + \int_{-N}^N \varphi'(\xi) \,dx  =  \int_{-N}^N \varphi'(\xi) \,dx\]
 dove si è usata la compattezza di $\varphi$ ed il teorema di Lagrange. Questa funzione è anche limitata infatti:
 \[\left|\left(P\frac{1}{x},\varphi\right)\right| = \left|\int_{-N}^N \varphi'(\xi) \,dx\right|\leq \|\varphi\|_1 \int_{-N}^N \,dx \]
dato che è limitata è anche continua.\\
Tra la parte principale di Cauchy e la delta di Dirac valgono le formule di Sokhotski:
\[ \frac{1}{x\pm i\varepsilon}  \xrightarrow[\varepsilon\to 0]{\mathcal{D}'} \frac{1}{x\pm i0} = P\frac{1}{x} \mp i\pi\delta(x) \]
per dimostrarle prendiamo una funzione test e valutiamo:
\[\left(\frac{1}{x + i\varepsilon},\varphi\right) =\int_{-N}^N \frac{\varphi(x)}{x + i\varepsilon}\, dx  = \int_{-N}^N \frac{\varphi(x)-\varphi(0)+\varphi(0)}{x+ i\varepsilon}\, dx = \varphi(0)\int_{-N}^N \frac{1}{x + i\varepsilon}\, dx + \int_{-N}^N \frac{\varphi(x)-\varphi(0)}{x+ i\varepsilon}\, dx\]
\[= \varphi(0)\int_{-N}^N \frac{x-i\varepsilon}{x^2+ \varepsilon^2}\, dx +  \int_{-N}^N \frac{x\varphi'(\xi)}{x+ i\varepsilon}\, dx = \varphi(0)\int_{-N}^N \frac{x}{x^2+ \varepsilon^2}\, dx -i\varepsilon\varphi(0)\int_{-N}^N \frac{1}{x^2+ \varepsilon^2}\, dx +  \int_{-N}^N \frac{x\varphi'(\xi)}{x+ i\varepsilon}\, dx \]
adesso il primo termine si annulla perche è dispari, l'ultimo termine per $\varepsilon\to 0$ tende alla funzione parte principale il secondo termine invece:
\[-i\varepsilon\varphi(0)\int_{-N}^N \frac{1}{x^2+ \varepsilon^2}\, dx = -i\varphi(0)\int_{-N}^N \frac{\frac{1}{\varepsilon}}{1+ \left(\frac{x}{\varepsilon}\right)^2}\, dx  = -i\varphi(0) \int_{-\frac{N}{\varepsilon}}^\frac{N}{\varepsilon} \frac{dy}{1+y^2}  \xrightarrow[\varepsilon\to 0]{} -i\pi\varphi(0) = (-i\pi\delta,\varphi) \]
da cui quindi:
\[\left(\frac{1}{x + i\varepsilon},\varphi\right) = \left(P\frac{1}{x} - i\pi\delta(x),\varphi\right)\]
Da questa formula girandola otteniamo anche:
\[\delta(x) = \frac{1}{2\pi i }\left(\frac{1}{x-i0} -\frac{1}{x+i0} \right) \qquad P\frac{1}{x} = \frac{1}{2}\left(\frac{1}{x+i0}+\frac{1}{x-i0}\right)\]
\subsection{Operazioni su distribuzioni}
Definiamo il prodotto di una distribuzione per una funzione $y\in C^\infty$ come
\[(yf,\varphi) = (f,y\varphi)\]
possiamo definire una traslazione:
\[(f(x-x_0),\varphi) = (f,\varphi(x-x_0))\]
il prodotto tra due distribuzioni non è definito infatti è facile arrivare ad un assurdo:
\[0 = 0P\frac{1}{x} = (x\delta(x))P\frac{1}{x} = \delta(x)xP\frac{1}{x} = \delta(x)\]
le distribuzioni si possono derivare però, infatti:
\[\left(\d{f}{x},\varphi\right) = \int_{-\infty}^\infty\d{f}{x}\varphi(x)\,dx = f(x)\varphi(x)\Big|_{-\infty}^\infty - \int_{-\infty}^\infty\d{\varphi}{x}f\,dx =-\left(f,\d{\varphi}{x}\right) \]
e vale ancora la regola di leibniz:
\[\left(\d{af}{x},\varphi\right) = -\left(f,\d{a\varphi}{x}\right)  = -(f,a'\varphi+a\varphi') = (fa' + af' ,\varphi)\]
la funzione di derivazione nel senso delle distribuzioni è lineare e continua infatti:
\[\left(\d{f_n}{x},\varphi\right) = - \left(f_n,\d{\phi}{x}\right)  \xrightarrow[n\to \infty]{\mathcal{D}'} -\left(f,\d{\phi}{x}\right) =  \left(\d{f}{x},\varphi\right)\]
Calcoliamo a titolo di esempio la derivata della funzione di Heavside:
\[(\theta',\varphi) = -(\theta,\varphi') = -\int_0^\infty\varphi'(x)\,dx = \varphi(0) = (\delta,\varphi)\]
Possiamo ora estendere il concetto di derivazione classica anche a funzioni discontinue che hanno un salto, ad esempio prendiamo una funzione che in $x_1$ ha un salto, possiamo scriverla come:
\[f(x) = A(x)\theta(x_1-x) + B(x)\theta(x-x_1)\]
la sua derivata vale:
\[f'(x) = A'(x)\theta(x_1-x)-A(x)\delta(x_1-x) + B'(x)\theta(x-x_1) + B(x)\delta(x-x_1)\]
il primo e il terzo termine ci fornisco la derivata classica che indichiamo come $\{f'(x)\}$ mentre gli altri termini ci danno;
\[f'(x)= \{f'(x)\} + \delta(x)[A(x_1)-B(x_1)] =  \{f'(x)\} + h\delta(x-x_1)\]
dove $h$ è il salto che la funzione fa nel punto $x_1$. Se la funzione ha più salti si può facilmente generalizzare come:
\[f'(x) = \{f'(x)\} +\sum_i h_i\delta(x-x_i)\]
\begin{thm}
Sia $h(x)$una funzione localmente sommabile, tale che sia $xh(x)\in L_1$ allora sia $c =\int_{-\infty}^\infty h(x)dx$ vale:
\[\frac{1}{c\varepsilon}h\left(\frac{x}{\varepsilon}\right)  \xrightarrow[\varepsilon\to 0]{\mathcal{D'}} \delta(x) \qquad \frac{n}{c}h(nx)   \xrightarrow[n\to \infty]{\mathcal{D}'}\delta(x)\] 
\end{thm}
\hspace{-1.6em}\textbf{Dimostrazione:}\\
Definiamo per comdità $h_\varepsilon = \frac{1}{c\varepsilon}h\left(\frac{x}{\varepsilon}\right)$ allora:
\[|(h_\varepsilon,\varepsilon)-(\delta,\varepsilon)| = \left|\frac{1}{c}\int_{-\infty}^\infty \frac{h\left(\frac{x}{\varepsilon}\right)\varphi(x) }{\varepsilon}\,dx -\varphi(0)\right| =\left|\frac{1}{c}\int_{-\infty}^\infty \frac{h\left(\frac{x}{\varepsilon}\right)\varphi(x) }{\varepsilon}\,dx -\frac{\varphi(0)}{c}\int_{-\infty}^\infty h(y)\,dy\right|  \]
cambiando variabile nel primo integrale $y = \frac{x}{\varepsilon}$ otteniamo:
\[\left|\frac{1}{c}\int_{-\infty}^\infty h(y)\varphi(y\varepsilon)\,dy -\frac{\varphi(0)}{c}\int_{-\infty}^\infty h(y)\,dy\right| = \left|\frac{1}{c}\int_{-\infty}^\infty h(y)[\varphi(y\varepsilon)-\varphi(0)]dy \right| = \left|\frac{1}{c}\int_{-\infty}^\infty h(y)[y\varepsilon\varphi'(\xi)]dy \right|\]
\[ \leq \frac{\varepsilon}{|c|}\int_{-\infty}^\infty |h(y)y||\varphi'(\xi)|dy \leq \frac{\varepsilon}{|c|}\|\varphi\|_1\int_{-\infty}^\infty |h(y)y|dy \xrightarrow[\varepsilon\to 0]{} 0\]
\begin{thm}
Sia $F(x)\in C^\infty$ con $F(x_k) = 0$ e  $F'(x_k)\neq 0$ allora vale:
\[\delta(F(x)) = \sum_k\frac{\delta(x-x_k)}{|F'(x_k)|}\] 
\end{thm}
\hspace{-1.6em}\textbf{Dimostrazione:}\\
\[\varphi(0) = (\delta,\varphi) = \lim_{\varepsilon\to 0} \int_{-\infty}^\infty h_\varepsilon(t)\varphi(t)\,dt\]
con un cambiamento di variabile $t = F(x)$,$dt = F'(x)dx$
\[ \lim_{\varepsilon\to 0}\int_{-\infty}^\infty h_\varepsilon(t)\varphi(t)\,dt =  \lim_{\varepsilon\to 0}\int_{-\infty}^\infty h_\varepsilon(F(x))\varphi(F(x))F'(x)\,dx\]
definiamo ora $\psi(x) =  \varphi(F(x))F'(x)$, allora:
\[ \lim_{\varepsilon\to 0}\int_{-\infty}^\infty h_\varepsilon(F(x))\varphi(F(x))F'(x)\,dx =  \lim_{\varepsilon\to 0}\int_{-\infty}^\infty h_\varepsilon(F(x))\psi(x)\,dx = (\delta(F),\psi)\]
ovvero $\varphi(0) = (\delta(F(x)),\varphi)$ ma noi sappiamo che:
\[\varphi(F(x)) = \frac{\psi(x)}{F'(x)} \implies \varphi(0) = \varphi(F(x_k)) = \frac{\psi(x_k)}{F'(x_k)} = \left(\frac{\delta(x-x_k)}{|F'(x_k)|},\psi \right) = (\delta(F(x)),\varphi)\]
\subsection{Applicazioni alle serie di Fourier}
\begin{lem}
Se $\{f_n\}$ è localmente sommabile allora 
\[f_n \rightrightarrows f \implies f_n(x) \xrightarrow[n\to \infty]{\mathcal{D}'} f(x)\]
\end{lem}
\hspace{-1.6em}\textbf{Dimostrazione:}\\
Valutiamo $|(f_n,\varphi)-(f,\varphi)|$, esso è uguale a:
\[|(f_n-f,\varphi)| = \left|\int_{-N}^N (f_n(x)-f(x))\varphi\,dx\right| \leq \|\varphi\|_0 \int_{-N}^N |f_n(x)-f(x)|\,dx \leq \varepsilon  \|\varphi\|_0 2N  \xrightarrow[n\to \infty]{} 0\]
\begin{coro}
dato che la derivazione nell'ambito delle distribuzioni è una funzione continua abbiamo che:
\[\frac{d^j f_n}{dx^j} \xrightarrow[n\to \infty]{\mathcal{D}'} \frac{d^j f}{dx^j}\]
\end{coro}
\begin{thm}
Se abbiamo in $\mathcal{D}'$ la serie $\displaystyle\sum_jf_je^{ijx}$ con $|f_j|\leq A(1+|j|)^m$ con $m\geq0$ allora la serie $\displaystyle\sum_jf_je^{ijx}$ è covergente in $\mathcal{D}'$
\end{thm}
\hspace{-1.6em}\textbf{Dimostrazione:}\\
Consideriamo la serie data da
\[G(x) = \frac{f_0 x^{m+2}}{(m+2)!}+\sum_{j\neq0} \frac{f_je^{ijx}}{(ij)^{m+2}}\]
studiamo la sua convergenza utilizzando il criterio di Weierstrasse:
\[\sum_{j\neq0}\left| \frac{f_je^{ijx}}{(ij)^{m+2}}\right| = \sum_{j\neq0}\left| \frac{f_j}{(j)^{m+2}}\right|\leq \sum_{j\neq0}\frac{A(1+|j|)^m}{(|j|)^{m+2}}<\infty\]
Allora dato che $G$ converge utilizzando il lemma convergono anche le sue derivate, in particolare converge:
\[\frac{d^{m+2}G}{dx^{m+2}} = f_0 + \sum_{j\neq0} f_je^{ijk}\]
\\
\newline Dimostriamo ora la formula Poisson-Jacobi. Consideriamo la funzione $f(x) = \frac{1}{2}-\frac{x}{2\pi}$ con $x\in[-\pi,\pi]$ estesa su periodo $2\pi$ e sviluppiamola in serie di Fourier.
\begin{figure}[H]
\centering
\begin{tikzpicture}
\draw [->](-4,0) -- (4,0)node[below=1.5pt] {\color{black}$x$};
\draw[->](0,-3)--(0,3)node[right=1.5pt] {\color{black}$y$};
\draw[red,domain=0:6.28] plot ({\x*0.5},{4*(0.5-\x/6.28)});
\draw[red,domain=0:6.28] plot ({0.5*(\x-6.28)},{4*(0.5-\x/6.28)});
\draw(0,2)node[left=1.5pt]{\scriptsize$\frac{1}{2}$};
\draw(0,-2)node[right=1.5pt]{\scriptsize$-\frac{1}{2}$};
\draw(1.57,0)node[below=1.5pt]{\scriptsize$\pi$};
\draw(3.14,0)node[below=1.5pt]{\scriptsize$2\pi$};
\draw(-1.57,0)node[above=1.5pt]{\scriptsize$-\pi$};
\draw(-3.14,0)node[below=1.5pt]{\scriptsize$-2\pi$};
\draw[dotted](3.14,-2)--(3.14,2);
\draw[dotted](-3.14,-2)--(-3.14,2);
\end{tikzpicture}
\caption{$f(x) = \frac{1}{2}-\frac{x}{2\pi}$ estesa periodicamente con periodo $2\pi$}
\label{serie}
\end{figure}
Lo sviluppo in serie di Fourier è del tipo:
\[f(x) = \sum_j f_j e^{ijx}\]
con i coefficienti $f_j$ dati da:
\[f_j = \frac{1}{2\pi}\int_0^{2\pi} f(x)e^{-ijx}\, dx\]
calcoliamo il primo:
\[f_0 = \frac{1}{2\pi}\int_0^{2\pi}\left(\frac{1}{2}-\frac{x}{2\pi}\right)\, dx = 0\]
lo si vede anche facilmente dalla figura \eqref{serie}. GLi altri termini invece
\[f_j = \frac{1}{2\pi}\int_0^{2\pi} \left(\frac{1}{2}-\frac{x}{2\pi}\right)e^{-ijx}\, dx  =  \frac{1}{4\pi}\int_0^{2\pi}e^{-ijx}\, dx - \frac{1}{4\pi^2}\int_0^{2\pi} xe^{-ijx}\, dx\]
il primo termine è nullo perchè l'esponenziale complesso è periodico di periodo $2\pi$ il secondo termine si integra per parti:
\[- \frac{1}{4\pi^2}\int_0^{2\pi} xe^{-ijx}\, dx = - \frac{1}{4\pi^2}\left(-\frac{xe^{-ijx}}{ij}\bigg|_0^{2\pi}+\frac{1}{ji}\int_0^{2\pi}e^{-ijx}\, dx\right) =\frac{1}{2\pi i j}\]
in quanto di nuovo l'ultimo integrale è nullo e perchè $e^{-i2j\pi} = 1$ per ogni $j$. In definitiva quindi abbiamo che la funzione si può scrivere come:
\[f(x) = \sum_{j\neq 0} \frac{e^{ijx}}{2\pi i j}\]
Prendimone la derivata, dato che la funzione è discontinua nei punti $x=2\pi j$ con un salto pari a $h=1$ dobbiamo derivarla nel senso delle distribuzioni:
\[\d{f}{x} = \frac{1}{2\pi}\sum_{j\neq 0} e^{ijx} = \bigg\{\d{f}{x}\bigg\} + \sum_j \delta(x-2\pi j) = -\frac{1}{2\pi}  + \sum_j \delta(x-2\pi j)\]
da cui quindi la \emph{Resonmation Formula}:
\[\frac{1}{2\pi}\sum_{j} e^{-ijx} = \sum_j \delta(x-2\pi j)\]
%da questa formula moltiplicando da entrambe le parti per $e^{-\lambda x^2}$ con $\lambda>0$ otteniamo l'identità di Poisson:
%\[ \frac{1}{2\pi}\sum_{j} e^{-ijx}e^{-\lambda x^2} = \sum_j \delta(x-2\pi j)e^{-\lambda x^2} \]
%dove il secondo termine diventa:
%\[\sum_j \delta(x-2\pi j)e^{-\lambda x^2} = \sum_j e^{-\lambda (2\pi j)^2} = \sum_j e^{-4\pi^2\lambda j^2} \]
%mentre il primo:
%\[\frac{1}{2\pi}\sum_{j} e^{-ijx}e^{-\lambda x^2} = \frac{1}{2\pi}\]
\section{Distribuzioni temperate}
Se definiamo la trasformata di Fourier di una distribuzione come:
\[(\F(f),\varphi) = (f,\F(\varphi))\]
allora può capitare che $\F(\varphi)$ non sia più a supporto compatto. Per questo si introduce un nuovo spazio di distribuzioni rispetto al quale l'operazione di trasformata di Fourier è chiusa.
Iniziamo a descrivere lo spazio delle funzioni test:
\[\mathcal{S} = \{f(x)\in C^\infty(\R):\, |x^\alpha f^{(\beta)}(x)|\xrightarrow[|x|\to \infty]{} 0 \;\forall\alpha,\beta  \} \]
Ora l'operatore di trasformata è: $\F:\mathcal{S}\to\mathcal{S}$. Vediamo però meglio lo spazio che abbiamo appena definito, esso è SVNS completo, quindi di Frechet non appena definiamo le seguenti seminorme:
\[\|\varphi\|_{p,r} =  \underset{r\leq p }{sup}\left|(1+|x|^2)^{\frac{p}{2}}\varphi^{(r)}(x)\right|\]
con la seguente convergenza:
\[\varphi_n \xrightarrow[n\to \infty]{\mathcal{S}} \varphi \iff \|\varphi_n(x)-\varphi(x)\|_{r,p} \xrightarrow[n\to\infty]{} 0\quad \forall p,r \]
Si può dimostrare inoltre che la Trasformata di Fourier $\F:\mathcal{S}\to\mathcal{S}$ è lineare e continua. Inoltre si vede facilmente che $\mathcal{D}\subset\mathcal{S}$.\\
A questo punto prendiamo il duale $\mathcal{S}'$, esso è lo spazio dei funzionali lineari continui su $\mathcal{S}$, e si chiama \emph{spazio delle funzioni temperate}. In $\mathcal{S}'$ $f(x)$ è limitata se esistono $r,p$ tali che:
\[|(f,\varphi)|\leq C_f \|\varphi\|_{r,p}\]
Ancora la limitatezza implica la continuità e viceversa.\\
Se $f(x)$ è una funzione a crescita lenta ovvero se:
\[\exists m :\quad \int_{-\infty}^\infty \frac{|f(x)|}{(1+|x|^2)^{m}}\,dx <\infty\]
allora $f(x)$ determina una distribuzione temperata regolare come:
\[(f,\varphi) = \int_{-\infty}^\infty f(x)\varphi(x)\,dx\]
si può vedere che tale funzione è limtata e continua infatti:
\[ \left|\int_{-\infty}^\infty f(x)\varphi(x)\,dx\right|\leq \int_{-\infty}^\infty |f(x)||\varphi(x)|\frac{(1+|x|^2)^{m}}{(1+|x|^2)^{m}}\,dx \leq \|\varphi\|_{2m,1}\int_{-\infty}^\infty |f(x)| \frac{|f(x)|}{(1+|x|^2)^{m}}\,dx = C_f\|\varphi\|_{2m,1}\]
Quindi ogni funzione a crescita lenta è una distribuzione temperate, ma non è detto che lo siano anche le funzioni sommabili, per cui:
\[\mathcal{S}'\subset\mathcal{D}'\]
Sono distribuzioni temperate anche la $\delta,P\frac{1}{x},\theta(x)$, infatti:
\[|(\delta,\varphi)| = |\varphi|\leq \|\varphi\|_{0,0}\]
\[\left|\left(P\frac{1}{x},\varphi\right)\right| = \left|\int_{-\infty}^\infty \varphi'(\xi)\,dx \right| \leq \int_{-\infty}^\infty |\varphi'(\xi)|\frac{(1+|x|^2)}{(1+|x|^2)} \,dx\leq C\|\varphi\|_{2,1}\]
in $\mathcal{S}'$ valgono ancore le stesse operazioni di $\mathcal{D}'$ come il prodotto per una funzione e la derivazione.
\subsection{Trasformata di Fourier di distribuzioni temperate}
Lo scopo dell'introduzione dello spazio $\mathcal{S}$ è stato proprio quello di permetterci di lavorare sempre nello stesso spazio usando le trasformate di Fourier. Cerchiamo allora di definire la trasformata di una distribuzione temperata.
\[(\F(f),\varphi)  = \int_{-\infty}^\infty\F(f)(k)\varphi(k)\,dk =  \int_{-\infty}^\infty dk\, \int_{-\infty}^\infty f(x)e^{-ikx}\varphi(k)\,dx = \int_{-\infty}^\infty dx\,f(x) \int_{-\infty}^\infty e^{-ikx}\varphi(k)\,dk  \]
\[ = \int_{-\infty}^\infty f(x) \F(\varphi)(x)\,dx = (f,\F(\varphi))  \]
viene cosi naturale definire la trasformata di una distribuzione in questo modo:
\[(\F(f),\varphi) = (f,\F(\varphi))\]
La trasformata $\F:\mathcal{S}'\to\mathcal{S}'$ così definita è lineare e continua, infatti:
\[(\F(f_n),\varphi) =(f_n,\F(\varphi)) \xrightarrow[n\to\infty]{} (f,\F(\varphi)) = (\F(f),\varphi) \]
Nel caso $f$ sia una funzione sommabile e a crescita lenta vale l'usuale trasformata di Fourier:
\[\tilde{f}(k) = \int_{-\infty}^\infty e^{-ikx}f(x)\,dx\qquad f(x) = \frac{1}{2\pi}\int_{-\infty}^\infty e^{ikx}\tilde{f}(x)\,dx\]
vediamo invece la trasformata della $\delta$:
\[(\F(\delta),\varphi) = (\delta,\F(\varphi)) = \tilde{\varphi}(0) = \int_{-\infty}^\infty e^{-ikx}\varphi(x)\,dx \bigg|_0 = \int_{-\infty}^\infty\varphi(x)\,dx =(1,\varphi)\]
con la formula di inversione troviamo una definizione alternativa di $\delta$
\[\delta(x) = \frac{1}{2\pi}\int_{-\infty}^\infty e^{ikx}\,dk\]
Per la trasformata valgono le seguenti proprietà:
\begin{enumerate}[a)]
\item $\F\left(\d{f}{x}\right) = ik\F(f)$
\item $\frac{d}{dx}\F(f) = \F(-ixf)$
\end{enumerate}
\textbf{Dimostrazione (a):}\\
\[\left(\F\left(\d{f}{x}\right),\varphi\right) =\left(\d{f}{x},\F(\varphi)\right) = - \left(f,\frac{d}{dx}\F(\varphi)\right) = -(f,\F(-ik\varphi)) = -(\F(f),-ik\varphi) = (ik\F(f),\varphi)\]
\textbf{Dimostrazione (b):}\\
\[\left(\d{}{x}\F(x),\varphi\right) = - \left(\F(f),\d{\phi}{x}\right) = - \left(f,\F\left(\d{\phi}{x}\right)\right) = -(f,ix\F(\varphi)) = (-ixf,\F(\varphi)) = (\F(-ixf),\varphi)\]
\\
Definiamo il prodotto di convuluzione tra due distribuzioni notando che:
\[(f*g,\varphi) = \int_{-\infty}^\infty f*g(u)\varphi(u)\,du = \int_{-\infty}^\infty\varphi(u)\,du  \int_{-\infty}^\infty f(u-y)g(y)\,dy \]
cambiamo variabile $x = u-y$, $du = dx$ otteniamo:
\[\int_{-\infty}^\infty\varphi(x+y)\,dx  \int_{-\infty}^\infty f(x)g(y)\,dy  =  \int_{-\infty}^\infty f(x)\,dx  \int_{-\infty}^\infty \varphi(x+y)g(y)\,dy = (f_x,(g_y,\varphi(x+y))) \]
il prodotto di convoluzione quindi è definito come:
\[(f*g,\varphi) = (f_x,(g_y,\varphi(x+y)\eta(y)))\]
dove $\eta$ è stata introdotta per questioni tecniche ed è una funzione $C_0^\infty$ e vale $\eta(y)=1$ se $y$ appartiene al supporto di $g$.
\begin{thm}
Se $f,g\in\mathcal{S}'$ e se $g$ è a supporto compatto allora $f*g$ esiste e $\delta$ ne è l'identità.
\end{thm}
\hspace{-1.4em}\textbf{Dimostrazione:}\\
\[(f*\delta,\varphi) = (f_x,(\delta_y,\varphi(x+y)\eta(y))) = (f_x,\varphi(x)) = (f,\varphi)\]
\begin{thm}
Se $f,g\in\mathcal{S}'$  allora valgono:
\begin{enumerate}[i.]
\item $\F(f*g) = \F(f)\F(g)$
\item $\partial_{x_i}(f*g) = (\partial_{x_i}f*g)$
\end{enumerate}
\end{thm}
\section{Equazioni differenziali con le distribuzioni}
\subsection{Soluzione fondamentale}
Consideriamo la seguente equazione differenziale:
\[L\mathcal{E} =\delta(t)\]
con $L = a_0\frac{d^n}{dt^n}+a_1(t)\frac{d^{n-1}}{dt^{n-1}}\dots a_n(t)$ operatore lineare a coefficienti non costati (a parte $a_0$). Allora $\mathcal{E}$ si dice \emph{soluzione fondamentale} di $L$ e vale:
\[\mathcal{E}_R(t) = \theta(t)Z(t)\qquad \text{Soluzione fondamentale ritardata}\]
\[\mathcal{E}_A(t) = -\theta(-t)Z(t)\qquad \text{Soluzione fondamentale anticipata}\]
con $Z(t)$ soluzione dell'equazione omogenea $LZ=0$ con le seguenti condizioni inziali:
\[\begin{cases}
    Z(0)=Z'(0)=\dots=Z^{N-2}(0) = 0\\
   Z^{N-1}(0)=\frac{1}{a_0}\\
  \end{cases} \]
La cosa si può verificare nel caso $n=2$, $L=a_0\frac{d^2}{dt^2}+a_1(t)\frac{d}{dt}+a_2(t)$ per verifica diretta, infatti:
\[\frac{d}{dt}\mathcal{E} = \delta(t)\mathcal{E}(t)+\theta(t)\frac{d}{dt}\mathcal{E} = \theta(t)\frac{d}{dt}\mathcal{E}\]
\[\frac{d^2}{dt^2}\mathcal{E} = \delta(t)\frac{d}{dt}\mathcal{E} + \theta(t)\frac{d^2}{dt^2}\mathcal{E} = \frac{\delta(t)}{a_0} + \theta(t)\frac{d^2}{dt^2}\mathcal{E}\]
da cui $L\mathcal{E} = \delta(t)$
\subsection{Metodo di Hormander e della discesa}
Sia $L = \sum_{\alpha}C_\alpha D^\alpha$ un operatore differenziale a coefficienti costanti proviamo a studiare il problema inomogeneo
\[Lu = g\]
La soluzione sarà data da:
\[u=u_0+u_1\]
dove $u_0$ è la soluzione del problema omogeneo associato, mentre $u_1$ è una soluzione particolare. Si può vedere che:
\[u_1 = \mathcal{E}*g\]
dove $\mathcal{E}$ è la soluzione fondamentale dell'operatore. Infatti:
\[Lu_1 = L(\mathcal{E}*g) =  \sum_{\alpha}C_\alpha D^\alpha (\mathcal{E}*g) = ( \sum_{\alpha}C_\alpha D^\alpha\mathcal{E})*g = L\mathcal{E}*g = \delta*g=g\]
Un problema che può sorgere è il fatto che $\mathcal{E}$ non è unicamente determinata infatti se prendiamo:
\[\mathcal{E}^* = \mathcal{E}+u_0 \implies L\mathcal{E}^* = L\mathcal{E} +Lu_0 = L\mathcal{E}\]
Per risolvere il problema di trovare la soluzione fondamentale ci sono due modi. Descriviamo per primo il cosiddetto metodo di Hormander.\\
Si tratta di prendere
\[L\mathcal{E}=\delta\]
e applicare la trasformata di Fourier nel senso delle distribuzioni temperate, in modo che otteniamo:
\[\F(L\mathcal{E}) = 1 \]
\[\F(L\mathcal{E}) = \F(\sum_{\alpha}C_\alpha D^\alpha\mathcal{E}) = \sum_{\alpha}C_\alpha \F(D^\alpha\mathcal{E}) =  \sum_{\alpha}C_\alpha (ik)^\alpha\F(\mathcal{E})\]
Definito $P(k) =\sum_{\alpha}C_\alpha (ik)^\alpha$ otteniamo:
\[\tilde{\mathcal{E}} = \frac{1}{P(k)}\]
in generale la frazione potrebbe non essere definita nel senso delle distribuzioni quindi è necessario assegnarli una certa regolarizzazione. Se invece non abbiamo troppi problemi il teorema di Hormander ci dice che esiste sempre una soluzione non unica di questa equazione.\\
\newline Il secondo modo viene chiamato \emph{metodo della discesa}. Si tratta sempre di un modo per trovare la soluzione fondamentale associata all'operatore $L$ se quest'ultimo è di tipo ellittico. Definiamo innanzitutto cos'è un operatore ellittico. Definiamo il simbolo di $L$ come:
\[\sigma(L) = e^{i\vettore{k}\cdot \vettore{x}}L(e^{-i\vettore{k}\cdot \vettore{x}})\]
se $\sigma(L)$ è una forma quadratica in $\vettore{k}$ definita allora $L$ è un operatore ellittico.\\
Il metodo della discesa dice di passare dall'operatore del calore $P =\partial_t + L$. Allora sia $k_t(\vettore{x})$ una funzione $t-$integrabile in $\R$, possiamo scrivere l'equazione:
\[Pk_t = \partial_tk_t +Lk_t= 0\]
se $k_t$ soddisfa:
\[\lim_{t\to0^+}k_t=\delta(\vettore{x})\qquad \lim_{t\to\infty}k_t = 0\]
Allora la soluzione fondamentale di $L$ è data da:
\[G(\vettore{x}) = \int_0^\infty k_t(\vettore{x})\,dt\]
infatti:
\[LG(\vettore{x}) =  \int_0^\infty Lk_t(\vettore{x})\,dt = - \int_0^\infty \partial_tk_t(\vettore{x})\,dt = \lim_{t\to0^+}k_t - \lim_{t\to\infty}k_t = \delta(\vettore{x})\]

\chapter{Spazi di Hilbert}
\section{Richiami e definizioni}
In questi appunti ci riferiremo sempre ad uno spazio vettoriale definito sul campo dei complessi $\C$ che costituirà l'insieme degli scalari.
\begin{dfn}
Uno spazio euclideo $E$ è uno spazio vettoriale munito di prodotto scalare, ovvero una funzione $(\cdot,\cdot):E\times E\to \C$ che soddisfa le seguenti
\begin{enumerate}[i.]
\item $(f,g) = \overline{(g,f)}$
\item $(f,\lambda_1g_1+\lambda_2g_2) = \lambda_1(f,g_1) + \lambda_2(f,g_2)$
\item $(f,f)\geq 0 \quad (f,f)=0\iff f=0$
\end{enumerate}
per ogni $f,g\in E$ e $\lambda_1,\lambda_2\in \C$
\end{dfn}
\leavevmode
\\Da questa definizione segue l'antilinearità del primo argomento, ovvero:
\[(\lambda_1f_1+\lambda_2f_2,g) = \overline{\lambda_1}(f_1,g) + \overline{\lambda_2}(f_2,g)\]
\begin{thm}(Disuguaglianza di Schwartz)
\begin{equation}
|(f,g)|^2 \leq (f,f)(g,g) \qquad \forall f,g
\end{equation}
\end{thm}
\hspace{-1.6em}\textbf{Dimostrazione:}\\
Dalle proprietà del prodotto scalare abbiamo che:
\[0\leq (g+\lambda f,g+\lambda f) = (g,g) + |\lambda|^2(f,f) + \lambda(g,f) + \overline{\lambda(g,f)} = (g,g) + |\lambda|^2(f,f) + 2\text{Re}[\lambda(g,f)]\]
dato che vale per ogni $\lambda$ ne possiamo scegliere uno in particolare ad esempio:
\[\lambda = \frac{(f,g)}{|(f,g)|}t\]
con $t\in \R$, notiamo allora che valgono:
\[|\lambda|^2 = t^2\qquad \lambda (g,f) = \frac{(f,g)(g,f)}{|(f,g)|}t = \frac{(f,g)\overline{(f,g)}}{|(f,g)|}t = |(f,g)|t\]
otteniamo per cui dalla disuquaglianza precedente la seguente:
\[0\leq (g,g) + t^2(f,f) + 2t|(f,g)|\]
dato che deve essere verificata per ogni valore di $t$, allora il discrimiante di questa disuguaglianza deve essere negativo, ovvero:
\[4|(f,g)|^2 - 4(f,f)(g,g)\leq 0 \implies |(f,g)|^2\leq(f,f)(g,g) \]
come volevasi dimostrare.
\begin{dfn}
\label{def:norma}
Si dice spazio vettoriale normato uno spazio vettoriale $V$ su $\C$ dotato di una norma $\|\cdot \|:V\to\R^+$
per cui valgono le seguenti
\begin{enumerate}[i.]
\item $\|f\|\geq 0 $ e $\|f\| = 0\iff f=0$
\item $\|\lambda f\|= |\lambda| \|f\|$
\item $\|f+g\|\leq \|f\|+\|g\|$
\end{enumerate}
\end{dfn}\leavevmode
\\
è possibile ammorbidire la prima condizione mettendo solo l'implicazione ovvero: $f=0\implies \|f\|=0$ in questo caso si parla allora di seminorma e di spazio seminormato.

\begin{thm}
Ogni spazio euclideo è uno spazio vettoriale normato la cui norma è data da:
\[\|f\|^2 = (f,f)\]
\end{thm}
\hspace{-1.6em}\textbf{Dimostrazione:}\\
Le condizioni i. e ii. della norma sono banalmente verificate. Vediamo invece di dimostrare la disuguaglianza triangolare:
\[\|f+g\|^2 = (f+g,f+g) = (f,f)+(g,g)+2\text{Re}[(f,g)]\leq \|f\|^2 + \|g\|^2 + 2|(f,g)|\] 
\[\leq  \|f\|^2 + \|g\|^2  + 2\|f\|\,\|g\| = (\|f\|+\|g\|)^2\]
dove nel penultimo passaggio si è usata la disuguaglianza di Schwarzt.\\
\newline 
Facciamo ora qualche esempio di spazi vettoriali su $\C$ ed euclidei che ci saranno utili in futuro.\\
$\C^n = \{z_1,z_2,\dots,z_n\}$ è evidentemente uno spazio vettoriale con la somma e il prodotto per scalare dati da
\[\lambda \in\C,z,w\in\C^n \quad \lambda z =   \{ \lambda z_1, \lambda z_2,\dots, \lambda z_n\}\quad z+w =\{z_1+w_1,z_2+w_2,\dots,z_n+w_n\} \]
Molto più interessante è l'insieme definito dall'insieme delle successioni in $\C$ come segue:
\[l_2 = \left\{\{z_n\}=(z_1,z_2,\dots) : \sum_{k=1}^{+\infty}|z_k|^2<+\infty\right\}\]
molto naturalmente possiamo dotare questo insieme di somma e prodotto per scalare come segue:
\[\lambda \in\C,z,w\in l_2 \quad \lambda z =   (\lambda z_1, \lambda z_2,\dots)\in l_2 \quad z+w =(z_1+w_1,z_2+w_2,\dots) \]
dobbiamo dimostrare però che $z+w\in l_2$, ovvero che $\displaystyle\sum_{k=1}^{+\infty}|z_k+w_k|^2<+\infty$:
\[|z_k+w_k|^2 = (z_k+w_k)(\overline{z_k}+\overline{w_k}) = |z_k|^2+|w_k|^2 + w_k\overline{z_k} + z_k\overline{w_k} = |z_k|^2+|w_k|^2 + 2\text{Re}[(z_k\overline{w_k})]\]
\[\leq |z_k|^2+|w_k|^2 + 2|z_kw_k|\leq 2|z_k|^2+2|w_k|^2 \]
dove nel'ultimo passaggio si è usata l'identità\footnote{$0\leq (|a|-|b|)^2 = |a^2| + |b|^2 - 2|a|\,|b| \implies 2|a|\,|b| \leq |a^2| + |b|^2$}:
\begin{equation}
\label{nonsochenomedarci}
|ab|\leq \frac{1}{2}(|a|^2+|b|^2)\quad \forall a,b\in\C
\end{equation}
continuiamo a lavorare con $l_2$ e ora che abbiamo dimostrato che è uno spazio vettoriale, dotiamolo di prodotto scalare definendolo come:
\begin{equation}\label{innerproduct}(z,w) = \sum_{k=1}^{+\infty} \overline{z_k}{w_k} \quad z,w\in l_2\end{equation}
il prodotto scalare è ben definito infatti grazie all'identità \eqref{nonsochenomedarci} possiamo scrivere
\[|(z,w)| = \sum_{k=1}^{+\infty}| \overline{z_k}{w_k}| \leq \sum_{k=1}^{+\infty}\frac{1}{2}|z_k| + \frac{1}{2}|w_k| <+\infty\]
quindi $l_2$ è uno spazio euclideo su $\C$ infinito dimensionale.\\Facciamo ora un altro esempio e definiamo il seguente insieme:
\[C_2(I) = \left\{f(x)\,\, \text{funzioni continue}: \int_I |f(x)|^2\,dx<+\infty\right\}\]
vediamo la struttura di spazio vettoriale verificando che le operazioni di somma e prodotto per scalare siamo chiuse rispetto a $C_2(I)$:
\[\lambda \in\C;f,g\in C_2(I) \quad \int_I|\lambda f|^2\,dx <+\infty \quad \int_I |f+g|^2\,dx \leq   \int_I 2|f|^2+2|g|^2\,dx <+\infty\]
dove si è sfruttata la disuguaglianza triangolare e l'identità \eqref{nonsochenomedarci}:
\begin{equation}
\label{disutileinseguito}
|f+g|^2 \leq (|f|+|g|)^2 = |f|^2 + |g|^2 + 2|f|\,|g| \leq 2|f|^2 + 2|g|^2
\end{equation}
dotiamo $C_2(I)$ di prodotto scalare definendo
\[(f,g) = \int_I \overline{f}(x)g(x)\,dx \quad f,g\in C_2(I) \]
vediamo che è ben definito infatti:
\[|(f,g)| = \int_I |f(x)g(x)| \,dx \leq \int_I \frac{1}{2}(|f(x)|^2+|g(x)|^2)\,dx <+\infty\]
sempre sfruttando \eqref{nonsochenomedarci}, quindi $C_2(I)$ è uno spazio euclideo infinito dimensionale. Piccola parentesi notazionale, i seguenti modi per indicare il prodotto scalare sono equivalenti:
\[(f,g)\qquad \braket{f|g}\qquad \braket{f,g}\]
useremo prevalentemente il primo, il secondo invece è molto utilizzato in fisica quando si usa la notazione di Dirac.\\
\begin{dfn}
Se abbiamo uno spazio $E$ normato una successione $\{f_n\}$ converge fortemente a $f$ quando
\[f_n \xrightarrow[n\to \infty]{E} f \iff \|f_n-f\| \xrightarrow[n\to \infty]{} 0 \]
\end{dfn}
\hspace{-1.6em}si dice inoltre che una successione è di Cauchy se:
\[\|f_n - f_m\| \xrightarrow[n,m\to \infty]{}0 \]
è facile mostrare che ogni successione convergente è anche di Cauchy. Il contrario non è sempre vero però e abbiamo la seguente definizione
\begin{dfn}
Uno spazio in cui ogni successione di Cauchy è convergente si dice completo. Uno spazio normato completo si dice spazio di Banach.
\end{dfn}
\begin{dfn}
Ogni spazio euclideo completo con la norma indotta dal prodotto scalare è uno \textbf{spazio di Hilbert} $\H$
\end{dfn}
\hspace{-1.6em}Se uno spazio non è completo esiste il teorema del completamento che ci dice che uno spazio incompleto si può sempre completare (non ci dice però in che modo).\\
\newline$l_2$ è un esempio di spazio completo quindi è anche uno spazio di Hilbert. $C_2(I)$ invece non è completo, per mostrarlo basta fornire un controesempio. Prendiamo il caso di $C_2(\R)$ e prendiamo come elemento della successione $\{f_n\}$ il seguente:
\[f_n =  \begin{cases}
    1      & \quad|x|< a\\
    e^{-n^2(|x|-a)^2} & \quad |x|\geq a\\
  \end{cases}
  \]
 verifichiamo che $\{f_n\}$ è di cauchy
 \[\|f_n-f_m\|^2 = \int_{-\infty}^{+\infty}|f_n-f_m|^2\,dx = 2\int_0^{+\infty}|f_n-f_m|^2\,dx = 2\int_a^{+\infty}(e^{-n^2(x-a)^2} - e^{-m^2(x-a)^2})^2\,dx \]
\[2\int_0^{+\infty}(e^{-n^2\rho^2} - e^{-m^2\rho^2})^2\,d\rho = 2\int_0^{+\infty}[ e^{-2n^2\rho^2} + e^{-2m^2\rho^2} - 2e^{-(n^2+m^2)\rho^2}]\,d\rho = \]
\[\sqrt{\frac{\pi}{2n^2}} + \sqrt{\frac{\pi}{2m^2}} - 2\sqrt{\frac{\pi}{n^2+m^2}}=\sqrt{\frac{\pi}{2}}\left[\frac{1}{n}+\frac{1}{m}-\frac{2\sqrt{2}}{\sqrt{n^2+m^2}}\right] \xrightarrow[n,m\to \infty]{}0\]
la successione quindi è di Cauchy, però vediamo che converge alla funzione
\[\chi =  \begin{cases}
    1      & \quad|x|< a\\
    0 & \quad |x|\geq a\\
  \end{cases}\]
infatti:
\[\|f_n-\chi\|^2 = 2\int_0^{+\infty}|f_n-\chi|^2\,dx = 2\int_a^{+\infty}(e^{-n^2(x-a)^2})^2\,dx \xrightarrow[n\to \infty]{}0\]
ma dato che $\chi$ non è continua non appartiene a $C_2(\R)$.
\begin{figure}[H]
\centering
\begin{tikzpicture}
%\begin{axis}[colormap/hot,domain=0:2,colorbar sampled,colorbar style={samples=6},point meta min=0,point meta max=10]
%\pgfplotsinvokeforeach{1,...,5}{
%	\addplot[
%	execute at begin plot visualization = {
%	\pgfplotscolormapdefinemappedcolor{\numexpr#1*10\relax}},mapped color, thick]{#1*2.718^(-(x-1)^2)};
%	}
%\end{axis}
\draw [<->](-5,0) -- (5,0)node[below=1.5pt] {\color{black}$x$};
\draw[->](0,0)--(0,4)node[right=1.5pt] {\color{black}$y$};

\draw [red,domain=-1.5:1.5,samples = 100] plot ({\x}, {3});
\draw [red,domain=1.5:5,samples = 100] plot ({\x}, {3*2.718^(-(\x-1.5)^2)});
\draw [red,domain=-5:-1.5,samples = 100] plot ({\x}, {3*2.718^(-(-\x-1.5)^2)});

\draw [blue,domain=-1.5:1.5,samples = 100] plot ({\x}, {3});
\draw [blue,domain=1.5:5,samples = 100] plot ({\x}, {3*2.718^(-4*(\x-1.5)^2)});
\draw [blue,domain=-5:-1.5,samples = 100] plot ({\x}, {3*2.718^(-4*(-\x-1.5)^2)});

\draw [green,domain=-1.5:1.5,samples = 100] plot ({\x}, {3});
\draw [green,domain=1.5:5,samples = 100] plot ({\x}, {3*2.718^(-9*(\x-1.5)^2)});
\draw [green,domain=-5:-1.5,samples = 100] plot ({\x}, {3*2.718^(-9*(-\x-1.5)^2)});

\draw [green,domain=-1.5:1.5,samples = 100] plot ({\x}, {3});
\draw [green,domain=1.5:5,samples = 100] plot ({\x}, {3*2.718^(-9*(\x-1.5)^2)});
\draw [green,domain=-5:-1.5,samples = 100] plot ({\x}, {3*2.718^(-9*(-\x-1.5)^2)});

\draw [brown,domain=-1.5:1.5,samples = 100] plot ({\x}, {3});
\draw [brown,domain=1.5:5,samples = 100] plot ({\x}, {3*2.718^(-16*(\x-1.5)^2)});
\draw [brown,domain=-5:-1.5,samples = 100] plot ({\x}, {3*2.718^(-16*(-\x-1.5)^2)});

\draw(.1,2.9) node[above]{\scriptsize$1$};

\draw(3.4,4)[red]--(3.6,4)node[right,text=black]{\scriptsize$n=1$};
\draw(3.4,3.7)[blue]--(3.6,3.7)node[right,text=black]{\scriptsize$n=2$};
\draw(3.4,3.4)[green]--(3.6,3.4)node[right,text=black]{\scriptsize$n=3$};
\draw(3.4,3.1)[brown]--(3.6,3.1)node[right,text=black]{\scriptsize$n=4$};

\draw[dotted](1.5,0)node[below]{\footnotesize$a$} -- (1.5,3);
\draw[dotted](-1.5,0)node[below,yshift=.35ex]{\footnotesize$-a$} -- (-1.5,3);

%\draw (3,0) node[below=1.5pt] {$x$}; 
%\draw[dotted] (3,0) -- (3,2.7);
%\draw [blue,thick,domain=0.01:10,samples = 100] plot ({\x}, {2*2.718^(-3+3*ln(3)-((\x-3)^2)/6  )});
\end{tikzpicture}
\caption{Funzione $f_n$ graficata per diversi valori di $n$}
\end{figure}

$C_2(\R)$ quindi non è completo ma si può completare grazie al teorema del completamento, il completamento non è banale e si può fare sostituendo all'integrale di Riemann quello di Lebesgue, si ottiene quindi l'insieme:
\[L_2(I) = \{f(x):\int_I|f(x)|^2\,dx<+\infty\} \]
dove l'integrale si intende nel senso di Lebesgue. Si incappa però in un altro problema tecnico delle funzioni nulle quasi ovunque, la norma di queste funzioni è nulla sebbene esse non sono nulle ovunque, quindi questo spazio è definito con una seminorma, si può facilmente sistemare questo problema identificando le funzioni quasi nulle ovunque con le funzioni nulle ovunque e in questo caso lo spazio $\mathcal{L}_2(I)$ diventa uno spazio di Hilbert.
\begin{thm}(Della convergenza dominata)
Se $\{f_n\}$ sono L-sommabili ($\int_I|f(x)|\,dx<+\infty$ con l'integrale inteso nel senso di Lebesgue) e se $\displaystyle\lim_{n\to+\infty} f_n(x) = f(x)$ e se esiste una $g(x)$ L-sommabile tale che $|f_n(x)|\leq g(x)$, $\forall n$ allora:
\[\lim_{n\to+\infty} \int_I f_n(x)\,dx = \int_I f(x)\,dx\]
\end{thm}
\section{Applicazioni lineari}
Siano $E,F$ due spazi vettoriali normati consideriamo l'applicazione lineare $A:E\to F$, nel caso in cui $E=F$ allora $A$ si dice \emph{operatore}, se $F=\C$ allora $A$ si dice \emph{funzionale lineare}. $A$ è continua se 
\[f_n \xrightarrow[n\to \infty]{E}f \implies Af_n \xrightarrow[n\to \infty]{F}Af\]
inoltre si dice che $A$ è limitata se esiste $C_A$ tale che
\[\|Af\|_F \leq C_A \|f\|_E\quad \forall f\in D_A\]
dove $D_A$ è il dominio di $A$
\begin{thm}
$A$ è continua $\iff$ A è limitata
\end{thm}
\hspace{-1.6em}Facciamo degli esempi di operatori. Prendiamo il caso di $E=F=\H = L_2(0,2\pi)$ e definiamo l'operatore di moltiplicazione (o di posizione) $Qf = xf(x)$ e come dominio $D_Q = \{f : Qf\in L_2(0,2\pi)\}$. Vediamo innanzitutto se è limitato:
\[\|Qf\|^2 = \int_0^{2\pi} x^2 |f(x)|^2\,dx\leq 4\pi^2 \int_0^{2\pi}|f(x)|^2\,dx\leq 4\pi^2 \|f\|^2 \quad \forall f\in D_Q\]
l'operatore è quindi limitato per cui è anche continuo. Notiamo che se avessimo preso $L_2(\R)$ l'operatore non sarebbe più stato limitato quindi nemmeno continuo. Inoltre se l'operatore è limitato non ci sono problemi con il dominio al contrario di quanto accade invece con gli operatori non limitati. Vediamone un esempio, ovvero l'operatore coniugato a $Q$, dove per operatore coniugato si intende $P$ tale che $[Q,P] = QP-PQ = i\mathbbm{1}$ (detta anche regola di quantizzazione canonica, si è usata la convenzione di $\hbar =1$).\\
$P$ è l'operatore impulso dato da $P = -i\frac{d}{dx}$, vediamo che non è continuo. Per farlo prendiamo una base ortonormale definita da:
\[e_n(x) = \frac{e^{inx}}{\sqrt{2\pi}}\quad n\in \mathbb{Z}\]
e facciamoci agire il nostro operatore $P$:
\[Pe_n = -i \frac{d}{dx}\frac{e^{inx}}{\sqrt{2\pi}} = ne_n(x)\]
in questo caso si dice che gli $e_n$ sono gli autovettori di $P$ con $n$ autovalori. Prendiamo a questo punto la seguente successione e facciamone la norma:
\[g_n = \frac{e_n}{n} \quad \|g_n\| = \frac{1}{n} \implies \|g_n\|  \xrightarrow[n\to \infty]{L_2(0,2\pi)}0 \]
Proviamo a far agire $P$ su $g_n$
\[Pg_n = \frac{1}{n}Pe_n = e_n \quad \implies \|Pg_n\| = 1\quad \forall n\]
quindi l'operatore $P$ non è ne continuo ne limitato.\\
\newline
Consideriamo ora un'applicazione $A:E\to E$, con $E$ spazio di banach, e diamo una definizione per l'applicazione inversa $A^{-1}$, vediamo innanzitutto quando un applicazione è invertibile.
\begin{dfn}
Se $Af=g$, allora $A$ si dice invertibile se e solo se esiste $f$ unica tale che $f = A^{-1}g$
\end{dfn}
\begin{thm}
Se $A^{-1}$ esiste $\iff$ $Ker(A) = \{0\}$
\end{thm}
\hspace{-1.6em}Dove $Ker(A)$ è il nucleo di $A$ definito come l'insieme $Ker(A)= \{f\in E:Af=0\}$.\\
Facciamo subito un esempio, consideriamo l'operatore di derivazione $A = \frac{d}{dx}$ nello spazio $L_2(0,1)$ con dominio $D_A=\{f,f'\in L_2(0,1)\}$.
Esso non è invertibile in quanto evidentemente il nucleo di $A$ non contiene solo lo $0$ ma anche tutte le costanti $c\in \C$. Si può però modificare leggermente il dominio con una condizione al bordo per far diventare l'operatore $A$ invertibile. Consideriamo infatti il dominio $D_A =\{f,f'\in L_2(0,1),f(0)=0\}$, in questo caso il nucleo di $A$ contiene solo lo $0$ e l'operatore diventa invertibile con inversa:
\[f = A^{-1}g\quad\text{con}\quad A^{-1}g = \int_0^xg(x)\,dx\]
Tentiamo a questo punto di formalizzare un po quello che è l'insieme delle applicazioni lineari. Prendiamo un applicazione lineare $A:E\to F$ limitata dove $E,F$ sono spazi di Banach e definiamo il seguente insieme:
\[\mathfrak{L}(E,F) =\{A:E\to F\, \text{limitate}\}\]
nel caso $F=\C$, $\mathfrak{L}(E,\C)$ si dice duale di $E$ e si indica con $E'$
\begin{thm}
$\mathfrak{L}(E,F)$ è uno spazio vettoriale normato con somma e prodotto per scalare dati da:
\[(A+B)f = Af+Bf \qquad (\lambda A)f = \lambda (Af) \quad \forall f\in E\]
e norma operatoriale data da:
\[\|A\| = \underset{f\neq0}{sup}\frac{\|Af\|}{\|f\|}\]
\end{thm}
\hspace{-1.6em}\textbf{Dimostrazione:}\\
Mostriamo per prima cosa che $A+B$ è ancora un elemento di $\mathfrak{L}(E,F)$, ovvero stabiliamo se è limitata:
\[\|(A+B)f\| = \|Af+Bf\| \leq \|Af\| + \|Bf\| \leq C_{AB}\|f\|\]
per la limitatezza di $A$ e $B$. La dimostrazione che $\lambda A$ è ancora limitata è triviale. Vediamo invece che la norma operatoria è una norma, in particolare mostriamo che vale la iii. della definizione \eqref{def:norma}. Dalla definizione di norma operatoriale abbiamo l'immediata conseguenza che $\|Af\|\leq \|A\|\|f\|$,
prendiamo quindi il seguente rapporto:
\[\frac{\|(A+B)f\|}{\|f\|}\leq \frac{\|Af\|+\|Bf\|}{\|f\|}\leq \frac{(\|A\|+\|B\|)\|f\|}{\|f\|} = \|A\|+\|B\|\]
dato che deve valere per ogni $f$ otteniamo che, data la definizione di norma operatoriale:
\[\|A+B\| = \underset{f\neq0}{sup}\frac{\|(A+B)f\|}{\|f\|}\implies \|A+B\|\leq \|A\|+\|B\|\]
\\
Possiamo anche costruire una successione di operatori lineari $\{A_n\}\in \mathfrak{L}(E,F)$, la cui convergenza viene in maniera naturale definita come segue:
\[A_n \xrightarrow[n\to \infty]{\mathfrak{L}(E,F)} A \iff \|A_N - A\|\xrightarrow[n\to \infty]{} 0 \] 
quindi, analogamente per quanto fatto per gli spazi precendenti, si possono definire il criterio di Cauchy per le successioni e quindi la completezza dello spazio. Otteniamo quindi il seguente teorema
\begin{thm}
Se $F$ è di Banach allora anche $\mathfrak{L}(E,F)$ è di Banach.
\end{thm}
\hspace{-1.6em}Definiamo a questo punto il prodotto tra operatori, prendiamo $A,B\in\mathfrak{L}(E,E)$ e definiamo il prodotto tra $A$ e $B$ come $(AB)f = A(Bf),\,\forall f\in E$. È chiaro che con questo prodotto $\mathfrak{L}(E,E)$ ha un algebra non commutativa. Dimostriamo alcune identità del prodotto:
\[\|(AB)f\| = \|A(Bf)\| \leq \|A\|\|B\|\|f\|\]
dividendo per $\|f\|$ e prendendo il $sup$ a sinistra otteniamo anche:
\[\|AB\| \leq \|A\|\|B\|\]
questo ha importanti conseguenze, perchè se prendiamo $B=A$ otteniamo che anche $A^2$ è limitato cosi come $A^n$, per cui anche un polinomio di operatori $P(A) = \sum_{k=1}^{n}c_k A^k$ è limitato. Più interessante è quando il polinomio ha infiniti termini, in questo caso bisogna studiarne la convergenza. Prendiamo infatti la successione $S_N = \sum_{k=1}^{N}c_k A^k $, vediamo quando questa successione è di Cauchy e quindi per completezza è convergente:
\[\|S_N -S_M\| = \| \sum_{k=1}^{N}c_k A^k -  \sum_{k=1}^{M}c_k A^k\| = \|\sum_{k=N+1}^{M}c_k A^k\| \leq \sum_{k=N+1}^{M} |c_k|\|A^k\| \leq \sum_{k=N+1}^{M} |c_k|\|A\|^k\]
la convergenza di quest'ultima serie dipende dall'operatore e dai coefficienti $c_k$, ad esempio se abbiamo $|c_k|=1$ la serie converge per $\|A\|<1$, o alternativamente se abbiamo i coefficienti $c_k = \frac{1}{k!}$ abbiamo la serie $\sum_{k=1}^\infty\frac{A^k}{k!}$ che corrisponde a $e^A$ e quindi non abbiamo alcun vincolo su $\|A\|$.
\section{Operatori su spazi di Hilbert}
Ricordiamo che uno spazio di Hilbert $\H$ è uno spazio euclideo completo con la norma indotta dal prodotto scalare, possiamo scrivere quindi le seguenti indentità:
\begin{equation}
\label{identitautile}
\|f+g\|^2 = (f+g,f+g)=\|f\|^2 + \|g\|^2 + 2\text{Re}(f,g)\quad\|f-g\|^2 = \|f\|^2 + \|g\|^2 - 2\text{Re}(f,g)
\end{equation}
Sommando queste due identità otteniamo l'identità del parallelogramma:
\[\|f+g\|^2 + \|f-g\|^2 = 2\|f\|^2+2\|g\|^2\]
sottraendole invece otteniamo:
\[\|f+g\|^2 - \|f-g\|^2 = 4\text{Re}(f,g)\]
\begin{thm}
Sia $E$ di Banach, allora $E$ è di Hilbert se e solo se la norma dello spazio di Banach soddisfa l'identità del parallelogramma.
\end{thm}
\hspace{-1.6em}Un altrà identità che non dimostriamo è la seguente:
\[(f,g) = \frac{1}{4}(\|f+g\|^2 - \|f-g\|^2 -i\|f+ig\|^2 + i\|f-ig\|^2)\]
e ricordiamo anche un identità più facile: $\text{Re}(f,ig) = -\text{Im}(f,g)$.
\begin{dfn}
$G$ si dice sottospazio di $\H$ se $G\subseteq \H$, $G$ varietà lineare ovvero $g_1,g_2\in G\implies \lambda_1g_1\lambda_2g_2\in G$ e $G$ chiuso, ovvero è completo.
\end{dfn}
\begin{thm}
(Della proiezione ortogonale) Sia $G$ sottospazio di $\H$ allora esiste unica la decomposizione $h = g+f$ $\forall h\in\H$ con $g\in G$ ed $f\perp G$.
Inoltre vale:
\[\|f\| =  \underset{g'\in G}{inf}\|h-g'\|\]
\end{thm}
\hspace{-1.6em}Possiamo quindi esprimere $\H$ con l'operazione di somma diretta e il complemento ortogonale di $G$:
\[\H = G \oplus G^{\perp}\]
\begin{thm}
(Di rappresentazione di Riesz)
Prendiamo un elemento del duale di $\H$: $F\in\H'$, esso ammette la rappresentazione unica:
\[F(h) = (f^*,h)\quad f^*\in\H,\forall h\in\H \]
\end{thm}
\hspace{-1.6em}\textbf{Dimostrazione:}\\
il $ker(F) = \{g:F(g)=0\}$ è un sottospazio di $\H$, per il teorema della proiezione ortogonale esiste unica $f$ tale che $h = f+g$ con $(f,g) = 0$ e $g\in Ker(F)$.
Consideriamo $g_1 = F(h)f-F(f)h$, si può calcolare subito per linearità che:
\[F(g_1) = F(F(h)f-F(f)h) = F(h)F(f)-F(f)F(h) = 0 \]
quindi evidentemente $g_1\in Ker(F)$, ovvero:
\[0=(f,g_1) = (f,F(h)f-F(f)h ) = F(h)\|f\|^2 - F(f)(f,h)\]
ovvero:
\[F(h) = \frac{F(f)}{\|f\|^2}(f,h) = (f^*,h)\]
dove abbiamo definito $f^* = \frac{F(f)}{\|f\|^2}f$.\\
\newline
Questo teorema è di notevole importanza, infatti ci permette di costruire una mappa tra lo spazio duale di $\H$ e $\H$ stesso, ovvero il duale di uno spazio di Hilbert coincide con se stesso $\H'=\H$. Uno spazio in cui accade questo si dice riflessivo.\\
Vediamo ora come, sempre grazie a questo teorema, possiamo in maniera naturale introdurre il concetto di aggiunto di un operatore. Consideriamo $A$ operatore limitato e continuo e definiamo il seguente funzionale lineare:
\[F(f) = (g,Af)\]
dato un qualsiasi vettore $g$. È immediato vedere che tale funzionale è limitato, infatti:
\[|F(f)| = |(g,Af)| \leq \|g\|\|Af\|\leq \|g\|\|A\|\|f\| = C_F \|f\|\]
essendo limitato significa che è un elemento del duale dello spazio di Hilbert, quindi possiamo sfruttare il teorema di Riesz e scrivere che esiste unico $g^*$ tale che:
\[F(f) = (g^*,f)\]
ma dato come abbiamo definito il nostro funzionale otteniamo:
\[(g^*,f) = (g,Af)\]
visto che $g^*$ è unico, esiste un operatore $A^\dagger$ tale che possiamo scrivere:
\[(A^\dagger g,f) = (g,Af)\]
e chiamiamo $A^\dagger$ \textbf{operatore aggiunto} di $A$.
\begin{lem}
Se $A$ è limitato allora anche $A^\dagger$ è limitato e vale che $\|A\|=\|A^\dagger\|$
\end{lem}
\hspace{-1.6em}\textbf{Dimostrazione:}\\
Consideriamo la quantità positiva seguente:
\[\|A^\dagger g\|^2 = |(A^\dagger g, A^\dagger g)| = |(g,AA^\dagger g)| \leq \|g\| \|AA^\dagger g\| \leq  \|g\| \|A\|\|A^\dagger g\|\]
da cui otteniamo dividendo dalle due parti per $\|A^\dagger g\|$:
\[\|A^\dagger g\|\leq \|g\| \|A\|\]
il che dimostra la limitatezza di $A^\dagger$, dividendo ancora per $\|g\|$ e dalla definizione di norma operatoriale arriviamo a:
\[\|A^\dagger \| \leq \|A\|\]
ma dato che $(A^\dagger)^\dagger = A$ allora:
\[\|A \| \leq \|A^\dagger\|\]
da cui il lemma.\\
\newline
Dimostriamo due proprietà degli operatori aggiunti:
\begin{enumerate}[i.]
\item $(AB)^\dagger = B^\dagger A^\dagger$, infatti presi due qualsiasi vettori $g,f$:
\[((AB)^\dagger g,f) =(g,ABf) = (A^\dagger g,Bf) = (B^\dagger A^\dagger g,f) \]
\item $(\alpha A + \beta B)^\dagger = \overline{\alpha}A^\dagger + \overline{\beta}B^\dagger$ con $\alpha,\beta\in\C$, infatti nuovamente presi due qualsiasi vettori $g,f$:
\[((\alpha A + \beta B)^\dagger g,f) = (g,(\alpha A + \beta B)f) = \alpha(g,Af) + \beta(g,Bf) = \alpha(A^\dagger g,f) + \beta(B^\dagger g,f)\]
\[ (\overline{\alpha}A^\dagger g,f) + (\overline{\beta}B^\dagger g,f)= ((\overline{\alpha}A^\dagger + \overline{\beta}B^\dagger)g,f) \]
\end{enumerate}
Fino ad adesso abbiamo parlato di operatori limitati e definito per essi l'operatore aggiunto, facciamo la stessa cosa per gli operatori non limitati e diamo la seguente definizione
\begin{dfn}
\label{aggiunto_nonlimitato}
Sia $A$ operatore non limitato con $D_A\subset\H$ e $\overline{D_A}=\H$ allora si definisce l'operatore aggiunto $A^\dagger$ l'operatore tale che valga:
\[(g,Af) = (A^\dagger g,f)\]
\end{dfn}\leavevmode
\\La condizione sul dominio è stata messa per far in modo che l'operatore aggiunto sia univocamente determinato. Diamo ora due importanti definizioni per gli operatori aggiunti.
\begin{dfn}
$A$ operatore è detto simmetrico se vale:
\[(Ag,f) - (g,Af) = 0\quad \forall f,g\in D_A\]
\end{dfn}
\begin{dfn}
$A$ si dice autoaggiunto (hermitiano per i fisici) se $A$ è simmetrico e $D_{A^\dagger} = D_A$, scriveremo che $A^\dagger = A$
\end{dfn}
\begin{thm}
Se $A^\dagger = A $ in $\H$ allora valgono le seguenti:
\begin{enumerate}[i.]
\item $(\psi,A\psi) \doteq \braket{\psi}$ è una quantità reale
\item se $\varphi_i,\lambda_i$ esistono tali che $A\varphi_i = \lambda_i\varphi_i$ allora i $\lambda_i$ sono reali.
\item Se $\lambda_i\neq \lambda_j$ allora $(\varphi_i,\varphi_j)=0$
\end{enumerate}
\end{thm}
\hspace{-1.6em}\textbf{Dimostrazione:}
\begin{enumerate}[i.]
\item $\overline{(\psi,A\psi)} = (A\psi,\psi) = (\psi,A\psi)$ il complesso coniugato è uguale a se stesso quindi è una quantità reale.
\item $A\varphi_i = \lambda_i\varphi_i \implies (\varphi_i,A\varphi_i) =(\varphi_i, \lambda_i\varphi_i) = \lambda_i\|\varphi_i\|^2 \implies \lambda_i = \frac{(\varphi_i,A\varphi_i)}{\|\varphi_i\|^2} $, ma dato che la quantità al numeratore è reale per il punto i. di questo teorema e il denominatore è reale per definizione di norma allora $\lambda_i$ è reale.
\item $A\varphi_i = \lambda_i\varphi_i \implies (\varphi_j,A\varphi_i) = (\varphi_j,\lambda_i\varphi_i)$ ma anche $A\varphi_j = \lambda_j\varphi_j \implies (A\varphi_j,\varphi_i) = (\lambda_j\varphi_j,\varphi_i) \implies (\varphi_j,A\varphi_i) = (\lambda_j\varphi_j,\varphi_i) $
facendo la differenza tra la prima equazione e la seconda otteniamo:
\[0 = \lambda_i(\varphi_j,\varphi_i)- \lambda_j(\varphi_j,\varphi_i) = (\lambda_i-\lambda_j)(\varphi_j,\varphi_i) \]
da cui è chiaro che se $\lambda_i\neq \lambda_j$ allora $(\varphi_i,\varphi_j)=0$.
\end{enumerate}\leavevmode
\newline
Facciamo allora degli esempi sugli operatori autoaggiunti. Consideriamo lo spazio $L_2(0,2\pi)$ e l'operatore di moltiplicazione $Qf = xf(x)$ già incontrato in precedenza. Abbiamo già dimostrato che è limitato quindi il dominio sarà $D_Q =\H$, vediamo se è anche simmetrico:
\[(Qf,g)-(f,Qg) = \int_0^{2\pi}x\overline{f}(x)g(x)\,dx - \int_0^{2\pi}\overline{f}(x)xg(x)\,dx = 0\]
quindi è simmetrico, inoltre $D_Q^{\dagger} = D_Q =\H$ quindi $Q$ è autoaggiunto.\\Vediamo invece se l'operatore $P = -i\frac{d}{dx}$ con dominio $D_P = \{f,f'\in L_2(0,2\pi)\}$ è simmetrico:
\[(Pf,g) - (f,Pg) = \int_0^{2\pi}\overline{(-if')}(x)g(x)\,dx - \int_0^{2\pi}-\overline{f}(x)ig'(x)\,dx =i\int_0^{2\pi}[\overline{f'}(x)g(x)+\overline{f}(x)g'(x)]dx\]
\[ i\int_0^{2\pi}\frac{d}{dx}(\overline{f}g)dx = i\big(\overline{f}g\big|_0^{2\pi}\]
vediamo che allora affinchè $P$ sia simmetrico il suo dominio deve essere $D_P = \{f,f'\in L_2(0,2\pi),f(0) = f(2\pi) = 0\}$. Con questo dominio $P$ è simmetrico ma non è autoaggiunto, infatti il dominio di $P^\dagger$ che è $D_{P^\dagger} = \{f,  i\big(\overline{f}g\big|_0^{2\pi} = 0,\forall g\in D_P\} $ ha la condizione al contorno automaticamente soddisfatta dal dominio di $P$ quindi $D_{P^\dagger} = \{f,f'\in L_2(0,2\pi)\}\supset D_P$ quindi $P$ non è autoaggiunto.\\ Modifichiamo però il dominio di $P$ con una condizione al contorno e prendiamo $D_P = \{f,f'\in L_2(0,2\pi),f(0) = f(2\pi)\}$, con questo dominio $P$ è simmetrico, inoltre se prendiamo la condizione al contorno del dominio di $P^\dagger$ abbiamo che:
\[\overline{f}(2\pi)g(2\pi)-\overline{f}(0)g(0)=0\implies [\overline{f}(2\pi) - \overline{f}(0)]g(0) = 0\implies \overline{f}(2\pi) = \overline{f}(0)\implies f(0)=f(2\pi)\]
che è esattamente la condizione al contorno del dominio di $P$, quindi i domini coincidono e $P$ è di conseguenza autoaggiunto.
\subsection{Paradosso del commutatore}
La scelta del dominio è fondamentale quando si parla di operatori non limitati, infatti come abbiamo visto poco fa un dominio diverso può trasformare un operatore da autoaggiunto a non. Un'altro esempio dove il dominio risultata di primaria importanza è il cosiddetto paradosso del commutatore. Prendiamo i soliti operatori $Q$ e $P$ e il loro commutatore $C = [Q,P] = QP-PQ$, e dato che l'operatore $P$ non è limitato lavoriamo nel suo dominio $D_P = \{f,f'\in L_2(0,2\pi),f(0) = f(2\pi)\}$, dove abbiamo dimostrato che è anche autoaggiunto, per non avere problemi. Il commutatore come abbiamo già visto si comporta n questo modo:
\[Cf = QPf-PQf = -iQ(f')x - P(xf(x)) = -ixf'(x) + if(x) + ixf'(x) = if(x)\]
difatti avevamo definito così il commutatore nel paragrafo 1.2 parlando degli operatori coniugati.\\
Degli elementi di $D_P$ sono ad esempio:
\[e_n = \frac{e^{inx}}{\sqrt{2\pi}}\quad n\in\mathbb{Z}\]
facciamo agire il commutatore su questi autovettori:
\[Ce_n = ie_n \implies (e_n, Ce_n) = i(e_n,e_n) = i\]
inseriamo dentro la definizione di commutatore e svolgiamo i conti:
\begin{align*} 
(e_n, Ce_n) &=  i \\ 
(e_n, QPe_n-PQe_n) &= i\\ 
(e_n, nQe_n)-(e_n,PQe_n) &= i\\ 
n(e_n, Qe_n)-(Pe_n,Qe_n) &= i\\ 
n(e_n, Qe_n)-n(e_n,Qe_n) &= i\\
0&= i \quad \forall n
\end{align*}
evidentemente c'è qualche problema e come già anticipato il problema sta nel dominio del commuatore. $P$ non è limitato, per cui ci siamo ristretti al suo dominio, però nel commutatore si trova $PQ$, ora l'operatore $Q$ non ha problemi di dominio, mentre $P$ si, può capitare quindi che $Qf$ non appartenga al dominio di $P$ nonostante $f$ ci appartenga. Un esempio è $e_n$ come abbiamo mostrato, è evidente che nel nostro dominio di funzioni $2\pi$ periodiche dopo aver fatto agire l'operatore $Qf$ si ottiene $xf(x)$ che non soddisfa più la condizione al contorno. Ci sono due modi per sistemare il problema, uno è banalmente quello di modificare il dominio del commutatore e prendere:
\[D_C = \{f,f'\in L_2(0,2\pi),xf(x)\,\text{periodica in}\, (0,2\pi)\, \text{con}\, f(2\pi) = 0\}\]
Un altro modo è quello di scomodare le distribuzioni e definire la funzione $\Phi(x) = x$ su $(0,2\pi)$ estesa periodicamente.
\begin{figure}[H]
\centering
\begin{tikzpicture}
\draw [->](-4,0) -- (4,0)node[below=1.5pt] {\color{black}$x$};
\draw[->](0,0)--(0,4)node[right=1.5pt] {\color{black}$y$};

\draw[red,domain=0:3.14] plot ({\x},{\x)});
\draw[red,domain=0:3.14] plot ({(\x-3.14)},{\x});
\draw(0,3.14)node[left=1.5pt]{\scriptsize$2\pi$};
%\draw(0,-2)node[right=1.5pt]{\scriptsize$-\frac{1}{2}$};
%\draw(1.57,0)node[below=1.5pt]{\scriptsize$\pi$};
\draw(3.14,0)node[below=1.5pt]{\scriptsize$2\pi$};
%\draw(-1.57,0)node[above=1.5pt]{\scriptsize$-\pi$};
\draw(-3.14,0)node[below=1.5pt]{\scriptsize$-2\pi$};
\draw[dotted](3.14,0)--(3.14,3.14);
\draw[dotted](-3.14,0)--(-3.14,3.14);
\end{tikzpicture}
\caption{La funzione $\Phi(x)$}
\end{figure}
il commutatore allora diventa:
\[C = i \left(\frac{d}{dx}\Phi(x)\right)f = i\left[1+\sum_n^{salti}h_n\delta(x-2\pi n)\right]f(x) =i\left[1-\sum_n^{salti}2\pi\delta(x-2\pi n)\right]f(x) \]
Se adesso calcoliamo il valore di aspettazione del commutatore come abbiamo fatto prima:
\[(f,Cf) = (f,i\left[1-\sum_n^{salti}2\pi\delta(x-2\pi n)\right]f(x)) = i\|f\|^2 - 2\pi i \sum_n^{salti}\int_0^{2\pi} \overline{f}(x)f(x)\delta(x-2\pi n )dx\]
\[ = i\|f\|^2 -2\pi i \overline{f}(2\pi)f(2\pi) \]
riprendiamo ora in mano il paradosso sopra e calcoliamo $Ce_n$:
\[(e_n,Ce_n) = i -2\pi i\overline{e_n(2\pi)}e_n(2\pi) = i-i = 0 \]
Per cui l'uguaglianza di prima diventa $0=0$. Notiamo che questo problema c'è l'abbiamo solo con $L_2(0,2\pi)$, con $L_2(\R)$ non avremmo questi problemi.
\section{Sistemi ortornomali}
Introduciamo i concetti di base di spazi infinito dimensionali come può essere una base dello spazio di Hilbert.
\begin{dfn}
$\H$ si dice separabile se esistono $\{g_n\}$ linearmente indipendenti e numerabili.
\end{dfn}
Un esempio di tale successione in $L_2(0,1)$ potrebbe essere: $g_0 = 1,g_1=x,g_2=x^2\dots$\\
Una successione del genere possiamo normalizzarla e farla diventare un sistema ortonormale numerabile (S.O.N) con il seguente procedimento detto di \textbf{Graham-Schmidt}. L'idea è quella di partire da una base $\{g_n\}$ e ottenere un S.O.N $\{e_n\}$, il procedimento segue i seguenti step:
\begin{itemize}
\item $g_1 \to e_1 = \frac{g_1}{\|g_1\|}$
\item $\hat{e}_{n+1} = g_{n+1}-\displaystyle\sum_{k=1}^n(g_{n+1},e_k)e_k$
\item $e_{n+1} = \frac{\hat{e}_{n+1}}{\|\hat{e}_{n+1}\|}$ 
\end{itemize}
e si iterano gli ultimi due step per tutti gli $n$.\\
Grazie ad un S.O.N $\{e_n\}$ nello spazio di Hilbert $\H$ possiamo andare a generalizzare le serie di Fourier. Consideriamo infatti un $f\in H$ e chiamiamo i coefficienti di Fourier generalizzati $f_k=(e_k,f)$ e la serie di Fourier generalizzata $\sum_k f_ke_k$, si può introdurre anche una ridotta di Fourier generalizzata come $f_N = \displaystyle \sum_{k=1}^Nf_ke_k$. Ora quello che vogliamo fare è studiare la convergenza della ridotta di Fourier generalizzata e se l'elemento a cui converge è effettivamente $f$, questo fatto è legato alla completezza (alla densità) del sistema ortonormale $\{e_n\}$. Quello che vogliamo vedere è se avvengono le seguenti:
\begin{enumerate}
\item $f_N \xrightarrow[n\to \infty]{\H} f^* $ ?  $f^*=\displaystyle \sum_{k=1}^\infty f_ke_k$
\item $\forall f$, $f^*=f$ ? $f=\displaystyle \sum_{k=1}^\infty f_ke_k$
\end{enumerate}
Iniziamo a dare dei lemmi preliminari importanti per il seguito, consideriamo 
\[f_N = \displaystyle \sum_{k=1}^Nf_ke_k \qquad \hat{f_N} = \displaystyle \sum_{k=1}^Na_ke_k \quad a_k\in\C\]
caloliamo allora $\|\hat{f_N}\|^2 = (\hat{f_N},\hat{f_N}) = (\displaystyle \sum_{k=1}^Na_k e_k,\displaystyle \sum_{j=1}^Na_je_j) = \sum_{k=1}^N |a_k|^2$
tenendo conto che $(e_k,e_j) = \delta_{kj}$, otteniamo così la formula di pitagora generalizzata:
\[\|\hat{f_N}\|^2 =\sum_{k=1}^N |a_k|^2\quad \text{se } a_k=f_k \implies \|f_N\|^2 =\sum_{k=1}^N |f_k|^2\]
un altra identità che ci sarà utile nell'immediato futuro è la seguente:
\[(f,\hat{f_N}) = (f,\sum_{k=1}^Na_k e_k) = \sum_{k=1}^Na_k(f,e_k) = \sum_{k=1}^Na_k\overline{(e_k,f)} = \sum_{k=1}^Na_k\overline{f_k}\]
andiamo ora a calcolare la seguente quantità:
\begin{equation}\label{roba}\|f-\hat{f_N}\|^2 = \|f\|^2 + \|\hat{f_N}\|^2 - 2\text{Re}(f,\hat{f_N}) = \|f\|^2+\sum_{k=1}^N |a_k|^2 - 2\text{Re}\left(\sum_{k=1}^N\overline{f_k}a_k\right)\end{equation}
ora teniamo conto che vale anche:
\[\sum_{k=1}^N |a_k-f_k|^2 = \sum_{k=1}^N\left[|a_k|^2+|f_k|^2-2\text{Re}(\overline{f_k}a_k)\right]\]
che riarrangiando i termini a sinstra e a destra otteniamo:
\[\sum_{k=1}^N\left[|a_k|^2-2\text{Re}(\overline{f_k}a_k)\right] = \sum_{k=1}^N (|a_k-f_k|^2 -|f_k|^2) \]
ora mettendo questa nella \eqref{roba} otteniamo la seguente identità:
\begin{equation}\label{star}\|f-\hat{f_N}\|^2 =\|f\|^2 -\sum_{k=1}^N|f_k|^2+\sum_{k=1}^N |a_k-f_k|^2\end{equation}
facciamo il caso in cui $a_k = f_k$, la quantità appena calcolata è ovviamente non negativa ricaviamo:
\begin{equation}\label{quasibessel}0\leq\|f-f_N\|^2  = \|f\|^2- \sum_{k=1}^N|f_k|^2 \end{equation}
dai cui la \textbf{disuguaglianza di Bessel:}
\[\sum_{k=1}^N|f_k|^2\leq \|f\|^2\]
ora mettiamo la \eqref{quasibessel} nella \eqref{star} otteniamo:
\[\|f-\hat{f_N}\|^2 = \|f-f_N\|^2 +  \sum_{k=1}^N |a_k-f_k|^2 \geq 0 \]
dai cui essendoci a destra due termini positivi:
\[\|f-\hat{f_N}\|^2 \geq \|f-f_N\|^2\]
Siamo ora in grado di dare risposta alle domande che ci siamo posti prima, iniziamo dalla prima e consideriamo $f_N$, se è di Cauchy è anche convergente quindi mostriamo che è di Cauchy:
\[\|f_N-f_M\|^2 = \| \sum_{k=1}^Nf_ke_k- \sum_{k=1}^Mf_ke_k \|^2 = \| \sum_{k=N+1}^Mf_ke_k\| = \sum_{k=N+1}^M|f_k|^2 \xrightarrow[N,M\to \infty]{}0 \]
dove nell'ultimo passaggio si è usato pitagora generalizzato e la convergenza della serie è garantita dalla disuguaglianza di Bessel. Quindi concludiamo che esiste $f^* =\displaystyle \sum_{k=1}^\infty f_ke_k $ con $f^*\in\H$. La risposta alla seconda domanda ci porta alla seguenti definizioni:
\begin{dfn}
Se vale che $f^*=f$ allora diciamo che $f$ è sviluppabile in serie di Fourier rispetto a $\{e_k\}$
\end{dfn}
\begin{dfn}
$\{e_k\}$ è ortonormale completo (S.O.N.C) se è denso, ovvero $\overline{\{e_k\}} = \H$, ovvero $\|f-\sum_k f_ke_k\|<\varepsilon$ $\forall f$
\end{dfn}
\begin{dfn}
$\{e_k\}$ è chiuso se $\forall f$ vale \textbf{l'uguaglianza di Parseval:}
\[\|f\|^2 = \sum_{k=1}^\infty |f_k|^2\]
\end{dfn}\leavevmode
\\queste definizione e i risultati ottenuti ci fanno ad arrivare alla seguente catena di implicazioni:
\[\forall f\,\, f^*=f\iff \text{completezza di S.O.N}\iff \text{chiusura di S.O.N (Parseval)}\]
\begin{thm}
(Di chiusura) $\{e_k\}$ è chiuso $\iff$ 
\[\forall f,g \quad (f,g) = \sum_{k=1}^\infty(f,e_k)(e_k,g)\]
\end{thm}
\hspace{-1.6em}\textbf{Dimostrazione:}\\
$\impliedby)$ se $f=g$ otteniamo l'uguaglianza di Parseval quindi per definizione $\{e_k\}$ è chiuso.\\
$\implies)$ consideriamo il vettore $h=g+\lambda f$ con $\lambda\in\C$ arbitrario e applichiamo Parseval:
\[(g+\lambda f,g+\lambda f) = \sum_{k=1}^\infty(e_k,g+\lambda f)(g+\lambda f,e_k)\]
sviluppiamo i due membri dell'equazione indipendentemente:
\begin{align*} 
(g+\lambda f,g+\lambda f) &= \sum_{k=1}^\infty(e_k,g+\lambda f)(g+\lambda f,e_k) \\ 
\|g\|^2 + |\lambda|^2\|f\|^2 + 2\text{Re}(\lambda(g,f))&=\sum_{k=1}^\infty|(e_k,g)|^2 + |\lambda|^2\sum_{k=1}^\infty|(e_k,f)|^2 + 2\text{Re}\left[\lambda\sum_{k=1}^\infty (g,e_k)(e_k,f) \right]
\end{align*}
notiamo che $(e_k,g) = g_k$ e lo stesso $(e_k,f)=f_k$, quindi applichiamo parseval e sostituiamo nella parte di destra le sommatorie con le norme che a questo punto si semplificano con quelle a sinistra e ci rimane:
\[\text{Re}(\lambda(g,f)) = \text{Re}\left[\lambda\sum_{k=1}^\infty (g,e_k)(e_k,f) \right] \]
ma data l'arbitrarietà di $\lambda$ deve essere:
\[(g,f) = \sum_{k=1}^\infty (g,e_k)(e_k,f)\]
infatti consideriamo $z_1,z_2\in\C$, allora se in $\text{Re}(\lambda z_1) = \text{Re}(\lambda z_2)$ prendiamo $\lambda = 1$ abbiamo l'equivalenza delle parti reali del numero complesso, mentre se prendiamo $\lambda = i$ abbiamo l'equivalenza delle parti immaginarie da cui quindi l'equivalenza dei due numeri complessi.
\begin{thm}
(Riesz-Fischer) Sia $\{e_k\}$ S.O.N di $\H$, se esiste $\{a_k\}\in\C$ con $\sum_{k=1}^\infty|a_k|^2<+\infty$ allora esiste $h\in\H$ tale che $a_k=h_k=(e_k,h)$ e vale l'uguaglianza di Parseval $\|h\|^2 = \sum_{k=1}^\infty|a_k|^2$
\end{thm}
\begin{thm}
(Criterio di chiusura-completezza) $\{e_k\}$ è S.O.N.C $\iff$ preso un $f$ tale che $(e_k,f)=0\,\forall k$ allora $f=0$
\end{thm}
\hspace{-1.6em}\textbf{Dimostrazione:}\\
$\implies$) se $\{e_k\}$ è S.O.N.C allora vale Parseval quindi $\forall f$ $\|f\|^2 = \sum_{k=1}^\infty|(e_k,f)|^2$, è evidente che se $(e_k,f)=0$ per ogni $k$ allora $\|f\|^2=0$ quindi per definizione di norma $f=0$.\\
$\impliedby)$ Procediamo per assurdo e diciamo che $\{e_k\}$ non è S.O.N.C quindi Parseval non vale, quindi esiste $g$ tale che $\sum_{k=1}^\infty|g_k|^2<\|g\|^2$ allora per Riesz-Fischer esiste un $h$ tale che $g_k=h_k=(e_k,h)$ e per il quale vale Parseval $\|h\|^2 = \sum_{k=1}^\infty|(e_k,h)|^2 = \sum_{k=1}^\infty|g_k|^2$ ma allora dato che $g_k=(e_k,g)$ abbiamo che $(e_k,g-h)=0$ da cui per ipotesi $h=g$ da cui un assurdo perchè Parseval è diverso.
\begin{coro}
Se $A$ è non limitato e vale $(g^*,f)=(g,Af)\, \forall f\in D_A$ con $\overline{D_A}=\H$ allora $g^*$ è unico.
\end{coro}
\hspace{-1.6em}\textbf{Dimostrazione:}\\
Se esistesse $g_1^*$ tale che $(g^*,f) = (g_1^*,f)\implies (g^*-g_1^*,f)=0$ che per il criterio di chiusura ci da: $g^*_1=g^*$.\\
\newline
Il criterio di chiusura ci permette quindi di giustificare la richiesta della condizione $\overline{D_A}=\H$ nella definizione \eqref{aggiunto_nonlimitato}  che come avevamo già anticipato ci permette di dare una definizione univoca dell'operatore autoaggiunto.
\begin{dfn}
Siano $\H,\H_1$ due spazi di Hilbert, essi si dicono unitariamente equivalenti se esiste un applicazione $U:\H\to\H_1$ lineare e biunivoca che preserva il prodotto scalare, overro:
\[(f,g)_\H = (U(f),U(g))_{\H_1}\quad \forall f,g\in\H\]
\end{dfn}
\begin{thm}
(Di Schr{\"o}dinger) Ogni spazio di Hilbert separabile $\H$ è unitariamente equivalente a \[l_2 = \left\{\{z_n\}=(z_1,z_2,\dots) : \sum_{k=1}^{+\infty}|z_k|^2<+\infty\right\}\]
\end{thm}
\hspace{-1.6em}\textbf{Dimostrazione:}\\
Prendiamo un S.O.N.C. $\{e_k\}$ nello spazio di Hilbert, sappiamo quindi che $\forall f\in \H$ possiamo scomporre gli elementi in serie di Fourier con coefficienti $f_k =(e_k,f)$. Possiamo allora scegliere come successione $\{x_k\}=\{f_k\}$, essa fa parte di $l_2$ in quanto per Parseval $\displaystyle\sum_{k=1}^\infty|f_k|^2<+\infty$. Inoltre notiamo che se esiste $\{x_k\}$ per Riesz-Fischer esiste $f$ tale che $x_k = f_k$, questo ci dimostra la biunivocità tenendo conto che $f$ è unico infatti se esistesse $f^*_k = x_k\implies (f,e_k)=(f^*,e_k)\implies (f-f^*,e_k)=0\implies f=f^*$.\\
Mostriamo inoltre che questa mappa $U:f\in\H\mapsto \{f_k\}=(e_k,f)\in l_2$ è lineare:
\[\alpha f+\beta g\in\H \mapsto (e_k,\alpha f+\beta g) = \alpha(e_k,f)+\beta(e_k,g) = \alpha U(f)+\beta U(g) \]
Vediamo infine che preserva il prodotto scalare, ricordiamo che per $l_2$ il prodotto scalare è definito nella \eqref{innerproduct}, per dimostrare che il prodotto scalare è preservato utilizziamo il teorema di chiusura e scriviamo:
\[\forall f,g\in\H \quad (f,g)_\H =\sum_{k=1}^\infty (f,e_k)(e_k,g) = \sum_{k=1}^\infty \overline{(e_k,f)}(e_k,g) = \sum_{k=1}^\infty \overline{x_k}^fx_k^g = (x^f,x^g)_{l_2}\]
Questo teorema ha una grande importanza storica per la fisica, infatti agli inizi della meccanica quantistica si andavano sviluppando due diverse teorie, la meccanica della matrici di Heinsemberg, Born e Jordan e la meccanica ondulatoria di Schr{\"o}dinger. La prima utilizzava lo spazio $l_2$ e come operatori delle matrici infinito dimensionali, la meccanica ondulatoria invece utilizzava gli spazi di Hilbert e gli operatori su essi definiti. Entrambe le teorie funzionavano bene e non si capiva quale fosse quella ``giusta'', questo teorema dimostra la biunivocità di tali descrizioni della natura ponendo fine alla discussione. Nel tempo si è andata ad affermare la meccanica ondulatoria di Schr{\"o}dinger e la descrizione matematica sviluppata in questi appunti, questo perchè dal punto di vista del calcolo è molto più comoda rispetto che a lavorare con successioni infinite e matrici infinito dimensionali.
\section{Problemi sul dominio}
Approfondiamo al questione dei problemi di dominio negli operatori, abbiamo già visto che gli operatori limitati non hanno problemi, negli operatori non limitati invece come abbiamo visto in vari esempi la scelta del dominio è di vitale importanza.
\begin{dfn}
$A$ operatore si dice chiuso se presi $g_n\xrightarrow[n\to \infty]{\H}g$ con $Ag_n\xrightarrow[n\to \infty]{\H}h$ allora $h=Ag$
\end{dfn}
è evidente che se A è limitato allora è anche chiuso.
\begin{thm}
Dato $A$ non limitato con $D_A =\H$ allora $A^\dagger$ è chiuso
\end{thm}
\begin{coro}
Se A è simmetrico allora $A\subseteq A^\dagger$
\end{coro}
Un esempio di questo l'abbiamo già visto, è l'operatore $P$ simmetrico con dominio $D_P=\{f,f'\in L_2(0,2\pi),f(0) = f(2\pi) = 0\}\supseteq D_{P^\dagger}$.\\
Se $A$ non è chiuso esiste la sua chisura $\overline{A}$ e se vale che $\overline{A}^\dagger = \overline{A}$ alloa $A$ si dice essenzialmente autoaggiunto.\\
A parte questa sfumatura consideriamo ora operatori autoaggiunti $A$ e definiamo quello che si chiama lo spettro di un operatore autoaggiunto.
\begin{dfn}
Si chiama il risolvente di $A$ il seguente operatore:
\[R_z(A) = (z\mathbbm{1}-A)^{-1}\quad z\in\C\]
\end{dfn}
\begin{dfn}
Si dice che $z$ è un valore regolare se:
\begin{enumerate}[i.]
\item $R_z(A)$ esiste $\iff$ $\text{Ker}(z\mathbbm{1}-A) = \{0\}$
\item $R_z(A)$ è limitato $\iff$ $\|R_z(A)\psi\|\leq C_A \|\psi\|\, \forall \psi \in D_A$
\end{enumerate}
inoltre l'insieme dei valori regolari si indica con $P(A)$
\end{dfn}
\begin{dfn}
\label{spettro}
Si chiama lo spettro di $A$ il seguente insieme:
\[\sigma(A) = \C-P(A)\]
quindi evidentemente dalla definizione prima $z\in\sigma(A)$ se:
\begin{enumerate}
\item $\exists \varphi$ tale che $A\varphi = z\varphi$ con $\varphi\neq 0$ e in tal caso si dice che lo spettro è discreto
\item $R_z(A)$ non è limitato e allora si dice che lo spettro è continuo
\end{enumerate}
\end{dfn}
\begin{lem}
Se $A^\dagger = A$ allora vale:
\[\|(z-A)\psi\|\geq |\text{Im}(z)|\|\psi\|\implies \|R_z(A)\psi\|\leq \frac{1}{|\text{Im}(z)|\|\psi\|} \quad\forall\psi\in D_A\]
\end{lem}
da questo lemma segue il seguente corollario tenendo conto che per forza deve essere $|\text{Im}(z)\neq 0|$
\begin{coro}
Se $A=A^\dagger$ allora $\sigma(A)\subseteq \R$
\end{coro}
Tempo di fare alcune esempi con i nostri classici operatori $Q$ e $P$. Iniziamo da $Q$ che sappiamo essere limitato e autoaggiunto, come insieme consideriamo $L_2(0,2\pi)$. Quello che vogliamo fare è calcolare lo spettro di $Q$, $\sigma(Q)$, per farlo possiamo procedere in due modi, il primo è quello di calcolarsi il risolvente di $Q$:
\[g(x) = (z\mathbbm{1} -Q)f(x) = zf(x) - xf(x) \implies f(x) = (z\mathbbm{1} -Q)^{-1}g(x) = \frac{1}{z-x}g(x) = R_z(Q)g(x)   \]
vediamo dove è definito questo operatore:
\[\|R_z(Q)f\|^2 = \int_0^{2\pi}\left|\frac{f(x)}{z-x}\right|^2dx =\int_0^{2\pi}\frac{|f(x)|^2}{|z-x|^2}dx \]
da cui ovviamente $\sigma(Q) = [0,2\pi]$, quindi $Q$ ha uno spettro continuo. Notiamo che nel caso di $L_2(\R)$ avremmo avuto dallo stesso calcolo $\sigma(Q) = \R$.\\
Il secondo modo è quello di trovare gli autovalori di $Q$ in modo da sfruttare la i. della definizione \eqref{spettro}. Scriviamo l'equazione agli autovalori:
\[Qf = \lambda f \implies (x-\lambda)f(x) = 0\]
evidentemente non esistono autovettori ne autovalori in $L_2(0,2\pi)$ per cui lo spettro è per forza continuo. Prima di passare allo studio dello spettro di $P$ soffermiamoci un attimo sull'equazione $(x-\lambda)f(x) = 0$, essa avrebbe una soluzione che equivale alla delta di Dirac $\delta(x-\lambda)$ che ovviamente non fa parte di $L_2$. Questo ha portato Gelfand all'introduzione del concetto di autovettori generalizzati che sono vettori appartenente allo spazio $\mathcal{S}'$ in questo senso si può scrivere la cosiddetta tripletta di Gelfand $\mathcal{S}\subset L_2(\R) = L_2(\R)' \subset \mathcal{S}'$.\\
Studiamo lo spettro di $P$ con dominio $D_P=\{f,f'\in L_2(0,2\pi),f(0) = f(2\pi)\},$ abbiamo già mostrato che su questo dominio è autoaggiunto. In questo caso il primo metodo perdeterminare lo spettro di $P$ ci porta ad una rappresentazione integrale del risolvente:
\[R_z(P)f = \int_0^{2\pi}G_z(x,y)f(y)\,dy\]
con $G_z(x,y)$ detto nucleo integrale non facile da ricavare, per questo sfruttiamo il secondo modo e scriviamo l'equazione agli autovalori con le sue condizioni al contorno:
\[i\frac{df}{dx}(x) = \lambda f(x)\quad f(0)=f(2\pi),\lambda\in\R\]
la soluzione è evidentemente un esponenziale complesso $f(x) = e^{i\lambda x}$ se imponiamo le condizioni al contorno otteniamo:
\[f(0) = A = Ae^{i\lambda 2\pi} = f(2\pi) \implies  Ae^{i2\pi n} = Ae^{i\lambda 2\pi} \implies \lambda_n = n\]
in questo caso allora abbiamo $n$ autovalori, lo spettro è quindi discreto. Gli autovettori normalizzati associati sono oggetti che abbiamo già incontrato $e_n(x) = \frac{e^{inx}}{\sqrt{2\pi}}$.\\
Ne caso di $P$ non limitato abbiamo visto che a seconda del dominio potrebbe o no essere autoaggiunto, esiste un modo per decidere se un operatore, aggiustando opportunamente il dominio sia autoaggiunto o meno, vediamo quindi di formalizzare questo fenomeno che abbiamo già visto nell'esempio di $P$. 
\begin{dfn}
Consideriamo due operatori $A,B$ simmetrici e non limitati, se $D_A \subseteq D_B$ si dice che $B$ è un estensione di $A$ ($A\subseteq B$).
\end{dfn}
Per operatori simmetrici vale che $A\subseteq A^\dagger$. Introduciamo a questo punto due spazi detti i \textbf{sottospazi di difetto}:
\[\mathcal{N}_\pm = \{h,(A^\dagger\pm i \mathbbm{1})h = 0\}\]
in pratica introduciamo l'operatore $A^\dagger\pm i \mathbbm{1}$ e questi spazi non sono altro che il suo Ker. Le dimensioni di questi sottospazi si chiamano \textbf{indici di difetto:} $n_\pm = \text{Dim}(\mathcal{N}_\pm)$.\\
Grazie al teorema della proiezione ortogonale possiamo scrivere $\H$ in funzione di tali sottospazi:
\[\H =\mathcal{N}_\pm \oplus \mathcal{N}_\pm^\perp \]
\begin{lem}
\label{lemmaacaso}
Se $A$ è simmetrico con $\overline{D_A}=\H$ allora $D_{A^\dagger} = D_A\oplus\mathcal{N}_+\oplus \mathcal{N}_- $
\end{lem}
\begin{thm}
(Di Von-Neumann) Se $A$ è chiuso e simmetrico possiamo avere la seguente casistica:
\begin{enumerate}[i.]
\item se $n_+ = n_- =0 \iff A^\dagger = A$
\item se $n_+\neq n_- \iff$ non esistono estensioni autoaggiunte
\item se $n_+=n_->0 \iff$ esistono infinite estensioni autoaggiunte parametrizzate da una matrice unitaria $U$ di dimensione $n\times n$
\end{enumerate}
\end{thm}
Ricordiamo che una matrice è unitaria quando vale $U^\dagger U = U U^\dagger = \mathbbm{1}$. Non dimostriamo questo teorema a parte il caso i. estremamente banale che segue dal lemma \eqref{lemmaacaso} che ci dice $D_A = D_{A^\dagger}$ quindi $A^\dagger = A$.\\
Diamo invece una ``ricetta'' per costruire le infinite estensioni autoaggiunte del caso iii., in particolare ci focalizzeremo nel caso in cui gli indici di difetto valgano 1 o 2.
\begin{itemize}
\item caso $n_+=n_- = 1$: Il domino dell'estensione autoaggiunta in questo caso è:
\[D_A = \{\varphi(x),A\varphi(x)\in\H : \varphi = h(x) +U_+(x) + e^{i\theta}U_-(x),\,h\in D_A \}\]
Con $U_\pm$ soluzioni dell'\textbf{equazione di Von-Neumann} con condizione al contorno:
\[A^\dagger U_\pm(x) = \mp i U_\pm(x)\qquad \|U_+\| = \|U_-\|\]
notiamo che la matrice unitaria $U$ in questo caso è data dall'elemento complesso $U=e^{i\theta}$
\item caso $n_+=n_- = 2$: in questo caso la matrice unitaria è $2\times2$ contiene quindi 4 parametri complessi $a,b,c,d\in\C$:
\[U = \begin{pmatrix}
  a & b \\
  c & d
 \end{pmatrix} \quad
	U^\dagger = \begin{pmatrix}
  \overline{a} & \overline{b} \\
  \overline{c} & \overline{d}
 \end{pmatrix}
 \]
imponendo la condizione di unitarietà otteniamo 4 parametri reali da imporre e gli altri che parametrizzano le infinite autoestensioni.% infatti:
%\[\begin{pmatrix}
 % a & b \\
 % c & d
 %\end{pmatrix}\begin{pmatrix}
 % \overline{a} & \overline{b} \\
 % \overline{c} & \overline{d}
 %\end{pmatrix} = \begin{pmatrix}
 % 1 & 0 \\
 % 0 & 1
 %\end{pmatrix}\]
 %\[\begin{cases}
 %   a\overline{a} + b\overline{c} = 1\\
 %   a\overline{b} + b\overline{d} = 0\\
 %   c\overline{a} + d\overline{c} = 0\\
 %   c\overline{d} + d\overline{d} = 1 
 % \end{cases} \]
\end{itemize}
Procediamo a fare i classici esempi di $P$ e $Q$ nonostante sappiamo già i probabili risultati in quanto già parzialmente visti in esempi precedenti. Consideriamo però in questo caso 3 diversi spazi di lavoro: $L_2(\Sigma)$ con $\Sigma = \R,\R^+,[a,b]$ con $a,b\in\R$.\\
Partiamo da $Q$, sappiamo già che nel caso $\Sigma= [a,b]$ l'operatore è limitato e autoaggiunto, studiamo quindi i casi più particolari con $\Sigma=\R,\R^+$, dove l'operatore non è limitato, utilizziamo come dominio $C_0^\infty(Q)$ e vediamo innanzitutto se è simmetrico:
\[(Qg,f)- (g,Qf) = \int_{\Sigma}(x\overline{g}f-\overline{g}xf)dx = 0 \]
quindi è simmetrico su questi domini, per vedere se ha estensioni autoaggiunte risolviamo l'equazione di Von-Neumann e troviamo quindi gli elementi di $\mathcal{N}_\pm$:
\[Q^\dagger U_\pm(x) = - \pm iU_\pm(x) \quad xU_\pm(x) \pm iU_\pm(x)=0 \quad (x\pm i)U_\pm(x) = 0 \quad \forall x\in\Sigma\]
da cui l'unica soluzione è l'elemento nullo $U_\pm = 0$ quindi le dimensioni di $\mathcal{N}_\pm$ sono nulle cosi come gli indici di difetto $n_\pm = 0$, ricadiamo quindi nel caso i. del teorema di Von-Neumann che ci dice che $Q$ è autoaggiunto su $\Sigma$, in tutti e tre i casi.\\
Studiamo ora il caso di $P$ che in tutti e tre i casi non è limitato, prendiamo inizialmente come dominio $D_P = \{f,f'\in L_2(\Sigma)\}$. Consideriamo i tre casi separatamente:
\begin{enumerate}
\item caso $\Sigma = \R$:\\
Per prima cosa vediamo se è simmetrico
\[(Pg,f)-(g,Pf) = -i\int_{-\infty}^{\infty}(\overline{g}'f + \overline{g}f')dx = -i(\overline{g}f\big|_{-\infty}^{+\infty} =0  \]
uguale a 0 in quanto le funzioni sono a quadrato sommabili. L'equazione di Von-Neumann in questo caso è:
\[-i\frac{d}{dx}U_\pm(x) = \mp iU_\pm(x) \implies \frac{d}{dx}U_\pm(x) = -U_\pm(x)\]
che è un equazione a variabili separabile che ha come soluzione $U_\pm(x) = C_\pm e^{\mp x}$ con $C\pm$ parametri. Notiamo però che le funzioni $U_\pm$ devono essere a quadrato sommabili, vediamo la norma quanto vale:
\[\|U_\pm\|^2 = C_\pm^2 \int_{-\infty}^{+\infty} e^{\mp2x}\,dx\]
che converge solamente quando $C_\pm = 0$ quindi in questo caso $U_\pm$ sono funzioni nulle e perciò le dimensioni di $\mathcal{N}_\pm$ sono 0, ricadiamo nel caso i. del teorema di Von-Neumann che ci dice quindi che l'operatore $P$ su $L_2(\R)$ è un operatore autoaggiunto. Il risultato era quasi scontato se si pensava che la trasformata di Fourier trasforma operazioni di derivata in operazioni di moltiplicazione, quindi il caso di $P$ sulla retta reale è equivalente al caso di $Q$ che abbiamo già visto essere autoaggiunto.
\item caso $\Sigma = \R^+$:\\
come al solto vediamo se è simmetrico su questo dominio:
\[(Pg,f)-(g,Pf) = -i\int_{0}^{\infty}(\overline{g}'f + \overline{g}f')dx = -i(\overline{g}f\big|_{0}^{+\infty}\]
questa quantità non è sempre uguale a 0, ma dobbiamo porre delle condizioni al contorno, per cui $P$ sulla semiretta reale è simmetrico solo sul seguente dominio:
\[D_P = \{f,f'\in L_2(\R^+), f(0)=0\}\]
La soluzione dell'equazione di Von-Neumann è la stessa di prima $U_\pm(x) = C_\pm e^{\mp x}$, quello che cambia sono le condizioni al contorno, calcoliamo la norma e vediamo quando queste soluzioni sono a quadrato sommabile:
\[\|U_+\|^2 = C_+^2 \int_{0}^{+\infty} e^{-2x}\,dx \quad \|U_-\|^2 = C_-^2 \int_{0}^{+\infty} e^{2x}\,dx\]
i due casi ora sono ben diversi, infatti nel caso di $U_+$ l'esponenziale va a 0 all'infinito quindi non abbiamo problemi e $C_+e^{-x}$ è un elemento di $\mathcal{N}_+$ che ha quindi dimensione 1, mentre nel caso $U_-$ l'esponenziale ci fa divergere l'integrale quindi dobbiamo imporre per forza $C_- = 0$ questo significa che $n_- = 0$ diverso da  $n_+ = 1$ quindi siamo nel caso ii. del teorema di Von-Neumann ovvero non esistono estensioni autoaggiunte per $P$ su $\R^+$.
\item caso $\Sigma = [a,b]$:
per semplicità mettiamoci nel caso $\Sigma =[0,2\pi]$ con il quale siamo già familiari. Abbiamo visto che l'operatore su questo insieme è simmetrico se prendiamo come dominio:
\[D_P = \{f,f'\in L_2(\R^+), f(0)=f(2\pi)=0\}\]
le soluzioni dell'equazione di Von-Neumann sono sempre quelle, vediamo la norma con queste condizioni al contorno:
\[\|U_+\|^2 = C_+^2 \int_{0}^{2\pi} e^{-2x}\,dx \quad \|U_-\|^2 = C_-^2 \int_{0}^{2\pi} e^{2x}\,dx\]
non hanno problemi di divergenza di integrali quindi le dimensioni di $\mathcal{N}_\pm$ sono $n_+=n_-=1$, siamo nel terzo caso del teorema di Von-Neumann, esistono infinite estensioni autoaggiunte, vediamo di costruirle. Innanzitutto imponiamo che le norme di $U_\pm$ siano uguali e ricaviamo quindi $C_\pm$:
\[\|U_+\|^2 = C_+^2\Big(\frac{e^{-2x}}{-2}\Big|_0^{2\pi} = \frac{C_+^2}{2}(1-e^{-4\pi}) =\frac{C_+^2}{2}\frac{(e^{4\pi}-1)}{e^{4\pi}} \]
\[\|U_-\|^2 = C_-^2\Big(\frac{e^{2x}}{2}\Big|_0^{2\pi} = \frac{C_-^2}{2}(e^{4\pi}-1) \]
\[\|U_+\|^2 =\|U_-\|^2 \implies \frac{C_+^2}{2}\frac{(e^{4\pi}-1)}{e^{4\pi}} = \frac{C_-^2}{2}(e^{4\pi}-1) \implies C_+^2 = e^{4\pi}C_-^2 \implies C_+=e^{2\pi}C_- \]
il dominio dell'operatore autoaggiunto avrà quindi come condizione:
\[ \varphi(x) = h(x) +C_+e^{-x} + e^{i\theta}C_-e^{x} = h(x) + (e^{2\pi}e^{-x} + e^{i\theta}e^x)C_- \]
\end{enumerate} 
Studiamo questa funzione $\varphi(x)$ e vediamo che proprietà ha, iniziamo col calcolarla in 0 e $2\pi$:
\[\varphi(0) = (e^{2\pi} + e^{i\theta})C_-\quad \varphi(2\pi) = (1+e^{i\theta}e^{2\pi})C_-\]
ricordo che $h(0)=h(2\pi) = 0$ per le condizioni di simmetria dell'operatore $P$. Studiamo la relazione tra queste due quantità facendone il rapporto:
\[\frac{\varphi(0)}{\varphi(2\pi)} = \frac{e^{2\pi} + e^{i\theta}}{1+e^{i\theta}e^{2\pi}} \doteq \gamma \in \C\]
vediamo quanto vale il modulo di $\gamma$:
\[\overline{\gamma} = \frac{e^{2\pi} + e^{-i\theta}}{1+e^{-i\theta}e^{2\pi}}\implies \gamma\overline{\gamma} = \left(\frac{e^{2\pi} + e^{i\theta}}{1+e^{i\theta}e^{2\pi}}\right)\left(\frac{e^{2\pi} + e^{-i\theta}}{1+e^{-i\theta}e^{2\pi}}\right)\]
moltiplichiamo sopra e sotto per $e^{-i\theta}$ e otteniamo:
\[|\gamma| =\left(\frac{e^{2\pi} + e^{i\theta}}{1+e^{i\theta}e^{2\pi}}\right)\left(\frac{e^{2\pi} + e^{-i\theta}}{1+e^{-i\theta}e^{2\pi}}\right)\frac{e^{-i\theta}}{e^{-i\theta}} = \left(\frac{e^{2\pi}e^{-i\theta} + 1}{e^{-i\theta}+e^{2\pi}}\right)\left(\frac{e^{2\pi} + e^{-i\theta}}{1+e^{-i\theta}e^{2\pi}}\right) = 1\]
ottimo, $\gamma$ ha modulo unitario quindi possiamo prendere $\alpha \in\R$ e scrivere $\gamma = e^{i\alpha}$ ovvero:
\[\frac{\varphi(0)}{\varphi(2\pi)} = e^{i\alpha} \implies \varphi(0) = e^{i\alpha} \varphi(2\pi)\]
in definitiva $P$ è autoaggiunto sui seguenti infiniti domini:
\[D_{P_\alpha} = \{f,f'\in L_2(0,2\pi):\varphi(0) = e^{i\alpha} \varphi(2\pi) \}\]
notiamo che nel caso $\alpha=0$ otteniamo lo stesso dominio che avevamo dimostrato rendere $P$ autoaggiunto per calcolo diretto in un passato esempio. Ora vediamo che anche condizioni antiperiodiche $\varphi(0)=-\varphi(2\pi)$ dato da $\alpha = \pi$ ci rendono l'operatore $P$ autoaggiunto e non solo è autoaggiunto con $\alpha$ arbitrario. Queste condizioni fisicamente corrispondono a particelle diverse, nei casi di condizioni periodiche e antiperiodiche otteniamo i bosoni e i fermioni nel caso di $\alpha$ arbitario troviamo delle particelle generalizzate chiamate \emph{anyons} (in italiano qualunquoni) la cui esistenza è stata dimostrata sperimentalmente.
\section{Teoremi spettrali}
In questa sezione ci occupiamo dei teoremi spettrali per operatori autoaggiunti $A=A^\dagger$, in particolare vedremo il caso di operatori compatti (i quali hanno sempre spettro discreto) ed operatori limitati e non. 
\subsection{Operatori compatti}
Iniziamo a definire la classe degli operatori compatti e le loro proprietà
\begin{dfn}
$A:\H\to\H$ è compatto se, sia $\{f_n\}$ limitata cioè $\|f_n\|<M$ $\forall n$ allora da $\{Af_n\}$ è possibile estrarre una sottosuccessione convergente quindi $\{Af_{nk}\}$ è di Cauchy.
\end{dfn}
\begin{thm}
\label{compatto}
$A:\H\to\H$ è compatto $\iff$ $f_n\xrightarrow[n\to \infty]{W} f \implies Af_n \xrightarrow[n\to \infty]{\H}Af$. Ovvero $A$ trasforma successioni debolmente convergenti in successioni fortemente convergenti.
\end{thm}
La compattezza di un operatore e la sua limitatezza sono cose diverse, infatti un operattore limitato non è detto che sia compatto mentre un operatore compatto è limitato, vediamo meglio questa cosa con un esempio.\\
Consideriamo l'operatore identità $A = c\mathbbm{1}$ che è banalmente limitato ma non compatto, infatti consideriamo una base ortonormale $\{\varphi_n\}$ che ha norma unitaria quindi è per forza limitata e vediamo che $\{A\varphi_n\} = c\{\varphi_n\}$ non è di Cauchy quindi non è convergente:
\[\|A\varphi_n-A\varphi_m\|^2 = |c|^2\|\varphi_n-\varphi_m\|^2 = |c|^2 (\|\varphi_n\|^2+\|\varphi_m\|^2) = 2|c|^2 \]
dove abbiamo sfrutatto la \eqref{identitautile} e il fatto che $(\varphi_n,\varphi_m)=0$.
\begin{lem}
Se $A$ è compatto allora $A$ è limitato
\end{lem}
un esempio importante di operatore compatto è: $A:\H\to \C^n$, questi operatori si dicono a rango finito. Un esempio è il seguente:
\[A = \sum_{n=1}^Na_n(\varphi_n,\cdot)\varphi_n \equiv \sum_{n=1}^Na_n\ket{n}\bra{n}\]
dove $\{\varphi_n\}$ è un S.O.N. Vediamo che è compatto, prendiamo $\{\varphi_n\}$ S.O.N. abbiamo allora che $\varphi_n\xrightarrow[n\to \infty]{W} 0$, questo perchè per definizione di convergenza debole deve valere $(f,\varphi_n)\xrightarrow[n\to \infty]{\C} 0$ $\forall f$, ma per la disuguaglianza di Bessel abbiamo che:
\[\sum_{n=1}^\infty |(f,\varphi_n)|^2\leq \|f\|^2\]
quindi per forza $(f,\varphi_n)$ deve tendere a 0, questo significa che ogni S.O.N converge a 0. Applichiamo l'operatore $A$ alla successione debolmente convergente:
\[A\varphi_k = \sum_{n=1}^N a_n (\varphi_n,\varphi_k)\varphi_n \xrightarrow[k\to \infty]{} 0\]
trivialmente per la proprietà di ortonormalità della base, quindi converge fortemente e per il teorema \eqref{compatto} $A$ è compatto.
\begin{thm}
Se $A$ è compatto e $B$ limitato allora $AB$ e $BA$ sono compatti
\end{thm}
\hspace{-1.6em}\textbf{Dimostrazione:}\\
Triviale per la continuità di B.
\begin{coro}
\label{inversocompatto}
Se $A$ è compatto e invertibile allora $A^{-1}$ non può essere compatto.
\end{coro}
\hspace{-1.6em}\textbf{Dimostrazione:}\\
procediamo per assurdo e diciamo che se $A$ è compatto allora anche $A^{-1}$ è compatto, per il teorema appena enunciato allora anche $AA^{-1}$ è compatto, ma $AA^{-1}=\mathbbm{1}$ che abbiamo mostrato prima esplicitamente non essere compatto.\\
\newline Questo teorema è molto importante nello studio di operatori non limitati, infatti pensiamo ad $A=A^\dagger$ compatto e invertibile, allora $L = A^{-1}$ in generale è non limitato, ma ha le stesse proprietà spettrali di $A$, quindi al posto di studiare operatori non limitati come $L$ possiamo studiare gli operatori inversi che sono limitati e quindi più facili da studiare.
\begin{thm}
Se $A:\H\to\H$ limitato e autoaggiunto allora $\|A\| = \underset{\|g\|=1}{sup}|(g,Ag)|$
\end{thm}
\begin{thm}(Dell'esistenza dell'autovettore)
Se $A=A^\dagger$ è compatto allora esiste $\varphi_1$ e $\lambda_1\in\R$ tale che $A\varphi_1 = \lambda_1\varphi_1$ con $|\lambda_1| = \|A\|$
\end{thm}
\begin{thm}(Di Hilbert-Schmidt)
Se $A$ è compatto e autoaggiunto $A:\H\to\H$ allora valgono:
\begin{enumerate}[i.]
\item $\exists \{\varphi_n\}$ S.O.N e $\{\lambda_n\}$ tale che $A\varphi_n = \lambda_n \varphi_n$
\item $\displaystyle \lim_{n\to +\infty} \lambda_n = 0$
\item la degenerazione di $\lambda_n$ è finita (i $\lambda_n$ hanno molteplicità finita)
\item $\forall h\in\H$ vale $Ah = \displaystyle\sum_{n=1}^{\infty}\lambda_n (\varphi_n,h)\varphi_n$
\end{enumerate}
\end{thm}
gli operatori $(\varphi_n,\cdot)\varphi_n$ vengono chiamati proiettori $P_n$ e sono utili perchè soddisfano la seguente proprietà:
\[P_nP_m = \delta_{nm} P_n\]
questo significa che valgono:
\[A = \sum_{n=1}^\infty \lambda_nP_n \quad A^2 = \sum_{n=1}^\infty \lambda_n^2P_n \quad f(A)  = \sum_{n=1}^\infty f(\lambda_n)P_n\]
\textbf{Dimostrazione Hilbert-Schmidt:}
\begin{enumerate}[i.]
\item Per il teorema di esistenza dell'autovettore esistono $\varphi_1,\lambda_1$ con $|\lambda_1| = \|A\|$ con le quali possiamo scrivere lo spazio di Hilbert come:
\[\H = \{\varphi_1\} \oplus H_2\]
dove $H_2$ è il complemente ortogonale di $\{\varphi_1\}$. Se $f\in H_2$ allora $(f,\varphi_1)=0$ e quindi $Af\in H_2$ infatti $(Af,\varphi_1) = \lambda_1(f,\varphi_1) = 0$. Consideriamo allora $A_2$ restrizione di $A$ su $H_2$, $A_2$ è autoaggiunto e compatto in quanto restrizione di $A$ autaggiunto e compatto, possiamo di nuovo evocare il teorema dell'esistenza del'autovettore e dire che esistono $\varphi_2$ e $\lambda_2$ tali che $A_2 \varphi_2 = \lambda_2\varphi_2$ e $|\lambda_2| = \|A_2\| \leq \|A\| = |\lambda_1|$ in quanto $A_2$ è restrizione, si può quindi scrivere lo spazio di Hilbert come:
\[\H = \{\varphi_1,\varphi_2\} \oplus H_3 \]
è evidente che possiamo reiterare il processo ed ottenere:
\[\H =\{\varphi_1,\varphi_2,\dots,\varphi_{n-1}\} \oplus H_{n} \]
a questo punto ci possono essere due casistiche, il caso in cui $A_{n}$ restrizione n-esima sia nulla, quindi a rango finito e possiamo scrivere che:
\[A = \sum_{k=1}^n \lambda_k P_k\]
nel caso io cui $A_n\neq 0$ allora vale che $A_n \varphi_n = \lambda_n \varphi_n$ con $\{\varphi_n\}$ S.O.N.
\item  Procediamo per assurdo e diciamo che $\displaystyle \lim_{n\to +\infty} \lambda_n = \lambda_r \neq 0$, consideriamo la successione $\{\displaystyle\frac{\varphi_n}{\lambda_n}\}$, essa è limitata e vale che:
\[A\frac{\varphi_n}{\lambda_n} = \frac{\lambda_n}{\lambda_n}\varphi_n = \varphi_n\]
ma dato che $\{\varphi_n\}$ non è di cauchy allora $A$ non è compatto.
\item Per assurdo poniamo degenerazione infinita, ma allora la successione porterà prima o poi ad $\displaystyle \lim_{n\to +\infty} \lambda_n = \lambda_r \neq 0$
\item $\forall h\in\H$ prendiamo il vettore $g_{n+1} = h-\displaystyle\sum_{k=1}^n (\varphi_k,h)\varphi_k$, abbiamo che $g_{n+1}\in H_{n+1}$ infatti:
\[(\varphi_j,g_{n+1}) = (\varphi_j,h) - \sum_{k=1}^n (\varphi_k,h)(\varphi_k,\varphi_j) =(\varphi_j,h)-(\varphi_j,h) = 0 \]
calcoliamo l'operatore $A$ applicato a $g_{n+1}$
\[Ag_{n+1} = Ah - \sum_{k=1}^n \lambda_k(\varphi_k,h)\varphi_k = A_{n+1} g_{n+1}\]
vediamo la differenza tra $A$ e $\sum_{k=1}^n \lambda_k(\varphi_k,h)\varphi_k$ tende a 0 quando $n$ tende all'infinito, infatti:
\[\|Ah - \sum_{k=1}^n \lambda_k(\varphi_k,h)\varphi_k\| = \|A_{n+1}g_{n+1}\| \leq \|A_{n+1}\|\|g_{n+1}\| = |\lambda_{n+1}| \|g_{n+1}\|\]
ora consideriamo che vale la seguente:
\[\|h\|^2 = \|g_{n+1} + \sum_{k=1}^n (\varphi_k,h)\varphi_k\|^2 = \|g_{n+1}\|^2 + \|\sum_{k=1}^n(\varphi_k,h)\varphi_k\|^2\]
da cui essendoci solo quantità positivi $\|g_{n+1}\|^2 \leq \|h\|^2$. La disuguagliaza di prima allora continua con:
\[|\lambda_{n+1}| \|g_{n+1}\| \leq |\lambda_{n+1}| \|h\|\xrightarrow[n\to \infty]{} 0\]
per la ii. e per il fatto che $\|h\|$ è finita.
\end{enumerate}
\begin{coro}
\label{basehilbertiana}
Se abbiamo $A=A^\dagger$ compatto e invertibile allora $\{\varphi_n\}$ è un S.O.N.C (detta anche base Hilbertiana)
\end{coro}
\textbf{Dimostrazione:}\\
prendiamo $h^* = \sum_n (\varphi_n,h)\varphi_n$ per ogni $h$, allora:
\[Ah^* = A \sum_n (\varphi_n,h)\varphi_n = \sum_n (\varphi_n,h)A\varphi_n = \sum_n \lambda_n(\varphi_n,h)\varphi_n = Ah\]
per la iv. del teorema di Hilbert-Schmidt, quindi:
\[Ah^* - Ah = 0 \implies A(h^*-h) = 0\]
ma dato che $A$ è invertibile e quindi l'unico elemento del nucleo è l'elemento nullo allora $h^*=h$.
\subsection{Alternativa di Fredholm}
Consideriamo $A$ compatto e autoaggiunto nello spazio di Hilbert $\H$ e $A\varphi_n = \lambda_n\varphi_n$. Il problema è quello di risolvere la seguente equazione:
\[f-\mu Af=g\]
con $f$ l'incognita, $\mu$ parametro e $g\in\H$, tale equazione si chiama equazione funzionale di Fredholm. La soluzione è data da:
\begin{enumerate}[i.]
\item se $1-\mu \lambda_n\neq 0$ per ogni $n$ allora esiste un unica soluzione data da:
\[f = g + \mu \sum_{n=1}^\infty \frac{\lambda_n(\varphi_n,g)}{1-\mu\lambda_n}\varphi_n\]
\item  se $1-\mu \lambda_j =  0$ per $j = r,\dots,r+\nu$ con condizione necessaria e sufficiente per l'esistenza della soluzione $(g,\varphi_j) = 0$ per $j = r,\dots,r+\nu$ allora esiste una soluzione non unica data da:
\[f = g+ \mu \sum_{n\neq j}^\infty \frac{\lambda_n(\varphi_n,g)}{1-\mu\lambda_n}\varphi_n  + \sum_{l=r}^{r+\nu} c_l \varphi_l\]
con $c_l$ costanti arbitrarie.
\end{enumerate}
vediamo di dimostrare il caso i., prendiamo l'equazione funzionale e facciamo il prodotto scalare per $\varphi_j$ da entrambe le parti:
\[(\varphi_j,f) - (\varphi_j,\mu Af) = (\varphi_j,g) \implies (1-\mu \lambda_j)(\varphi_j,f) = (\varphi_j,g) \implies (\varphi_j,f) = \frac{(\varphi_j,g)}{1-\mu \lambda_j} \]
notiamo che abbiamo trovato i coefficienti di Fourier di $f$. Ora possiamo considerare $f = g+ \mu Af$ e applichiamo la iv. di Hilbert-Schmidt:
\[f = g + \mu \sum_{n=1}^\infty\lambda_n(\varphi_n,f)\varphi_n = g + \mu \sum_{n=1}^\infty\lambda_n\frac{(\varphi_n,g)}{1-\mu \lambda_n}\varphi_n \]
la dimostrazione del ii. si lascia per esercizio.
\subsection{Nuclei integrali}
Consideriamo lo spazio di Hilbert $\H = L_2(a,b)$, dato che gli operatori sono funzioni lineari può sembrare naturale rappresentarli in forma integrale:
\[(Af)(x) = \int_a^b A(x,y)f(y)\,dy\]
$A(x,y)$ si dice nucleo dell'operatore $A$. In generale il nucleo di un operatore non è una funzione, infatti consideriamo l'operatore identità su $L_2(\R)$
\[\mathbbm{1}f = \int_{-\infty}^{\infty} I(x,y)f(y)\,dy = f(x)\]
è evidente che in questo caso $I(x,y) = \delta(y-x)$ è una distribuzione, in generale i nuclei quindi appartengono agli spazi delle distribuzioni $A(x,y) \in \mathcal{S}'\times \mathcal{S}'$. \\
Ci si può chiedere quale sia il nucleo dell'operatore aggiunto:
\[A^\dagger f = \int_a^bA^*(x,y)f(y)\,dy\]
proviamo a ricavarlo sfruttando il fatto che $(A^\dagger f,g) = (f,Ag)$:
\[\int_a^b dx\, g(x)\int_a^bdy\,\overline{A^*(x,y)f(y)} = \int_a^bdx\, \overline{f(x)}\int_a^bdy\, A(x,y)g(y) \]
cambiando variabili $x\leftrightarrow y$ si arriva facilmente ad:
\[A^*(x,y) = \overline{A(y,x)}\]
allora è evidente che $A= A^\dagger$ quando $A^*(x,y) = A(x,y) = \overline{A(y,x)}$, in questo caso il nucleo si dice simmetrico. Ad esempio il nucleo $K(x,y) = xy$ con $x,y\in\R$ è simmetrico.
\begin{dfn}
Si chiamano i nuclei di Hilbert-Schmidt (H-S) gli elementi dell'insieme:
\[\mathbb{L}_2(a,b) = \left\{A(x,y) : \int_a^bdx\int_a^bdy|A(x,y)|^2<+\infty\right\}\]
dove l'integrale si intende nel senso di Lebesgue.
\end{dfn}
Si può definire una norma in tale insieme come:
\[\|A\|_{H-S} =  \int_a^bdx\int_a^bdy|A(x,y)|^2\]
una proprietà importante è che se $A$ è di Hibert-Schmidt allora $A(x,y)\in L_2$.
\\Mostriamo una disuguaglianza utile per il seguito:
\[|Af(x)|^2 = \left|\int_a^b A(x,y)f(y)\,dy\right|^2 \leq \|f\|^2 \int_a^b |A(x,y)|^2\]
dove è stata sfruttata la disuguaglianza di Schwarzt sapendo che $f,A(x,y)\in L_2$, quindi vale solo nel caso di nuclei di H-S.
\begin{lem}
Se $A(x,y)$ è di Hilbert-Schmidt allora $A$ è limitato e $\|A\| \leq \|A\|_{H-S}$
\end{lem}
\hspace{-1.6em}\textbf{Dimostrazione:}
\[\|Af\|^2  =\int_a^b|Af|^2\,dx \leq \|f\|^2 \int_a^bdx\int_a^bdy|A(x,y)|^2 \implies \|Af\| \leq \|f\|\|A\|_{H-S} \]
dove si è sfruttata la disuguaglianza appena dimostrata. La tesi segue banalmente dalla definizione di norma operatoriale.
\begin{thm}\label{nucleihs}
Ogni nucleo di H-S definisce un operatore compatto
\end{thm}
In particolare se il nucleo di H-S è simmetrico allore $A=A^\dagger$ è compatto.\\
Facciamo un esempio interessante e prendiamo $G=G^\dagger$ compatto, per Hilbert-Schmidt lo possiamo scrivere come $G = \sum_n \lambda_n P_n$, ovviamente vale che $G\varphi_n = \lambda_n\varphi_n$. Il nucleo di questo operatore è dato da:
\[G(x,y) = \sum_n \lambda_n \overline{\varphi}_n (y)\varphi_n(x)\] 
infatti verifichiamolo applicandolo a $\varphi_n$:
\[(G\varphi_n)(x) = \int_a^b G(x,y) \varphi_n(y)\,dy = \int_a^b \sum_m \lambda_m \overline{\varphi}_m (y)\varphi_m(x) \varphi_n(y)\,dy =\sum_m \lambda_m \varphi_m(x)\int_a^b \overline{\varphi}_m (y) \varphi_n(y)\,dy \]
\[ = \sum_m \lambda_m \varphi_m(x) ( \varphi_m, \varphi_n) = \lambda_n \varphi_n\]
Si può dimostare anche che vale al seguente:
\[\|G\|_{H-S} =\int_a^bdx\int_a^bdy|G(x,y)|^2 =\sum_n \lambda_n^2 \]
mettendo dentro il nucleo che abbiamo usato prima e facendo ``qualche'' conto.
\begin{dfn}
Si chiama traccia di $G$ la seguente:
\[\text{Tr}(G) = \int_a^b G(x,x)\,dx\]
\end{dfn}
se $\int_a^b G(x,x)\,dx<\infty$ si dice che $G$ ammette traccia.\\ Facciamo un esempio, $K= xy$ in $L_2(0,1)$ allora $\text{Tr}(K) =  \int_0^1 x^2\,dx = 1/3$.\\
Se l'operatore ha traccia e ha il nucleo simmetrico ($G=G^\dagger$) allora $\text{Tr}(G) = \sum_n \lambda_n$. Le traccie ci forniscono un criterio molto utile per stabilire la compattezza degli operatori, infatti se $G(x,y)$ è simmetrico e se ha traccia allora $G$ è compatto, la dimostrazione è semplice in quanto se ha traccia allora $\sum_n \lambda_n$ converge quindi converge anche $\sum_n \lambda_n^2$ quindi $G$ ha norma di Hilbert-Schmidt e quindi definisce un operatore compatto per il teorema \eqref{nucleihs}.\\ 
Un esercizio utile adesso è vedere che il sistema trigonometrico $\{e_n\}$ fornisce una base completa dello spazio di Hilbert, per farlo consideriamo l'operatore $P$ di cui $e_n$ sono gli autovettori. Mettiamoci in $L_2(0,2\pi)$ e prendiamo un dominio con condizioni al contorno simmetriche. L'idea è quella di sfruttare il corollario \eqref{basehilbertiana}, quindi dato che $P$ è non limitato proviamo a vedere se ha l'inverso e nel caso l'avesse dobbiamo vedere se è compatto, questa è l'idea che viene fuori dal corollario \eqref{inversocompatto}, come abbiamo fatto notare precedentemente, questo si può fare perchè gli operatori inversi mantengono le stesse proprietò spettrali. Procediamo allora per passi e vediamo se $P$ ha l'inverso, il suo Ker purtroppo non contiene solamente l'elemento nullo, contiene anche $e_0$ come è facile vedere: $P(1/\sqrt{2\pi}) = 0$ quindi $P$ non è invertibile. Non tutto è perso però, possiamo renderlo invertibile con una translazione, consideriamo infatti l'operatore $L = P + M\mathbbm{1}$ con $M\notin\mathbb{Z}$, gli autovettori di $L$ sono gli stessi di $P$ infatti:
\[L(e_n) = P(e_n) + Me_n = n e_n +Me_n = (n+M)e_n\]
quelli che cambiano sono gli autovalori, ma gli autovalori non influiscono sulla complettezza del sistema trigonometrico, quindi per dimostrare la loro completezza possiamo lavorare con $L$ al posto $P$. Vediamo se $L$ è invertibile andando a vedere il suo Ker
\[Lf = 0 \implies -i\frac{d}{dx}f+Mf = 0 \implies f(x) = Ce^{-iMx}\]
le condizioni al contorno ci dicono che $f(0)=f(2\pi)$ quindi $C = C e^{-iM2\pi}$ con $M$ non intero, l'esponenziale a destra non si annulla mai quindi deve essere per forza $C=0$ da cui l'unica soluzione $f=0$, unico elemento del Ker di $L$, quindi è invertibile. Ora dobbiamo solo trovare l'inverso $L^{-1} = G$ e vedere se è compatto. Per trovare l'inverso cerchiamo il nucleo di $G$, si può vedere che vale $LG(x,y) = \delta(x-y)$ con naturalmente $G(x,y)$ che soddisfa le condizioni al contorno, abbiamo perciò il seguente sistema
\[ \begin{cases}
    LG(x,y) = \delta(x-y)\\
    G(0,y) = G(2\pi,y) \quad \forall y\in[0,2\pi]\\
  \end{cases}\]
l'equazione da risolvere è quindi la seguente:
\[\left(-i\frac{d}{dx}+M\right)G(x,y) = \delta(x-y)\]
si tratta quindi di trovare la soluzione fondamentale dell'operatore $L$, essa sarà una combinazione lineare della soluzione anticipata e quella ritardata, quindi:
\[G(x,y) = e^{-iM x}[\theta(y-x)a(y) + \theta(x-y)b(y)]\]
dove $e^{-iM x}$ è semplicemente la soluzione dell'omogenea. Mettiamo questa $G(x,y)$ nel sistema e ricaviamo le condizioni su $a$ e $b$:
\[PG(x,y) = -i[-iMG(x,y) + e^{-iMx}\delta(x-y)(b(y)-a(y))]\]
da cui portando il primo termine a sinistra otteniamo proprio:
\[L = PG(x,y) +MG(x,y) = -i e^{-iMx}\delta(x-y)[b(y)-a(y)] = \delta(x-y) \implies -i e^{-iMx}[b(y)-a(y)] = 1\]
dalla condizione di periodicità otteniamo:
\[a(y) = e^{-iM2\pi}b(y)\]
risolvendo il sistema per $a$ e $b$ troviamo che alla fine la soluzione è:
\[G(x,y) = \frac{e^{iM(x-y)}}{2\sin \pi M}[\theta(y-x)e^{-i\pi M} + \theta(x-y)e^{i\pi M}]\]
questo nucleo è simmetrico come si può controllare facilmente ed inoltre è di Hilbert-Schmidt perchè la sua norma è convergente, facile da vedere anche questo dato che in $[0,2\pi]$ non c'è nessun elemento che diverge. Questo significa che per il teorema \eqref{nucleihs} $G$ definisce un operatore compatto quindi
per il corollario \eqref{basehilbertiana} $e_n$ è una base Hilbertiana. Si può notare che questo operatore ha anche traccia, infatti prendendo come convenzione $\theta(0) = \frac{1}{2}$ abbiamo che:
\[\text{Tr}(G) = \int_0^{2\pi} G(x,x)\,dx = \int_0^{2\pi}\frac{1}{4\sin \pi M}[e^{-i\pi M} + e^{i\pi M}]dx  = \frac{\pi}{\sin \pi M}\frac{(e^{-i\pi M} + e^{i\pi M})}{2}dx = \pi \text{cotan}(\pi M)\]
ma noi sappiamo anche che, sia $\lambda_n = n+ M$, vale:
\[\text{Tr}(G) = \sum_n \frac{1}{\lambda_n} = \sum_{n=-\infty}^{+\infty} \frac{1}{n+M} = \frac{1}{M} + \sum_{n=1}^\infty\left(\frac{1}{n+M} + \frac{1}{M-n}\right) = \frac{1}{M} + \sum_{n=1}^\infty\left(\frac{2M}{M^2-n^2}\right)\]
che possiamo valutare con il metodo dei residui, infatti:
\[\sum_{n = -\infty}^{+\infty}\frac{1}{M^2-n^2} = \frac{1}{M^2} + 2\sum_{n=1}^\infty \frac{1}{M^2+n^2} = -\pi\left(\sum_i\text{Res}\left(\text{cotan}(\pi z) \frac{1}{(M-z)(M+z)}\right)\right)\]
da cui:
\[\frac{1}{M^2} + 2\sum_{n=1}^\infty \frac{1}{M^2+n^2} = \frac{\pi}{M} \text{cotan}(\pi M)\]
dalla quale moltiplicando da entrambe le parti per $M$ si ottiene lo stesso risultato di prima.\\
Allo stesso modo si potrebbe vedere se l'operatore laplaciano $L=-\frac{d^2}{dx^2}$ in $L_2(0,1)$ ha una base Hilbertiana, ma questo si lascia per esercizio.\\
Troviamo ora una base Hilbertiana di $L_2(\R)$, per farlo individuiamo una operatore autoaggiunto compatto invertibile come abbiamo appena fatto, i suoi autovettori saranno una base Hilbertiana. Consideriamo l'operatore  $H = -\frac{d^2}{dx^2} +x^2$, questo operatore è simmetrico e con un dominio opportuno anche autoaggiunto, prendiamo $D_{H} = \{\psi,H\psi\in L_2(\R)\}$. Gli autovalori e gli autovettori di questo operatore sono:
\[H\psi = E \psi \quad E_n(2n+1)\quad n=0,1,2,\dots \quad \psi_n(x) = H_n(x)e^{-\frac{x^2}{2}}\]
con $H_n$ polinomi di Hermite (ignoriamo anche la costante di normalizzazione degli autovettori). Questo operatore è invertibile perchè non ci sono autovalori nulli, dobbiamo mostrare che l'inverso $G = H^{-1}$ è compatto, per farlo vediamo se è di H-S, ricordiamo che gli autovalori di $G$ sono $\lambda_n = 1/(2n+1)$
\[\|G\|^2_{H-S} = \sum_{n=-\infty}^{+\infty} \frac{1}{(2n+1)^2} <+\infty\]
quindi $G$ è di H-S quindi è compatto e simmetrico per cui $\psi_n = H_n(x)e^{-\frac{x^2}{2}}$ è un S.O.N.C. di $L_2(\R)$. Ma vediamo di costruirci tale operatore inverso, per semplificare i conti e avere un risultato consideriamo però l'inverso dell'operatore $L=H+\mathbbm{1} = -\frac{d^2}{dx^2} +x^2+1$ e indichiamo il suo operatore inverso con $K=L^{-1}$. Per trovare $K$ dobbiamo risolvere il seguente sistema con le condizione al contorno:
\[\begin{cases}
    LK(x,y) = \delta(x-y)\\
    K(-\infty,y) = K(+\infty,y) \quad \forall y\in\R\\
  \end{cases}\]
per trovare la soluzione fondamentale ricorriamo al seguente lemma
\begin{lem}
Sia $L = -\frac{d^2}{dx^2} + V(x)$ e siano due soluzioni dell'omogenea $L\psi_{1,2} = 0$ con wronskiano non nullo:
\[W = \begin{vmatrix}
  \psi_1 & \psi_2 \\
    \psi_1' & \psi_2'
 \end{vmatrix} = \psi_1 \psi_2' - \psi_2\psi_1' \neq 0\]
 allora la soluzione fondamentale di $L$ è data da:
 \[K(x,y) = -\frac{1}{W}[\theta(y-x)\psi_1(x)\psi_2(y) + \theta(x-y)\psi_1(y)\psi_2(x)]\]
\end{lem}
\hspace{-1.6em}\textbf{Dimostrazione:}\\
Calcoliamo la derivata seconda di $K$, consideriamo che il wronskiano è una costante, infatti:
\[\frac{dW}{dx} = \psi_1'\psi'_2 + \psi_1\psi''_2 - \psi_1'' \psi_2 -\psi_2\psi_1' = \psi_1\psi''_2- \psi_1'' \psi_2 = 0\]
usando il fatto che $\psi_{1,2}$ sono soluzioni dell'omogenea.
\[\partial_xK(x,y) = -\frac{1}{W}[\delta(x-y)[\psi_1(y)\psi_2(y) - \psi_1(y)\psi_2(y)] + \theta(y-x)\psi_1'(x)\psi_2(y) + \theta(x-y)\psi_1(y)\psi_2'(x)]\]
\[\partial^2_xK(x,y) =  -\frac{1}{W}[\delta(x-y)[\psi_1(y)\psi_2'(y)-\psi_2(y)\psi_1'(y)]+\theta(y-x)\psi_1''(x)\psi_2(y) + \theta(x-y)\psi_1(y)\psi_2'(x)   ]\]
\[=-\frac{1}{W}[\delta(x-y)W+\theta(y-x)\psi_1''(x)\psi_2(y) + \theta(x-y)\psi_1(y)\psi_2'(x)   ]\]
possiamo adesso calcolare $LK$ che con alcuni conti viene essere:
\[LK = \delta(x-y) + \theta(y-x)L\psi_1(x) \psi_2(y) + \theta(x-y)\psi_1(y)L\psi_2(x) = \delta(x-y) \]
\newline
grazie a questo lemma possiamo scrivere $K$ per il nostro operatore $L = -\frac{d^2}{dx^2} +x^2+1$, per farlo abbiamo bisogno di due soluzioni dell'equazione omogenea, in questo caso sono:
\[\psi_1(x) = e^{\frac{x^2}{2}}\int_{-\infty}^xdy\, e^{-y^2} \qquad \psi_2(x) = e^{\frac{x^2}{2}}\int_{x}^{+\infty}dy\, e^{-y^2}\]
è facile verificarlo calcolando le derivate seconde. Il wronskiano di queste due soluzioni non è nullo infatti:
\[W = e^{\frac{x^2}{2}}\int_{-\infty}^xdy\, e^{-y^2}\left[x e^{\frac{x^2}{2}}\int_{x}^{+\infty}dy\, e^{-y^2} - e^{-\frac{x^2}{2}} \right] - \left[xe^{\frac{x^2}{2}}\int_{-\infty}^xdy\, e^{-y^2} + e^{-\frac{x^2}{2}}\right]e^{\frac{x^2}{2}}\int_{x}^{+\infty}dy\, e^{-y^2}\]
\[= -\left(\int_{-\infty}^xdy\, e^{-y^2} + \int_{x}^{+\infty}dy\, e^{-y^2}\right) = -\int_{-\infty}^{+\infty}dy\, e^{-y^2} = -\sqrt{\pi}\]

verifichiamo che il $K$ con queste due soluzioni soddisfa le condizioni al contorno, considerando che quando x va all'infinito consideriamo solo una delle due $\theta$:
\[\lim_{x\to +\infty} K(x,y) = -\frac{\psi_1(y)}{W} \lim_{x\to +\infty} \psi_2(x) =-\frac{\psi_1(y)}{W}\lim_{x\to +\infty} \frac{\int_{x}^{+\infty}dy\, e^{-y^2}}{e^{\frac{-x^2}{2}}} = 0  \]
utilizzando de l'Hopital, nello stesso modo si fa il limite di $x$ che tende a $-\infty$.
\subsection{Teorema spettrale generale}
Il nostro obbiettivo è quello di generalizzare il teorema di Hilbert-Schmidt ad operatori non compatti, sappiamo da tale teorema che vale la decomposizione spettrale $Ah = \sum_n \lambda_n P_n h$ con $A$ autoaggiunto compatto, $h\in\H$ e $P_n$ gli operatori di proiezione. Prendiamo questa proiezione e proviamo a fare il prodotto scalare da entrambe le parti:
\[(\psi, A\psi) = \sum_n \lambda_n (\psi,P_n\psi) = \sum_n \int_{-\infty}^{+\infty}d\lambda\, \lambda \delta(\lambda -\lambda_n)(\psi,P_n\psi)=\]
portando la delta di Dirac dentro al differenziale, tenendo conto che vale $d\lambda\,\delta(\lambda-\lambda_n) = d(\theta(\lambda - \lambda_n))$, otteniamo:
\[ =\int_{-\infty}^{+\infty} \lambda\, d\left(\sum_n \theta(\lambda -\lambda_n)(\psi,P_n\psi)\right) = \int_{-\infty}^{+\infty} \lambda\, d(\psi,E_\lambda(A)\psi)\]
dove abbiamo definito la \textbf{famiglia spettrale} associata ad $A$ autoaggiunto $E_\lambda(A) = \sum_n \theta(\lambda-\lambda_n)P_n$. La famiglia spettrale ha le seguenti proprietà:
\begin{enumerate}[i.]
\item $E_\lambda^\dagger(A) = E_\lambda(A)$, dal fatto che i $P_n$ sono autoaggiunti
\item $E_\lambda(A)E_\mu(A) = E_\lambda(A)$ se $\lambda>\mu$, segue dal fatto che $P_nP_m = P_n\delta_{nm}$
\item $E_\lambda(A)\xrightarrow[\lambda\to -\infty]{\H} 0$ e $E_\lambda(A)\xrightarrow[\lambda\to +\infty]{\H} \mathbbm{1}$
\end{enumerate}
\begin{thm}(Teorema spettrale per operatori autoaggiunti o di Von-Neumann)
Se $A^\dagger = A$ in $\H$ allora esiste un unica famiglia spettrale $E_\lambda(A)$ tale che sia $f(\lambda)$ continua vale:
\[(\psi,f(A)\psi) = \int_{-\infty}^{+\infty}f(\lambda)d(\psi,E_\lambda(A)\psi) \]
con dominio $D_A = \{\psi,\displaystyle\int_{-\infty}^{+\infty} \lambda^2 d(\psi,E_\lambda(A)\psi) <+\infty\}$
\end{thm}
una cosa che abbiamo tralasciato fino ad ora è il senso dell'integrale con una funzione al differenziale, non si tratta dell'integrale di Riemann, ma di una sua generalizzazione, l'integrale di Stieltjes, vediamo di darne una formulazione più rigorosa.

\begin{figure}[H]
\centering
\begin{minipage}{.5\textwidth}
\centering
\begin{tikzpicture}
\draw [<->](0,0) -- (5,0)node[below=1.5pt] {\color{black}$t$};
\draw[->](0,0)--(0,3)node[right=1.5pt] {\color{black}$y$};
\draw[red] plot [smooth,tension=1] coordinates {(1,1) (2,1.5) (3,2.5) (4,2)};
\draw[dotted](1,0)node[below]{\scriptsize$a$} -- (1,1);
\draw[dotted](4,0)node[below]{\scriptsize$b$} -- (4,2);
\draw(2,2.2) node[below]{\scriptsize$g(t)$};
\end{tikzpicture}
\end{minipage}%
\begin{minipage}{.5\textwidth}
\centering
\begin{tikzpicture}
\draw [<->](0,0) -- (5,0)node[below=1.5pt] {\color{black}$t$};
\draw[->](0,0)--(0,3)node[right=1.5pt] {\color{black}$y$};
\draw[dotted](1,0)node[below]{\scriptsize$a$} -- (1,.7);
\draw[dotted](4,0)node[below]{\scriptsize$b$} -- (4,2);
\draw(2,2.2) node[below]{\scriptsize$\alpha(t)$};
\draw[blue](1,.7)--(2,.7) (2,1.3)--(3,1.3) (3,2)--(4,2) ; 
\end{tikzpicture}
\end{minipage}
\end{figure}
Sia $g(t)$ una funzione continua tra $[a,b]$ e $\alpha(t)$ una funzione non decrescente continua a tratti, partizioniamo l'intervallo $[a,b]$ insottointervalli $a = t_0 <t_1<t_2<\dots <t_n = b$, da ognuno di questi sottintervalli consideriamo il punto $\xi_k \in [t_k,t_{k+1}]$, sia $h$ la massima ampiezza degli intervalli $h = \underset{t_k}{\text{sup}}|t_{k+1}-t_k|$ allora possiamo definire le somme di Stieltjes:
\[S_h = \sum g(\xi_k)[\alpha(t_{k+1}) -\alpha(t_k)]\]
notiamo subito che scegliendo $\alpha(t) = t$ ci si riconduce alle somme dell'integrale di Riemann
\begin{thm}(Di Stieltjes)
Il seguente limite esiste:
\[\lim_{h\to 0} S_h = \int_a^b g(t)\,d\alpha(t) \]
e il suo limite viene detto integrale di Stieltjes.
\end{thm}
facciamo un esempio e prendiamo $\alpha(t) = t_1\theta(t-t_1) + t_2\theta(t-t_2)$, facciamo il seguente integrale:
\[\int_a^b t \,d\alpha(t) = \int_a^b t[t_1\delta(t-t_1) + t_2 \delta(t-t_2)]\,dt = t_1^2+t_2^2  \]
notiamo che l'integrale è stato ricondotto ad un integrale classico facendo uso delle distribuzioni, questo ci suggerisce che possiamo riformulare il teorema spettrale utilizzando le distribuzioni, si tratta di un approccio che viene utilizzato più frequentemente tra i fisici che permette di fare più agilmente i calcoli. Riscriviamo il teorema spettrale come segue:
\[f(A) = \int_{-\infty}^{+\infty}d\lambda\, \delta(\lambda-A)f(\lambda)\]
notiamo però che la delta di Dirac ha come argomento un operatore, dobbiamo quindi definire cosa sono le distribuzioni a valori operatoriali. 
La delta di dirac possiamo scriverla utilizzando le formule di Sokhotski:
\[\delta(x) = \frac{1}{2\pi i}\lim_{\varepsilon\to0}\left[\frac{1}{x-i\varepsilon} - \frac{1}{x+i\varepsilon} \right]\]
possiamo definire quindi la \textbf{densità spettrale} $\rho_\lambda(A)$ come delta a valori operatoriali come:
\[\rho_\lambda(A) = \delta(\lambda-A) = \frac{1}{2\pi i}\lim_{\varepsilon\to0}\left[\frac{1}{\lambda-i\varepsilon-A} - \frac{1}{\lambda+i\varepsilon -A} \right] = \frac{1}{2\pi i}\left[R_{\lambda-i\varepsilon}(A) - R_{\lambda+i\varepsilon}(A)\right] \]
dove $R_z(A)$ è la risolvente. Questa definizione è nota come formula di Kodaira-Stone-Titchmarsh (KST). Si può definire anche la famiglia spettrale come distribuzione a valori operatoriali, infatti:
\[E_\lambda(A) = \theta(\lambda-A) = \frac{1}{2\pi i}\lim_{\varepsilon\to0} \int_{-\infty}^\lambda[R_{\lambda'-i\varepsilon}(A) - R_{\lambda'+i\varepsilon}(A)]d\lambda' \]
Possiamo ricavare anche una formula per il nucleo di $f(A)$, infatti vale che:
\[f(A)(x,y) =  \int_{-\infty}^{+\infty}d\lambda\, \delta(\lambda-A)(x,y)f(\lambda)\]
si tratta di trovare il nucleo della densità spettrale $\rho_\lambda(A)(x,y) = \delta(\lambda-A)(x,y)$.
\begin{thm}(Formula di Kodaira-Stone-Titchmarsh)
\[\rho_\lambda(A)(x,y) = \frac{1}{2\pi i}\lim_{\varepsilon\to0}[G_{\lambda-i\varepsilon}(A)(x,y) - G_{\lambda+i\varepsilon}(A)(x,y)]\]
\end{thm}
dove $G_z(A)(x,y)$ è la funzione di Green definita da $(A-z)G_z(x,y) = \delta(x-y)$.\\ Nel caso più generale il nucleo di $f(A)$ è dato da:
\[f(A)(x,y) = \sum_n f(\lambda_n) P_n(x,y) + \int_{\sigma(A)}d\lambda\,f(\lambda)\rho_\lambda(A)(x,y)\]
dove $P_n(x,y)$ è il nucleo degli operatori di proiezione che si può calcolare agevolmente usando la notazione di Dirac:
\[P_n(x,y) = \braket{x|P_n|y}\quad P_n = \ket{\varphi_n}\bra{\varphi_n} \implies P_n(x,y) = \braket{x|\varphi_n}\braket{\varphi_n|y} = \overline{\varphi}_n(x)\varphi_n(y) \] 
Cerchiamo ora di trovare le famiglie spettrali e le densità spettrali degli operatori $Q$ e $P$.\\ Iniziamo da $Q$ in $L_2(0,2\pi)$, sappiamo che è limitato autoaggiunto ma non compatto, abbiamo già mostrato che la sua risolvente è $R_z(Q)f(x) = \frac{f(x)}{z-x}$ e che il suo spettro è continuo $z\in[0,2\pi]$
per la formula KST la sua densità spettrale è:
\[\rho_\lambda(Q)f(x) =  \frac{1}{2\pi i}\lim_{\varepsilon\to0}\left[\frac{1}{\lambda-x-i\varepsilon} - \frac{1}{\lambda-x+i\varepsilon} \right]f(x)  = \delta(\lambda-x)f(x)\]
per la formula di Sokhotski. La famiglia spettrale di conseguenza è $E_\lambda(Q)f(x) = \theta(\lambda-x)f(x)$.\\
Studiamo $P$ sempre in $L_2(0,2\pi)$ con condizioni simmetriche al bordo $f(0)=f(2\pi)$, troviamo il nucleo della funzione di Green:
\[\begin{cases}
(P-z)G_z(x,y) = \delta(x-y)\\
G_z(0,y) = G_z(2\pi,y)
\end{cases}\]
il problema l'avevamo già risolto quando abbiamo dimostrato la completezza del sistema trigonometrico, la soluzione era:
\[G_z(x,y) = \frac{e^{i(x-y)z}}{2\sin(\pi z)} \left[\theta(y-x)e^{-i\pi z} + \theta(x-y)e^{i\pi z}\right]\]
notiamo che i poli della funzione sono in $\sin(\pi z) = 0$ ovvero per $z=n$ con $n\in\mathbb{Z}$, lo spttro quindi è discreto.\\
Più interessante è invece il caso dell'operatore $P$ in $L_2(\R)$ (l'operatore $Q$ in questo spazio ha le stesse caratteristiche spettrali che nello spazio $L_2(0,2\pi)$). In $L_2(\R)$ $P$ ha spettro continuo con autofunzioni $e^{ikx}$, per calcolare la famiglia spettrale passiamo allo spazio degli impulsi tramite trasformata di Fourier $\widetilde{Pf}(k) = k\widetilde{f}(k)$, notiamo che ci siamo ricondotti al caso di $Q$, la densità spettrale la famiglia spettrale quindi sono:
\[\widetilde{\rho f}(k) = \delta(\lambda-k)\widetilde{f}(k) \qquad \widetilde{E_\lambda(P)}(k) = \theta(\lambda-k)\widetilde{f}(k) \]
infatti la risolvente nello spazio degli impulsi è $\widetilde{R_z}(P) = \widetilde{f}(k)/(z-k)$. Per ritornare allo spazio delle configurazioni facciamo l'antitrasformata di Fourier:
\[E_\lambda(P)f(x) = \frac{1}{2\pi}\int_{-\infty}^{+\infty}dk\, e^{ikx}\theta(\lambda-k)\widetilde{f}(k) =  \frac{1}{2\pi}\int_{-\infty}^{\lambda}dk\, e^{ikx}\int_{-\infty}^{+\infty}dy\, e^{-iky}f(y)\]
cambiando ordine di integrazione:
\[= \int_{-\infty}^{+\infty}dy\,f(y)\frac{1}{2\pi}\int_{-\infty}^{\lambda}dk\, e^{ik(x-y)} =\int_{-\infty}^{+\infty}dy\,f(y)E_\lambda(P)(x,y) \]
dove abbiamo espresso la famiglia spettrale con il suo nucleo che abbiamo trovato essere:
\[E_\lambda(P)(x,y) = \frac{1}{2\pi}\int_{-\infty}^{\lambda}dk\, e^{ik(x-y)}\]
notiamo che il nucleo è simmetrico e soddisfa le 3 condizioni della famiglia spettrale.\\
Finiamo con un ultimo esempio sempre in $L_2(\R)$, vogliamo trovare la densità spettrale dell'hamiltoniano $H_0 = -\frac{d^2}{dx^2}$. Passimo nello spazio degli impulsi dove questo operatore diventa $\widetilde{H_0f}(k) = k^2\widetilde{f}(k)$ la risolvente è quindi:
\[\widetilde{R_zf}(k) = \frac{\widetilde{f}(k)}{z-k^2}\]
per la KST la densità spettrale vale:
\[\widetilde{\rho_\lambda(H_0)}f(k) = \frac{1}{2\pi i}\lim_{\varepsilon\to0}\left[\frac{1}{\lambda-k^2-i\varepsilon} - \frac{1}{\lambda-k^2+i\varepsilon} \right]\widetilde{f}(k) = \delta(\lambda-k^2)\widetilde{f}(k) = \]
\[\frac{1}{2\sqrt{\lambda}}\left[\delta(k-\sqrt{\lambda})+\delta(k+\sqrt{\lambda})\right]\widetilde{f}(k) = \frac{1}{2\sqrt{\lambda}}\left[\delta(k-\sqrt{\lambda})\widetilde{f}(\sqrt{\lambda})+\delta(k+\sqrt{\lambda})\widetilde{f}(-\sqrt{\lambda})\right]\]
notiamo che $\sqrt{\lambda}$ è ben definito e reale in quanto le autofunzioni di $H_0$ sono $e^{ikx}$ per cui gli autovalori sono $k^2$ che sono sempre positivi. Adesso ci rimane solo da antitrasformare:
\[\rho_\lambda(H_0)f(x) = \frac{1}{2\pi}\int_{-\infty}^{+\infty}dke^{ixk} \widetilde{\rho_\lambda(H_0)}f(k) = \frac{1}{4\pi\sqrt{\lambda}}\int_{-\infty}^{+\infty}dk e^{ixk}\left[\delta(k-\sqrt{\lambda})\widetilde{f}(\sqrt{\lambda})+\delta(k+\sqrt{\lambda})\widetilde{f}(-\sqrt{\lambda})\right]\]
\[= \frac{1}{4\pi\sqrt{\lambda}}\int_{-\infty}^{+\infty}dk\left[e^{ixk}\delta(k-\sqrt{\lambda})\int_{-\infty}^{+\infty}dy\,e^{-i\sqrt{\lambda}y}f(y)+e^{ixk}\delta(k+\sqrt{\lambda})\int_{-\infty}^{+\infty}dy\,e^{i\sqrt{\lambda}y}f(y)\right] = \]
\[\frac{1}{4\pi\sqrt{\lambda}}\left[e^{ix\sqrt{\lambda}}\int_{-\infty}^{+\infty}dy\,e^{-i\sqrt{\lambda}y}f(y)+e^{-ix\sqrt{\lambda}}\int_{-\infty}^{+\infty}dy\,e^{i\sqrt{\lambda}y}f(y)\right] = \]\[\frac{1}{4\pi\sqrt{\lambda}}\left[\int_{-\infty}^{+\infty}dy\,e^{i\sqrt{\lambda}(x-y)}f(y)+\int_{-\infty}^{+\infty}dy\,e^{-i\sqrt{\lambda}(x-y)}f(y)\right] = \int_{-\infty}^{+\infty}dy\,\rho_\lambda(H_0)(x,y)f(y)  \]
dove abbiamo scritto la densità utilizzando il suo nucleo che abbiamo trovato essere:
\[\rho_\lambda(H_0)(x,y) = \frac{1}{4\pi\sqrt{\lambda}}\left[e^{i\sqrt{\lambda}(x-y)}+e^{-i\sqrt{\lambda}(x-y)}\right]\]

\section{Gruppi abeliani di trasformazioni}
Se prendiamo $\alpha\in\R$ possiamo considere il numero $e^{i\alpha}$, che ha $|e^{i\alpha}|=1$, esso è di particolare importanza come è ben noto. In uno spazio di Hilbert questa quantità ha un suo analogo, consideriamo infatti l'operatore $A$ e studiamo meglio l'operatore $U_t = e^{-itA}$ con $A$ operatore autoaggiunto in uno spazio di Hilbert $\H$, dove $t\in\R$, vediamo innanzitutto che è unitario, ovvero conserva la norma, infatti:
\[\|U_t\psi\|^2 = (U_t\psi,U_t\psi) = (e^{-itA}\psi,e^{-itA}\psi) = (\psi, e^{-itA}e^{itA}\psi) = (\psi,\psi) = \|\psi\|^2\]
notiamo che in generale $e^Ae^B\neq e^{A+B}$ con $A,B$ operatori, l'uguaglianza c'è solo nel caso i cui i due operatori commutano tra di loro, nel nostro caso $A$ commuta certamente con se stesso per cui abbiamo potuto sfruttare questa proprietà. $U_t$ viene chiamata trasformazione unitaria e si dice che $A$ è il \textbf{generatore} della trasformazione unitaria. L'unitarietà di tale operatore si poteva dimostrare anche sfruttando il seguente lemma che ci sarà piuttosto utile in seguito tra l'altro
\begin{lem}
\label{lemmautilesullenorme}
\[\|f(A)\psi\|^2 = \int_{-\infty}^{+\infty}|f(\lambda)|^2(\psi,dE_\lambda(A)\psi)\]
\end{lem}
verifichiamo questo lemma nel caso di spettro discreto, dove il teorema spettrale ci dice che $f(A) = \sum_n f(\lambda_n) P_n $.
\[\|f(A)\psi\| = (f(A)\psi,f(A)\psi) = \left(\sum_n f(\lambda_n) P_n\psi,\sum_m f(\lambda_m) P_m\psi\right) = \sum_{n,m}\overline{f(\lambda_n)}f(\lambda_m)(P_n\psi,P_m\psi) = \]
\[\sum_{n,m}(\psi,P_nP_m\psi) = \sum_n |f(\lambda_n)|^2(\psi,P_n\psi)\]
Verifichiamo sfruttando questo lemma che $U_t$ è unitaria, ricordiamo che per il teorema spettrale vale che $f(A) = \int_{-\infty}^{+\infty}f(\lambda)\,dE_\lambda(A)$, quindi:
\[e^{-itA} = \int_{-\infty}^{+\infty}e^{-it\lambda}\,dE_\lambda(U_t) \quad \|U_t\psi\|^2 = \int_{-\infty}^{+\infty}|e^{-it\lambda}|^2(\psi,dE_\lambda(U_t)\psi) =  \int_{-\infty}^{+\infty}(\psi,dE_\lambda(U_t)\psi) = (\psi,\mathbbm{1}\psi)\]
$U_t$, oltre che essere unitaria, è un \textbf{gruppo abeliano di trasformazioni unitarie fortemente continue} (GATUC), soddisfa infatti la legge di gruppo commutativa:
\[U_tU_s = U_{t+s} = U_{s+t} = U_sU_t\]
che vale per l'osservazione di prima sulla proprietà degli esponenziali. Nel gruppo esiste anche l'inverso, infatti è ovvio che vale:
\[U_t^{-1} = U_{-t} \qquad U_tU_{-t} = \mathbbm{1}\]
gli aggettivi fortemente continua vengono dal fatto che nello spazio di Hilbert la trasformazione tende fortemente all'operatore identità $U_t \xrightarrow[t\to 0]{\H} \mathbbm{1}$, mostriamolo verificando che la quantità $\|U_t\psi - \mathbbm{1}\psi\|$ tenda a 0 quando $t$ tende a 0
\[\|e^{-itA}\psi -\psi\|^2 = \|(e^{-itA}-\mathbbm{1})\psi\|^2 = \int |e^{-it\lambda}-1|^2(\psi,dE_\lambda\psi) \]
dove abbiamo sfruttato il lemma \eqref{lemmautilesullenorme}. Sarebbe carino ora passare sotto il segno di integrale con il limite e arrivare alla conclusione, per farlo consideriamo che l'integrale sia di Lebesgue e sfruttiamo il teorema della convergenza dominata
\begin{thm}
(Della convergenza dominata)
se $|f_n(x)|\leq g(x)$ con $g(x)$ sommabile nel senso di Lebesgue allora vale:
\[\lim_{n\to+\infty} \int f_n(x)\,dx = \int \lim_{n\to+\infty} f_n(x)\,dx\]
\end{thm}
nel nostro caso la funzione è dominata in quanto $|e^{-it\lambda}-1|\leq 4$ per la \eqref{disutileinseguito}, dato che 4 è sommabile secondo Lebesgue vale il teorema e possiamo dire che:
\[\lim_{t\to0}\int |e^{-it\lambda}-1|^2(\psi,dE_\lambda\psi)  = \int \lim_{t\to 0}|e^{-it\lambda}-1|^2(\psi,dE_\lambda\psi) = 0\]
Per calcolare il generatore della trasformazione possiamo sfruttare la seguente:
\begin{equation}\label{generatore}
-iA = \text{s-}\lim_{t\to 0} \frac{U_t-\mathbbm{1}}{t}
\end{equation}
dove con $\text{s-}\lim$ si intende il limite forte. Dimostriamo questa relazione
\[\lim_{t\to0} \|\left(\frac{U_t-\mathbbm{1}}{t} +iA\right)\psi\|^2 = \lim_{t\to 0}\int\left|\frac{e^{-it\lambda}-1}{t} +i\lambda\right|^2(\psi,dE_\lambda\psi)\]
dove abbiamo nuovamente usato il lemma \eqref{lemmautilesullenorme}, vediamo se la funzione è dominata in modo da usare di nuovo il teorema della convergenza dominata:
\[ \left|\frac{e^{-it\lambda}-1}{t} +i\lambda\right|^2\leq 2\left|\frac{e^{-it\lambda}-1}{t}\right|^2 + 2\lambda^2 = 2\lambda^2\left|\frac{e^{-it\lambda}-1}{t\lambda}\right|^2+ 2\lambda^2  = 2\lambda^2\left[\left|\frac{e^{-i\alpha}-1}{\alpha}\right|^2+1  \right] \leq 4\lambda^2\]
dobbiamo vedere che $4\lambda^2$ è sommabile
\[4\int\lambda^2(\psi,dE_\lambda\psi) = 4\|A\psi\|^2<+\infty\]
per il teorema spettrale. Passiamo quindi passare sotto il segno di integrale e otteniamo:
\[\int\lim_{t\to 0}\left|\frac{e^{-it\lambda}-1}{t} +i\lambda\right|^2(\psi,dE_\lambda\psi) = 0\]
Riprendendo l'anaolgia con la quantità $e^{i\alpha}$ sui numeri complessi, è ben noto che un numero $u\in\C$ con modulo unitario $|u|=1$ è possibile esprimerlo tramite $u = e^{i\alpha}$, questà proprietà ha un analogo negli spazi di Hilbert che è data dal seguente teorema
\begin{thm}(Di Stone)
Se $U_t$ è un GATUC in $\H$ separabile allora esiste $A=A^\dagger$ con $\overline{D_A} =\H $ tale che $U_t = e^{-itA}$ e $A$ si può trovare con la \eqref{generatore}
\end{thm}
da questo teorema deriva il seguente corollario particolarmente utile ed importante per la meccanica quantistica.
\begin{coro}
Se $\psi_t = U_t\psi$ $\forall \psi\in\H$ allora 
\[i\frac{d\psi_t}{dt} = A\psi_t\]
\end{coro}
\hspace{-1.6em}\textbf{Dimostrazione:}\\
la definizione di derivata nello spazio di Hilbert è la seguente:
\[\frac{d}{dt}\psi_t = \text{s-}\lim_{\varepsilon\to0} \frac{\psi_{t+\varepsilon} -\psi_t}{\varepsilon}\]
applicando il teorema di Stone al corollario otteniamo:
\[i\frac{d\psi_t}{dt}  = i\,\text{s-}\lim_{\varepsilon\to0} \frac{U_{t+\varepsilon}\psi -U_t\psi}{\varepsilon} = i\,\text{s-}\lim_{\varepsilon\to0} \frac{U_tU_\varepsilon\psi -U_t\psi}{\varepsilon} = \text{s-}\lim_{\varepsilon\to0} i \frac{(U_{\varepsilon} - 1)}{\varepsilon}U_t\psi = A\psi_t  \]
\\
Vediamo l'importanza di questo corollario in fisica considerando un sistema quantistico non relativistico descritto da $\psi\in\H$ e assumiamo che l'evoluzione nel tempo $t>0$ sia data da $U_t$ GATUC (postulato di evoluzione), allora il teorema di Stone e il corollario ci dicono che esiste un operatore autoaggiunto $H=H^\dagger$ tale che
\[\psi_t = e^{-\frac{it}{\hbar}H}\psi \iff i\hbar \frac{d\psi_t}{dt} = H\psi_t\]
che è l'equazione di Schr{\"o}dinger dove abbiamo reintrodotto la costante $\hbar$ che abbiamo sempre tenuto ad 1. Questo risultato è limitato alle Hamiltoniane indipendenti dal tempo, esistono alcune formulazioni più complesse che ne tengono conto, ma spesso i fisici in questi casi preferiscono postulare l'equazione di Schr{\"o}dinger.\\
\newline Studiamo a questo punto un altro importante operatore, quello delle traslazioni di coordinate. Mettiamoci in $L_2(\Sigma)$ con $\Sigma = \R,\R^+,[0,2\pi]$, sia $\psi(x)\in L_2(\Sigma)$ definiamo l'operatore come:
\[T_a\psi = \psi(x-a)\]
mostriamo che $T_a$ è un gruppo abeliano fortemente continuo, la commutatività viene da:
\[T_aT_b = T_a(\psi(x-b)) = \psi(x-b-a)\]
\[T_bT_a = T_b(\psi(x-a)) = \psi(x-a-b)\]
che sono chiaramente uguali. L'inverso è dato da:
\[T_a^{-1} = T_{-a} \qquad  T_{-a}T_a\psi = \psi  \]
verifichiamo infine la convergenza forte
\[\|T_a-\psi\|^2 = \int_\Sigma|\psi(x-a)-\psi(x)|^2dx = \int_\Sigma |\psi(x)-a\psi'(\xi) -\psi(x)|^2dx = a^2\|\psi'(\xi)\|^2\xrightarrow[a\to0]{}0\]
grazie al teorema di Lagrange. Troviamo ora il generatore di questa trasformazione, chiamiamo il generatore non a caso $P$ e facciamo un tentativo provando con $P = -i\frac{d}{dx}$ (l'intuizione potrebbe venire dal teorema di Nother che ci garantisce la conservazione della quantità di moto per simmetrie traslazionali)
\[ \|\left(\frac{T_a-\mathbbm{1}}{a} +iP\right)\psi\|^2 = \int_\Sigma \left|\frac{T_a\psi-\psi}{a} +\psi' \right|^2dx = \int_\Sigma \left|\frac{\psi(x-a)-\psi(x) +a \psi'(x)}{a}\right|^2dx = \]
sviluppando con taylor
\[\int_\Sigma \left|\frac{\psi(x)-a\psi'(x) +O(a^2) -\psi(x)+  a \psi'(x)}{a}\right|^2dx = \int_\Sigma \left|\frac{O(a^2)}{a}\right|^2dx\xrightarrow[a\to0]{}  0\]
notiamo che fino ad adesso abbiamo dimostrato solo la commutatività e la forte continuità, abbiamo tralasciato l'unitarietà in quanto a seconda di $\Sigma$, $T_a$ può essere unitario o meno, infatti la norma della trasformazione vale:
\[\|T_a\psi\|^2 = \int_\Sigma|\psi(x-a)|^2dx\]
facciamo ora i vari casi
\begin{itemize}
\item caso $\Sigma=\R$:
\[\|T_a\psi\|^2 = \int_{-\infty}^{+\infty}|\psi(x-a)|^2dx = \int_{-\infty}^{+\infty}|\psi(y)|^2dy = \|\psi\|^2\]
in questo caso $T_a$ è unitario grazie all'invarianza della misura per traslazioni, quindi grazie al teorema di Stone possiamo concludere che il generatore $P$ è autoaggiunto in $L_2(\R)$, risultato che avevamo già ottenuto in maniera differente.
\item caso $\Sigma=\R$:
\[\|T_a\psi\|^2 = \int_{0}^{+\infty}|\psi(x-a)|^2dx \neq \|\psi\|^2\]
quindi $T_a$ non è unitario in questo caso e di conseguenza $P$ non è autoaggiunto come ci aspettavamo.
\item caso $\Sigma=[0,2\pi]$:
\[\|T_a\psi\|^2 = \int_{0}^{2\pi}|\psi(x-a)|^2dx = \int_{-a}^{-a+2\pi}|\psi(y)|^2dy =\int_{0}^{2\pi}|\psi(y)|^2dy =\|\psi\|^2\]
questo solo perchè per condizioni sul dominio di $P$ dobbiamo prendere $\psi$ con condizioni periodiche $\psi(0) = \psi(2\pi)$, quindi l'integrale su funzioni periodiche è invariante per traslazioni, concludiamo che in questo caso $T_a$ è unitaria e quindi $P$ è autoaggiunto per Stone. Notiamo che essendoci nell'integrale il modulo quadro della $\psi$ avremmo potuto anche scegliere delle condizioni al contorno tali che $\psi(0) = e^{i\alpha}\psi(2\pi)$ con $\alpha\in\R$, e avremmo ottenuto lo stesso risultato. Questo è perfettamente in accordo con quanto ricavato quando abbiamo usato il teorema di Von-Neumann per costruire le infinite estensioni autoaggiunte dell'operatore $P$.
\end{itemize}
Un altro operatore da studiare in $L_2(\R)$ è il seguente $(U_\lambda\psi)(x) = e^{\lambda/2}\psi(e^\lambda x)$ con $\lambda\in\R$. Dimostriamo che $U_\lambda$ è un GUTAC e lasciamo per esercizio il calcolo del suo generatore $D$. L'operatore è unitario in quanto:
\[\|U_\lambda\psi\|^2 = \int_{-\infty}^{+\infty}dx e^{\lambda}|\psi(e^\lambda x)|^2\]
facendo un cambio di coordinate $e^\lambda x = y \implies dx = e^{-\lambda} dy$ arriviamo a
\[\|U_\lambda\psi\|^2 = \int_{-\infty}^{+\infty}dy|\psi(y)|^2 = \|\psi\|^2\]
la commutatività sfrutta la proprietà degli esponenziali che abbiamo usato ad inizio paragrafo
\[U_{\lambda_1}U_{\lambda_2} = U_{\lambda_1}(e^{\lambda_1/2}\psi(e^{\lambda_1} x)) = e^{\lambda_2/2}e^{\lambda_1/2}\psi(e^{\lambda_2}e^{\lambda_1} x) = U_{\lambda_2}U_{\lambda_1} = U_{\lambda_1+\lambda_2}\]
l'inverso è dato semplicemente da $U_{-(\lambda_1+\lambda_2)}$
\chapter{Applicazioni alla fisica}
\section{Principio di indeterminazione di Heisenberg}
Il famoso principio di indeterminazione della meccanica quantistica può essere visto come teorema nella formulazione matematica qui presentata, vediamo come. Per prima cosa introduciamo la definizione di indeterminazione di un operatore simmetrico, prendiamo quindi $A=A^\dagger$ e $B=B^\dagger$, ricordiamo la definizione di valore d'attesa $\braket{A}_\psi = (\psi,A\psi)$.
\begin{dfn}
Si chiama scarto quadratico o indeterminazione di un operatore $A$ la quantità:
\[\Delta_\psi A = \|\hat{A}\psi\| \quad\text{con}\quad \hat{A} = A-\braket{A}_\psi\mathbbm{1}\]
\end{dfn}
e facile vedere che vale:
\[(\Delta_\psi A)^2 = \|\hat{A}\psi\|^2 = (A\psi-\braket{A}_\psi \psi,A\psi-\braket{A}_\psi \psi) = \|A\psi\|^2 + (A\psi,-\braket{A}_\psi \psi) + (-\braket{A}_\psi \psi,A\psi) + \braket{A}_\psi^2\|\psi\|^2\]
\[ = \|A\psi\|^2 +  \braket{A}_\psi^2\|\psi\|^2 - \braket{A}_\psi (A\psi,\psi) - \braket{A}_\psi (\psi,A\psi) = \|A\psi\|^2 +  \braket{A}_\psi^2\|\psi\|^2 -2\braket{A}_\psi^2 \]
ora tenendo conto che nel caso fisico la funzione d'onda è normalizzata quindi $\|\psi\|^2 = 1$ ed inoltre notiamo che il primo termine diventa:
\[\|A\psi\|^2 = (A\psi,A\psi) = (\psi,AA\psi) = (\psi,A^2\psi) = \braket{A^2}_\psi\]
otteniamo come conclusione:
\[(\Delta_\psi A)^2 =\braket{A^2}_\psi -\braket{A}_\psi^2 \]
Consideriamo ora la quantità $(\hat{A}\psi,\hat{B}\psi) - (\hat{B}\psi,\hat{A}\psi)$ calcoliamo il suo valore, iniziando dal primo termine e notando che il secondo è il primo con $A$ e $B$ scambiati:
\[(\hat{A}\psi,\hat{B}\psi)  = (A\psi-\braket{A}_\psi\psi,B\psi-\braket{B}_\psi\psi) = (A\psi,B\psi) - 2\braket{A}_\psi\braket{B}_\psi + \braket{A}_\psi\braket{B}_\psi \|\psi\|^2\]
\[= (A\psi,B\psi) - \braket{A}_\psi\braket{B}_\psi \]
otteniamo perciò che:
\[(\hat{A}\psi,\hat{B}\psi) - (\hat{B}\psi,\hat{A}\psi) =(A\psi,B\psi) - \braket{A}_\psi\braket{B}_\psi - (B\psi,A\psi) + \braket{B}_\psi\braket{A}_\psi = (A\psi,B\psi) -(B\psi,A\psi) \]
a questo punto potremmo scambiare gli operatori, ma ci potrebbereo essere problemi di dominio per cui li lasciamo cosi come sono. Prendiamo il modulo di questa quantità e maggioriamola:
\[|(A\psi,B\psi) -(B\psi,A\psi)| = |(\hat{A}\psi,\hat{B}\psi) - (\hat{B}\psi,\hat{A}\psi)| \leq |(\hat{A}\psi,\hat{B}\psi)| +|(\hat{B}\psi,\hat{A}\psi)| \]
la quale può essere maggiorata con Schwarzt e otteniamo:
\[ |(A\psi,B\psi) -(B\psi,A\psi)| \leq 2 \|\hat{A}\psi\|\|\hat{B}\psi\| = 2\Delta_\psi A \Delta_\psi B \]
da cui quindi:
\[\Delta_\psi A \Delta_\psi B\geq \frac{1}{2}|(A\psi,B\psi) -(B\psi,A\psi)| = \frac{1}{2}|(\psi,C\psi)| \]
dove l'uguaglianza con il commutatore $C=[A,B]$ si può fare solamente prestando particolare attenzione al dominio.\\
Se come opertori prendiamo $Q$ e $P$ (prendiamo in questo caso $P$ definito anche con la costante di Planck ridotta) in $L_2(\R)$ dove entrambi sono autoaggiunti la disuguaglianza diventa:
\[\Delta_\psi Q \Delta_\psi P\geq \frac{1}{2}|(P\psi,Q\psi) -(Q\psi,P\psi)| = \frac{\hbar}{2}\left|-i\int_{-\infty}^{+\infty}(\overline{\psi}'x\psi - x\overline{\psi}\psi')\,dx\right|\]
andiamo avanti con i conti dell'integrale facendolo per parti:
\[-i\int_{-\infty}^{+\infty}(\overline{\psi}'x\psi - x\overline{\psi}\psi')\,dx = -i\int_{-\infty}^{+\infty}(x(\overline{\psi}\psi)'\,dx = -i\overline{\psi}\psi\Big|_{-\infty}^{+\infty} +i\int_{-\infty}^{+\infty}\overline{\psi}\psi\,dx =\]
\[i\int_{-\infty}^{+\infty}|\psi|^2dx = i\]
quindi concludiamo che:
\[\Delta_\psi Q \Delta_\psi P\geq \frac{\hbar}{2}\]
\section{Operatore tempo}
In meccanica quantistica può venire naturale voler introdurre il tempo come operatore, quindi come osservabile, questa cosa non avviene in quanto si può dimostrare che un tale operatore non è autoaggiunto. Iniziamo considerando l'equazione di evoluzione del valore medio di un operatore $A$ autoaggiunto:
\[\frac{d}{dt}\braket{A} = \frac{i}{\hbar}\braket{[H,A]}\]
dove $t$ è un parametro di evoluzione temporale ed $H$ è l'hamiltoniana del sistema. Notiamo che tale equazione è uguale a quella classica con al posto delle parentesi di Poisson il commutatore. Se introduciamo l'operatore $T$ canonicamente coniugato ad $H$ ovvero $[T,H] = i\hbar$ si può vedere che sarebbe consistente con il parametro $t$ di evoluzione temporale, infatti:
\[\frac{d}{dt}\braket{T} = -\frac{i}{\hbar}\braket{[T,H]} = 1  \implies \braket{T} = t + c\]
l'operatore $H$ si può mostrare essere autoaggiunto e inferiormente limitato, ovvero: $\braket{H}\geq 0$, per il teorema spettrale quindi esiste una trasformazione unitaria tale che $H$ è diagonale questo implica che possiamo scrivere la relativa equazione agli autovalori:
\[H\psi(E) = E\psi(E)\]
se introduciamo l'operatore $T$ si può vedere che otteniamo:
\[T= -i\hbar\frac{\partial}{\partial E}\]
questo operatore è simmetrico, ma utilizzando il teorema di Von-Neumann, si dimostra non avere estensioni autoaggiunte, quindi non è un osservabile della meccanica quantistica.
\section{Operatori di Schr{\"o}dinger}
\subsection{Potenziali elementari}
Mettiamoci nello spazio di Hilbert $L_2(\Sigma)$ con $\Sigma=\R,\R^+,[0,1]$ e studiamo diversi opertori Hamiltoniani $H = -\nabla^2 + V(\mathbf{x})$. Notiamo che innanzitutto $H$ è simmetrico solo quando $V(\mathbf{x})$ è una quantità reale, inoltre non è limitato quindi per evitare problemi ci prendiamo come dominio $C_0^\infty(\Sigma)$: le funzioni infinitamente derivabili e a supporto compatto. 
\begin{itemize}
\item caso $V(\mathbf{x}) = 0$, particella libera.\\
\newline In questo caso l'operatore di Hamilton è semplicemente $H_0 = -\nabla^2$ e in una dimensione dove noi lavoreremo per semplicità vale $H_0 = -\frac{d^2}{dx^2}$. Questo operatore è simmetrico, vediamo se è autoaggiunto. Per determinarlo ricorriamo al teorema di Von-Neumann, calcoliamo perciò gli indici di difetto:
\[H_0U_\pm(x) = \mp iU_\pm(x) \implies U''_\pm \pm iU_\pm = 0 \]
si tratta dell'equazione di un oscillatore armonico le cui soluzioni sono:
\[\begin{cases}
U_+(x) = a_+e^{\sqrt{i}x} + b_+e^{-\sqrt{i}x} \\
U_-(x) = a_-e^{\sqrt{-i}x} + b_-e^{-\sqrt{-i}x} 
\end{cases} 
\implies
\begin{cases}
U_+(x) = a_+e^{\frac{x}{\sqrt{2}}}e^{\frac{ix}{\sqrt{2}}} + b_+e^{-\frac{x}{\sqrt{2}}}e^{-\frac{ix}{\sqrt{2}}} \\
U_-(x) = a_-e^{\frac{x}{\sqrt{2}}}e^{\frac{-ix}{\sqrt{2}}} + b_-e^{-\frac{x}{\sqrt{2}}}e^{\frac{ix}{\sqrt{2}}} 
\end{cases}
\]
facciamone la norma per stabilire quando converge:
\[|U_\pm|^2 = a_\pm^2e^{\sqrt{2}x} + b_\pm^2e^{-\sqrt{2}x} + 2\text{Re}(a_\pm b_\pm) \implies \|U_\pm\|^2 = \int_\Sigma|U_\pm|^2dx\]
vediamo i vari casi  di $\Sigma$
\begin{enumerate}
\item caso $\Sigma = \R$:\\
la norma di $U_\pm$ in questo caso vale:
\[ \|U_\pm\|^2 = \int_{-\infty}^{+\infty}[a_\pm^2e^{\sqrt{2}x} + b_\pm^2e^{-\sqrt{2}x} + 2\text{Re}(a_\pm b_\pm)]dx\]
abbiamo gli esponenziali che divergono sia da una parte che dall'altra della retta reale per cui necessariamente deve essere $a_\pm=b_\pm = 0$ quindi gli indici di difetto sono uguali e nulli, cioè $H_0$ è autoaggiunto. Questo significa che l'energia cinetica è una quantità osservabile.
\item caso $\Sigma = \R^+$:\\
\[ \|U_\pm\|^2 = \int_{0}^{+\infty}[a_\pm^2e^{\sqrt{2}x} + b_\pm^2e^{-\sqrt{2}x} + 2\text{Re}(a_\pm b_\pm)]dx\]
in questo caso il termine che diverge all'infinito è il primo quindi deve essre $a_\pm = 0$, questo significa che gli indici di difetto sono uguali e valgono 1, esistono infinite estensioni autoaggiunte. Si possono trovare come abbiamo fatto nel caso di $P$ e otteniamo il seguente domino:
\[D_{H_0} = \{\varphi,H_0\varphi\in L_2(\R^+), \varphi(0) = \beta \varphi'(0)\}\]
la condizione al contorno è detta condizione di Robin.
\item caso $\Sigma = [0,1]$:\\
l'integrale in questo caso non ha nessun problema di convergenza quindi $n_+=n_-=2$, esistono infinite estensioni autoaggiunte parametrizzate da una matrice unitaria di dimensione $2\times2$. La condizione del dominio si può dimostrare essere:
\[\begin{pmatrix}
  \varphi(0) +i \varphi'(0) \\
  \varphi(1) -i \varphi'(1)
 \end{pmatrix} = \begin{pmatrix}
  a &b\\
  c & d
 \end{pmatrix}\begin{pmatrix}
  \varphi(0) -i \varphi'(0) \\
 \varphi(1) +i \varphi'(1)
 \end{pmatrix}\]
 con la matrice $2\times2$ unitaria, quindi con 4 parametri reali liberi. Alcuni esempi di matrici unitarie che si possono utilizzare sono:
 \[UU^\dagger = \begin{pmatrix}
  e^{i\alpha} &0\\
  0 & e^{i\beta}
 \end{pmatrix} =\begin{pmatrix}
  e^{-i\alpha} &0\\
  0 & e^{-i\beta}
 \end{pmatrix} = \begin{pmatrix}
  1 &0\\
  0 & 1
 \end{pmatrix}\quad \alpha,\beta=\R\]
 con la quale otteniamo le condizioni di Robin ai bordi:
 \[\begin{cases}
 \varphi(0) = \tan\left(\frac{\alpha}{2}\right) \varphi'(0) \\
 \varphi(1) = \tan\left(\frac{\beta}{2}\right) \varphi'(1)
\end{cases} \]
che con $\alpha=\beta=0$ si riducono alle condizione di Richlet di cui è un bell'esercizio calcolarne lo spettro associato. Oppure la matrice unitaria:
\[UU^\dagger = \begin{pmatrix}
  0 &e^{i\alpha}\\
  e^{-i\alpha} & 0
 \end{pmatrix} =\begin{pmatrix}
   0 &e^{i\alpha}\\
  e^{-i\alpha} & 0
 \end{pmatrix} = \begin{pmatrix}
  1 &0\\
  0 & 1
 \end{pmatrix}\quad \alpha\in\R\]
con la quale otteniamo le condizioni al contorno periodiche generalizzate:
\[\begin{cases}
 \varphi(0) = e^{-i\frac{\alpha}{2}} \varphi'(0) \\
 \varphi(1) = e^{-i\frac{\alpha}{2}} \varphi'(1)
\end{cases} \]
di cui per $\alpha=0$ potrebbe essere un simpatico esercizio trovare lo spettro dell'operatore.
\end{enumerate}
\item caso $V(x) = \xi x$ con $\xi\in\R$, studiamo solo il caso $\Sigma=\R$:\\ \newline
L'equazione di Von-Neumann associata è in questo caso:
\[-U_\pm''(x) + \xi x U_\pm(x) = \mp i U_\pm(x)\]
le cui soluzioni sono le cosiddette funzioni di Airy, complicate da studiare. Possiamo ricorrere ad un trucco però tenendo conto che gli indici di difetto sono invarianti sotto trasformazioni unitarie come ad esempio la trasformata di Fourier, questo ci permette di passare l'equazione nello spazio $\tilde{L}_2(\R)$ dove l'equazione e le sue soluzioni sono molto più semplici da studiare. Vediamo come si trasformano i vari membri:
\[\F\left(-\frac{d^2}{dx^2}U_\pm\right) = p^2\tilde{U}_\pm(p)\quad \F(\xi x U_\pm) = \xi i \frac{d\tilde{U}_\pm}{dp}\]
l'equazione diventa:
\[p^2\tilde{U}_\pm(p) +\xi i \frac{d\tilde{U}_\pm}{dp} = \mp i \tilde{U}_\pm(p)\implies \xi\frac{d\tilde{U}_\pm}{dp}=(\mp1+ip^2)\tilde{U}_\pm\]
che si può svolgere per separazione di variabili e otteniamo come soluzione:
\[\tilde{U}_\pm(p) = C_\pm e^{\mp\frac{p}{\xi}}e^{\frac{ip^3}{3\xi}}\]
la cui norma è:
\[\|\tilde{U}_\pm(p) \|^2 = |C_\pm|^2\int_{-\infty}^{+\infty} e^{\mp\frac{2p}{\xi}}e^{\frac{2ip^3}{3\xi}}\,dp \]
che ha dei seri problemi di convergenza, quindi ci tocca porre $C_\pm=0$ quindi sono nulli pure gli indici di difetto e quindi l'operatore è autoaggiunto.
\item caso $V(x) = \frac{g}{x^2}$ con $g\in\R$, studiamo solo il caso $\Sigma=\R^+$:\\ \newline
dato che il potenziale è reale l'operatore è simmetrico, scriviamo l'equazione di Von-Neumann:
\[-U''_\pm(x) + \frac{g}{x^2}U_\pm(x) = \mp i U_\pm(x)\implies U''_\pm(x) - \left(\pm i+\frac{g}{x^2}\right)U_\pm(x)=0\]
la soluzione di questa equazione sono:
\[U_\pm(x) = C_\pm\sqrt{x} K_\nu(\sqrt{\pm i}x)\]
dove $K_\nu(z)$ è la funzione di Macdonald (dette anche di Bessel modificate) con $\nu = \sqrt{\frac{1}{4}+g}$. Per studiare la convergenza dell'integrale ci servono i seguenti comportamenti asintotici di $K_\nu(z)$:
\[K_\nu(z)\xrightarrow[z\to +\infty]{} e^{-z} \qquad K_\nu(z)\xrightarrow[z\to 0]{} \frac{z^{-\nu}}{\Gamma(\nu+1)}\]
la norma quindi è:
\[\|U_\pm\|^2 = C_\pm^2\int_0^{+\infty}x\left|K_\nu\left(\frac{1\pm i}{\sqrt{2}}x\right)\right|^2dx\]
all'infinito non ci sono problemi perchè l'integrando tende a $e^{-\frac{1\pm i}{\sqrt{2}}x}$, in 0 l'integrando va come $ x x^{-2\sqrt{\frac{1}{4}+g} }$ che può divergere o meno a seconda di $g$. Se l'integrale divergesse saremmo costretti a imporre le costanti nulle quindi gli indici di difetto sarebbero nulli e avremo un operatore autoaggiunto, vediamo con quali condizioni su $g$ otteniamo questo. L'integrale diverge quando
\[x^2 x^{-2\sqrt{\frac{1}{4}+g}}\xrightarrow[x\to 0]{}\infty \implies 2-2\sqrt{\frac{1}{4}+g}\leq 0 \implies g\geq \frac{3}{4}\]
nel caso in cui $g<\frac{3}{4}$ possono esserci altri casi in cui l'operatore è autoaggiunto.
\end{itemize}
Questo ultimo esempio in cui abbiamo trovato delle condizioni su $g$ per l'autoaggiunzione poteva essere visto come esempio particolare di un teorema più generale che pone delle condizioni su $V(x)$ affinchè l'hamiltoniana sia autoaggiunta, vediamolo ora.
\begin{thm}(Del punto fisso o di Weyl)
Sia lo spazio di Hilbert $L_2(\R^+)$ e l'operatore di Hamilton $H = -\frac{d^2}{dx^2} + V(x)$ con $V(x)$ reale continuo e con derivata continua, allora se:
\begin{enumerate}[i.]
\item $V(x)\geq \frac{3}{4}\frac{1}{x^2}$ quando $x\to 0$ (si dice che il potenziale è nel punto limite in $x=0$)
\item $V(x)$ è limitato $|V(x)|\leq k^2$ e valgono:
\[\lim_{x\to\infty}\frac{x}{\sqrt{k^2-V(x)}} = \infty \quad \lim_{x\to \infty} \left|\frac{V'(x)}{V^{\frac{3}{2}}(x)}\right|<+\infty\]
(si dice che $V$ è nel punto limite in $x=\infty$)
\end{enumerate}
allora $H$ è autoaggiunto e $n_+=n_-=0$
\end{thm}
Vediamo come l'ultimo esempio di prima è un caso particolare di questo teorema, $V(x) = \frac{g}{x^2}$: è reale, continuo e con derivata continua in $\R^+$, inoltre $V(x)\geq \frac{3}{4}\frac{1}{x^2}$ solo quando $g\geq \frac{3}{4}$, vediamo se sono sempre soddisfatti i limiti:
\[\lim_{x\to\infty}\frac{x}{\sqrt{k^2-\frac{g}{x^2}}} = \infty \qquad  \lim_{x\to \infty} \left|\frac{2gx^{-3}}{g^{\frac{3}{2}}x^{-3} }\right| = 2\sqrt{g}\qquad \forall g \]
quindi l'unica condizione è $g\geq \frac{3}{4}$ nella quale rientriamo nelle ipotesi del teorema che ci dice che $H$ è autoaggiunto con indici di difetto nulli.\\
\subsection{Potenziali a simmetria sferica}
Studiamo adesso un altra importante hamiltoniana, quella con il potenziale a simmetria centrale $H = -\nabla^2 + V(\mathbf{x})$ e studiamola in $L_2(\R^n)$. Per farlo è comodo passare a delle coordinate sferiche generalizzate $r,\Omega^\alpha$ in modo tale che il potenziale dipenda solo da $V(r)$.\\
Facciamo allora una piccola digressione sulle coordinate sferiche generalizzate che ci sarà utile per il calcolo. Quello che ci servirà sarà la metrica euclidea che nel caso di coordinate cartesiane è semplicemente:
\[ds^2  =d\mathbf{x}^2 = dx_1^2 + dx_2^2 +\dots + dx_n^2 = \delta_{ij}dx^idx^j\]
dove si è usata la convenzione di Einstein sugli indici ripetuti, convenzione adottata per tutto il seguito della trattazione. Nel caso generale possiamo esprimere la metrica euclidea con una matrice $g_{\mu\nu}$:
\[ds^2 = g_{\mu\nu} dx^\mu dx^\nu\]
che nel caso delle coordinate cartesiane è l'identità. Vediamo invece nel caso di coordinate polari $(r,\varphi)$ come esprimere tale matrice:
\[ds^2 = g_{\mu\nu} dx^\mu dx^\nu \implies ds^2 = dr^2+r^2d\varphi^2 \implies g_{\mu\nu} = \begin{pmatrix}
  1 & 0  \\
  0 & r^2
 \end{pmatrix} \]
per prendere dimestichezza vediamo anche le coordinate sferiche $(r,\theta,\varphi)$:
\[ds^2 = dr^2+r^2d\theta^2 +r^2\sin^2\theta d\varphi^2\implies g_{\mu\nu} = \begin{pmatrix}
  1 & 0 & 0 \\
  0 &r^2 & 0\\
  0 & 0 & r^2sin^2\theta
 \end{pmatrix}\]
in generale nelle coordinate sferiche in $n$ dimensioni $(r,\Omega^\alpha)$ abbiamo che la matrice vale:
\[g_{\mu\nu} = \begin{pmatrix}
  1 & 0  \\
  0 &r^2h_{ab}\\
 \end{pmatrix} \qquad ds^2 = dr^2 + r^2h_{ab} d\Omega^ad\Omega^b \qquad a,b=2,\dots,n\]
 con evidentemente $h_{ab}$ matrice di dimensione $(n-1)\times (n-1)$ diagonale. Se identifichiamo il determinate di $g_{\mu\nu}$ e di $h_{ab}$ come:
 \[g = \text{det}(g_{\mu\nu}) \qquad h = \text{det}(h_{ab})\]
possiamo allora esprimere il laplaciano in qualsiasi sistema di coordinate:
\[\nabla^2\psi = \frac{1}{\sqrt{g}}\partial_\mu(\sqrt{g}g^{\mu\nu}\partial_\nu \psi)\qquad \mu,\nu=1,\dots,n\]
stando attendi alla posizione degli indici che in questa formula sono in alto per il tensore metrico, ma sappiamo che vale $g_{\mu\alpha}g^{\nu\alpha} = \delta_\mu^\nu$. Verifichiamo che questa formula per le coordinate cartesiane ci ritorna il classico laplaciano, sappiamo che per le coordinate cartesiane vale:
\[ g_{\mu\nu} =  g^{\mu\nu} =
 \begin{pmatrix}
  1 & 0 &  0 \\
  0 & \ddots & 0\\
  0 & 0 & 1
 \end{pmatrix}\qquad g = 1 \]
il laplaciano diventa allora quello che conosciamo:
\[\nabla^2\psi = \partial_\mu(g^{\mu\nu}\partial_\nu \psi) = \partial_\mu\partial_\mu \psi\]
Facciamo ora il conto in coordinate sferiche generalizzate dove abbiamo:
\[g_{\mu\nu} = \begin{pmatrix}
  1 & 0  \\
  0 &r^2h_{ab}\end{pmatrix}
  \qquad g^{\mu\nu} = \begin{pmatrix}
  1 & 0  \\
  0 &\frac{1}{r^2}h^{ab}\end{pmatrix} \qquad g = r^{2(n-1)}h \implies \sqrt{g} = r^{n-1}\sqrt{h}\]
mettiamo dentro alla formula del laplaciano:
\[\nabla^2\psi = \frac{1}{r^{n-1}\sqrt{h}}\partial_\mu(r^{n-1}\sqrt{h} g^{\mu\nu}\partial_\nu \psi)  =  \frac{1}{r^{n-1}\sqrt{h}}\left[\partial_r(r^{n-1}\sqrt{h}\partial_r \psi) + \partial_a\left(\frac{r^{n-1}}{r^2}\sqrt{h} h^{ab}\partial_b\psi\right)\right] \]
dove abbiamo iniziato ad espandere il tensore metrico, adesso raccogliamo $\sqrt{h}$ e svolgiamo la derivata rispetto ad $r$ del prodotto che abbiamo:
\[\nabla^2\psi = \frac{1}{r^{n-1}}\left[r^{n-1}\partial_r^2\psi +(n-1)r^{n-2}\partial_r \psi + \partial_a\left(\frac{r^{n-1}}{r^2} h^{ab}\partial_b\psi\right)\right]  = \partial_r^2\psi + \frac{n-1}{r}\partial_r\psi + \frac{1}{r^2}(\partial_a h^{ab}\partial_b\psi)\]
tenendo conto che nell'ultimo pezzo compare il laplaciano sulla superficie $S^{n-1}$ possiamo scrivere:
\[\nabla^2\psi = \partial_r^2\psi + \frac{n-1}{r}\partial_r\psi + \frac{1}{r^2}\nabla_{S^{n-1}}^2\psi\]
vediamo meglio quest'ultimo termine, per $n=2$ coordinate polari sul piano abbiamo semplicemente che:
\[\nabla_{S^{1}}^2\psi =\frac{\partial^2}{\partial \varphi^2}\]
questo operatore è autoaggiunto e come autofunzioni e autovalori ha:
\[Y_{n}(\varphi) = \frac{e^{in\varphi}}{\sqrt{2\pi}}\quad \lambda_n = - n^2\]
nel caso $n=3$ come autofunzioni abbiamo le armoniche sferiche $Y_{l,m}$ e vale:
\[-\nabla^2_{S^2}Y_{l,m} = l(l+1)Y_{l,m}\]
nel caso $n$ dimensionale abbiamo le armoniche sferiche generalizzale $Y_{\alpha}$
\[-\nabla^2_{S^{n-1}}Y_{\alpha}(\Omega^\alpha) = \lambda^2_\alpha Y_\alpha(\Omega^\alpha)\]
questo ci permette di scrivere la $\psi$ fattorizzata come:
\[\psi = \varphi(r)Y_\alpha(\Omega)\]
con la quale è più facile calcolare l'hamiltoniana, infatti:
\[H_\alpha\psi = V(r)\varphi(r)Y_\alpha - Y_\alpha\left[\varphi''(r) + \frac{n-1}{r}\varphi'(r)\right] + \frac{\varphi(r)}{r^2}\lambda_\alpha^2Y_\alpha\]
\[ = \left[-\varphi''(r) - \frac{n-1}{r}\varphi'(r) + \frac{\lambda_\alpha^2}{r^2}\varphi(r)+V(r)\varphi(r)  \right]Y_\alpha\]
con le coordinate sferiche il nostro spazio di Hilbert è diventato una cosa del tipo $L_2(\R^+)\times L_2(S^{n-1})$ la norma in tale spazio vale:
\[\|\psi\|^2 = \int dS^{n-1}\int_0^\infty dr\,r^{n-1}|\varphi(r)|^2|Y_\alpha|^2 = V(S^{n-1})\int_0^\infty dr\,r^{n-1} |\varphi|^2\]
questo perchè nell'integrale bisogna tenere conto dello Jacobiano della trasformazione delle coordinate. Il problema ora è che nella norma si trova un fattore scomodo, conviene per simmetrizzare fare un cambio di variabile e scegliere come funzioni d'onda radiali $U = r^{\frac{n-1}{2}}\varphi$ in questo modo si può vedere la norma di $U$ vale:
\[\|U\| =  V(S^{n-1})\int_0^\infty dr |U|^2\]
facciamo la sostituzione anche nell'hamiltoniana, calcoliamo per prima cosa le derivate di $\varphi$ tenendo conto che vale $\varphi(r) = U(r)r^{\frac{1-n}{2}}$:
\[\frac{d\varphi}{dr} = \frac{dU}{dr}r^{\frac{1-n}{2}} + \frac{(1-n)}{2}U(r)r^{\frac{1-n}{2}-1} = \frac{dU}{dr}r^{\frac{1-n}{2}} + \frac{(1-n)}{2}U(r)r^{\frac{-1-n}{2}}  \]
\[\frac{d^2\varphi}{dr^2} = \frac{d^2U}{dr^2}r^{\frac{1-n}{2}} +  \frac{(1-n)}{2}\frac{dU}{dr}r^{\frac{-1-n}{2}} + \frac{(1-n)}{2}\frac{dU}{dr}r^{\frac{-1-n}{2}} + \frac{(-1-n)}{2}\frac{(1-n)}{2}U(r)r^{\frac{-3-n}{2}}=\]
\[\frac{d^2U}{dr^2}r^{\frac{1-n}{2}} + (1-n)\frac{dU}{dr}r^{\frac{-1-n}{2}} + \frac{(n^2-1)}{4}U(r)r^{\frac{-3-n}{2}}\]
all'interno dell'hamiltoniana la derivata prima di $\varphi$ compare con un fattore $\frac{n-1}{r}$, calcoliamola:
\[\frac{(n-1)}{r}\frac{d\varphi}{dr} = (n-1)\frac{dU}{dr}r^{\frac{-1-n}{2}} -\frac{(n-1)^2}{2}U(r)r^{\frac{-3-n}{2}}\]
tempo di mettere dentro all'hamiltoniana:
\[H_\alpha U = \Big[-\frac{d^2U}{dr^2}r^{\frac{1-n}{2}} - (1-n)\frac{dU}{dr}r^{\frac{-1-n}{2}} - \frac{(n^2-1)}{4}U(r)r^{\frac{-3-n}{2}} - (n-1)\frac{dU}{dr}r^{\frac{-1-n}{2}} +\]\[\frac{(n-1)^2}{2}U(r)r^{\frac{-3-n}{2}} + \frac{\lambda_\alpha^2}{r^2}U(r)r^{\frac{1-n}{2}}+V(r)U(r)r^{\frac{1-n}{2}}  \Big]Y_\alpha  \]
\[ =\Big[-\frac{d^2U}{dr^2}r^{\frac{1-n}{2}} + U(r)\frac{r^{\frac{1-n}{2}}}{r^2}\left(\frac{(n-1)^2}{2}- \frac{(n^2-1)}{4}\right) + \frac{\lambda_\alpha^2}{r^2}U(r)r^{\frac{1-n}{2}}+V(r)U(r)r^{\frac{1-n}{2}}  \Big]Y_\alpha \]
con semplici conti algebrici calcoliamo:
\[\frac{(n-1)^2}{2}- \frac{(n^2-1)}{4} = \frac{n^2-4n+3}{4} = \frac{(n-1)(n-3)}{4} \]
mettiamo dentro e raccogliamo il fattore comune $r^{\frac{1-n}{2}}$ e concludiamo:
\[H_\alpha U = r^{\frac{1-n}{2}}\Big[-\frac{d^2U}{dr^2} + \frac{(n-1)(n-3)}{4} \frac{U(r)}{r^2}+ \frac{\lambda_\alpha^2}{r^2}U(r)+V(r)U(r)\Big]Y_\alpha\]
teniamo conto che sia $T$ una trasformazione unitaria tale che $U=T\varphi$ l'operatore $H$ si trasforma come\footnote{infatti sia $g=Af$ e $\tilde{g} = Tg$, $\tilde{f} = Tf$ allora $T^{-1}\tilde{g} = AT^{-1}\tilde{f} \implies \tilde{g} = TAT^{-1}\tilde{f}$ per cui evidentemente $\tilde{A} = TAT^{-1}$} $\widetilde{H} = THT^{-1}$, quindi avendo noi calcolato appena adesso $HT^{-1}\varphi$ basta applicare di nuovo $T$ che nel nostro caso corrisponde a moltiplicare per $r^{\frac{n-1}{2}}$ per ottenere la nuova Hamiltoniana che possiamo scrivere introducendo il potenziale efficace:
\[\widetilde{H_\alpha} =-\frac{d^2}{dr^2} + V_{eff}(r)\qquad V_{eff}(r) = V(r) + \frac{\lambda_\alpha^2}{r^2} + \frac{(n-1)(n-3)}{4r^2} \]
notiamo che in $n=3$ l'ultimo termine non c'è e il secondo termine è il potenziale centrifugo.\\ Studiamo adesso il caso in cui $\lambda_\alpha^2 = 0$ che corrisponde fisicamente all'assenza di momento angolare. Vediamo il caso di $V(r)=0$ ovvero l'hamiltoniana della particella libera in $\R^n$, in questo caso il potenziale efficace è:
\[V_{eff}(r) = \frac{(n-1)(n-3)}{4r^2}\]
applichiamo Weyl, otteniamo la condizione che:
\[V_{eff}\geq \frac{3}{4r^2} \implies (n-1)(n-3) \geq 3 \implies n\geq 4 \]
quindi per $n\geq4$ $H$ in $L_2(\R^n)$ è autoaggiunto, non sappiamo niente però nei casi $n=1,2,3$.\\
Un caso più interessante è quello del potenziale coulombiano $V(r) = z/r$, vediamolo nel caso $n=3$ e proviamo ad applicare Weyl:
\[V_{eff} = \frac{z}{r}\geq \frac{3}{4r^2}\]
non otteniamo nulla di fatto, non possiamo applicare Weyl purtroppo e perciò non possiamo dire nulla sull'hamiltoniana con potenziale coulombinao in tre dimensione. Per dire qualcosa ci serve un altro teorema che andremo ad enunciare, vediamone un altro prima
\begin{thm}
(Di Kato-Rellich) Sia l'hamiltoniana $H =H_0 +V$ con $V$ reale e $H_0 = -\nabla^2$ in $L_2(\R^n)$ se $\forall \psi\in D$ con $D$ dominio dell'operatore vale:
\[\|V\psi\|\leq a\|H_0\psi\|+b\|\psi\|\quad\text{con}\quad 0\leq a<1\,\, b>0\]
allora $H$ è autoaggiunto.
\end{thm}
\hspace{-1.6em}\textbf{Dimostrazione:}\\(\textbf{WARNING: Alcuni punti oscuri, maneggiare con cura!} da rivedere)\\
$H$ autoaggiunta corrisponde agli indici di difetto uguali a 0, ovvero che l'operatore:
\begin{equation}\label{ciao}(H^\dagger \pm i\lambda)f=g\end{equation}
ha un unica soluzione (notiamo che dal paragrafo 1.5 si è aggiunto $\lambda>0$, si può dimostrare che gli inidici di difetto non dipendono da tale quantità)
per ipotesi del teorema abbiamo che:
\[\|V\psi\|\leq a\|H_0\psi\|+b\|\psi\|\quad\text{con}\quad 0\leq a<1\,\, b>0\]
$\forall \psi$, scegliamo in particolare $\psi = \frac{1}{H_0\pm i \lambda} U$ possiamo scrivere allora:
\[V\psi = V\frac{1}{H_0\pm i \lambda} U = BU\]
dove abbiamo definito $B = \frac{V}{H_0\pm i \lambda}$, la norma di questo operatore vale:
\[\|V\psi\| = \|BU\| \leq a\|H_0\frac{1}{H_0\pm i \lambda}U\| + b\|\frac{1}{H_0\pm i \lambda}U\|\]
vediamo di maggiorare il primo termine passando allo spazio degli impulsi per facilitare i conti:
\[\|H_0\frac{1}{H_0\pm i \lambda}U\|^2 = \int d\mathbf{p}\, \frac{\mathbf{p}^2}{|\mathbf{p} \pm i\lambda|}|U|^2 \leq \|U\|^2\]
calcoliamo ora il secondo termine delle disuguglianza:
\[\|\frac{1}{H_0\pm i \lambda}U\|^2 = \frac{1}{\lambda^2}\|U\|^2\]
otteniamo in definitiva:
\[ \|BU\|  \leq (a+\frac{b}{\lambda})\|U\|\]
si può scegliere opportunamente un $\lambda$ tale che $a+\frac{b}{\lambda}$ sia minore di 1 e quindi:
\[ \|BU\| \leq \|U\|\]
l'operatore $B$ è quindi limitato e $\|B\|<1$. Consideriamo ora l'operatore $C = \mathbbm{1} + B$, esso è limitato e invertibile e possiamo scrivere:
\[C^{-1} = \frac{1}{\mathbbm{1}+B} = \sum_{k=0}^\infty (-1)^k B^k\]
mostriamo ora che l'unica soluzione all'equazione \eqref{ciao} è:
\[f = \frac{1}{H_0\pm i\lambda}C^{-1}g\]
infatti:
\[(H_0 \pm i\lambda +V)\frac{1}{H_0 \pm i\lambda} C^{-1}g = \left(\mathbbm{1}+\frac{V}{H_0 \pm i\lambda}\right)C^{-1}g = (\mathbbm{1} + B)C^{-1}g = CC^{-1}g=g\]
\begin{thm}(Di Kato)
Sia l'hamiltoniana $H=-\nabla^2 + V(\mathbf{x})$ in $L_2(\R^3)$ con potenziale reale e:
\[V(\mathbf{x}) = V_2(\mathbf{x}) + W(\mathbf{x})\quad\text{con}\quad V_2(\mathbf{x})\in L_2(\R^3)\]
e $|W(\mathbf{x})|\leq M$ allora $H$ è autoaggiunto.
\end{thm}
vediamo due casi in cui possiamo applicare questo teorema:
\begin{itemize}
\item Potenziale di Yukawa $V_{yuk}(r) = \frac{Ae^{-\mu r}}{r}$ con $\mu>0$\\
Consideriamo con le stesse notazione del teorema $W = 0$ e $V_2 = V_{yuk}$, controlliamo che sia in $L_2(\R^3)$:
\[\|V_{yuk}\|^2 = 4\pi\int_0^\infty dr\, r^2\frac{A^2e^{-2\mu r}}{r^2} = \frac{2\pi A^2}{\mu}\]
conludiamo che per il teorema di Kato l'hamiltoniana è autoaggiunta.
\item Potenziale di Coulumb $V_c = \frac{z}{r}$\\
sfortunatamente $V_c\notin L_2(\R^3)$ ma possiamo scriverlo come:
\[V_c(r) = \frac{ze^{-\mu r}}{r} + \frac{z}{r}(1-e^{-\mu r})\] 
dove adesso il primo termine fa parte di $L_2(\R^3)$ è sarà quindi il nostro $V_2$ mentre il secondo termine sarà la nostra $W$, dobbiamo controllare che sia limitata
\[W(r) = \frac{z}{r}(1-e^{-\mu r}) \qquad r\geq 0 \]
ma è evidente che lo sia essendoci un esponenziale decrescente. Quindi possiamo concludere per il teorema di Kato che l'hamiltonia con il potenziale di Coulomb è autoaggiunta.
\end{itemize}
\subsection{Potenziali singolari}
In questa sezione vogliamo studiare i cosiddetti potenziali singolari, ovvero rappresentati da distribuzioni come vedremo in seguito. Per farlo, come vedremo a breve, è più conveniente studiare l'hamiltoniana nello spazio degli impulsi $\mathcal{F}(L_2(\R^n)) =  \widetilde{L}_2(\R^n)$. Iniziamo dunque a ricavarci l'hamiltoniana generica $H = -\mathbf{\nabla}^2 +V(\mathbf{x})$ nello spazio degli impulsi. Le convenzioni per la trasformata di Fourier che useremo sono:
\[\widetilde{\psi}(\mathbf{p}) = \int_{\R^n}e^{-i\mathbf{p}\cdot \mathbf{x}}\psi(\mathbf{x})\,d\mathbf{x} \qquad \psi(\mathbf{x}) = \frac{1}{(2\pi)^n}\int_{\R^n}e^{i\mathbf{p}\cdot \mathbf{x}}\widetilde{\psi}(\mathbf{p})\,d\mathbf{p} \]
applichiamo $H$ ad un generico stato $\psi$ e facciamo la trasformata di Fourier
\[(\widetilde{H\psi})(\p) = \p^2 \widetilde{\psi}(\p) + \int_{\R^n} d\x\, e^{-i\p\cdot\x}V(\x)\psi(\x) = \p^2 \widetilde{\psi}(\p) + \frac{1}{(2\pi)^n}\int_{\R^n} d\x\, e^{-i\p\cdot\x}V(\x)\int_{\R^n}d\mathbf{k}\,e^{i\mathbf{k}\cdot \mathbf{x}}\widetilde{\psi}(\mathbf{k}) \]
scambiando ordine di integrazione otteniamo
\[(\widetilde{H\psi})(\p)= \p^2 \widetilde{\psi}(\p) + \frac{1}{(2\pi)^n}\int_{\R^n} d\mathbf{k}\,\widetilde{\psi}(\mathbf{k})\int_{\R^n}d\mathbf{x}\, V(\x)e^{-i\x(\mathbf{p}- \mathbf{k})}  =   \p^2 \widetilde{\psi}(\p) + \frac{1}{(2\pi)^n}\int_{\R^n} d\mathbf{k}\,\widetilde{\psi}(\mathbf{k})\widetilde{V}(\p-\mathbf{k})\]
prendiamo adesso il potenziale singolare $\widetilde{V}(\p) = g$ con $g\in\R$ ovvero $V(\x) = g\delta(\x)$ e vediamo gli operatori nei diversi spazi così ci accorgiamo del motivo della scelta di lavorare nello spazio degli impulsi
\[%\begin{matrix}
  H\psi = -\nabla^2\psi + g \delta(\x) \psi(0)\qquad bruttina\]%\\ \\
\[ \widetilde{H\psi} =  \p^2 \widetilde{\psi} + \frac{g}{(2\pi)^n}\int_{\R^n} d\mathbf{k}\,\widetilde{\psi}(\mathbf{k})\qquad buonina
 \]
 si può notare anche che l'equazione di Schr{\"o}dinger indipendente dal tempo da equazione differenziale si trasforma in equazione integrale nello spazio degli impulsi. Vediamo di studiare l'autoaggiunzione di questa hamiltoniana con potenziale singolare nello spazio degli impulsi cercando di sfruttare il teorema di Von-Neumann. L'hamiltoniana è ovviamente simmetrica e non limitata, prendiamo come dominio $D_{\widetilde{H}} = \{\widetilde{\psi},\widetilde{H}\widetilde{\psi} \in \widetilde{L}_2(\R^n) \, \int_{\R^n}d\p\,\widetilde{\psi}(\p) = 0\}$. Calcoliamo gli indici di difetto risolvendo l'equazione di Von-Neumann
 \[\p^2 \widetilde{U}_\pm + gU_\pm(0) \pm i \widetilde{U}_\pm=0 \implies \widetilde{U}_\pm = -\frac{gU_\pm(0)}{\p^2\pm i}\]
vediamone la norma per determinare gli indici di difetto, per il calcolo sfruttiamo  coordinate sferiche generalizzate
\[\|\widetilde{U}_\pm\|^2 = g^2U_\pm^2(0)\int_{\R^n}d\p\,\frac{1}{(\p^2 \pm i)^2} = \Omega_ng^2U_\pm^2(0) \int_0^{+\infty}d\rho\, \frac{\rho^{n-1}}{(\rho^2\pm i)^2} \]
dove $\Omega_n$ rappresenta in contributi angolari, è evidente che in questo integrale l'unico problema di divergenza sta all'infinito, ma a seconda di $n$ abbiamo casistiche diverse, infatti per $\rho\to \infty$ l'integrale converge quando $\rho^{n-4}$ va a 0, ovvero per $n\geq 0$ l'integrale diverge quindi dobbiamo prendere necessariamente $g=0$ per cui indici di difetto nulli, l'hamiltoniana è autoaggiunta, nel caso che $n<4$ l'integrale converge e abbiamo $n_- = n_+ = 1$ infinite estensioni autoaggiunte. Costruiamocele nel caso $n=1$, per fare meglio i conti torniamo nello spazio $L_2(\R)$, abbiamo allora che:
\[U_\pm(x) = C_\pm \int_{-\infty}^{+\infty} \frac{e^{ipx}}{p^2\pm i} = C_\pm e^{-\sqrt{\pm i }|x|}\]
dove abbiamo raccolto tutte le costanti in $C_\pm$ e abbiamo svolto l'integrale col metodo dei residui. Il dominio delle estensione autoaggiunte è quindi:
\[D_\varphi = \{\varphi : \varphi(x) = h(x) +C[e^{-\sqrt{i}|x|} + e^{i\theta}e^{-\sqrt{-i}|x|}]\}\]
vediamo meglio questa condizione al contorno in $x=0$
\[\varphi(0) =C (1+e^{i\theta}) \qquad \varphi'(x) = h'(x) +C[-\sqrt{i}e^{-\sqrt{i}|x|} - \sqrt{-i}e^{i\theta}e^{-\sqrt{-i}|x|}](\theta(x)-\theta(-x))\]
\[\varphi'(0_+) = h'(0) + C(-\sqrt{i}-\sqrt{-i}e^{i\theta}) \qquad \varphi'(0_-) = h'(0) - C(-\sqrt{i}-\sqrt{-i}e^{i\theta})\]
da cui otteniamo:
\[\frac{\varphi'(0_+) -\varphi'(0_-) }{\varphi(0)} = \frac{-\sqrt{i} - \sqrt{-i}e^{i\theta}}{1+e^{i\theta}} = \gamma\]
si lascia per esercizio dimostrare che $\gamma$ è una quantità reale. Da cui infine la condizione al contorno:
\[\psi'(0_+) - \psi'(0_-) = \gamma \psi(0) \quad \gamma\in\R\]


\section{Problemi di Scattering}
Mettiamoci per semplicità in $\R$ e affrontiamo il problema di studiare un sistema caratterizzato da un Hamiltoniana del tipo $H = -\frac{\hbar^2}{2m}\frac{d^2}{dx^2} + g\delta(x)$ con $g$ costante di accoppiamento. In prima approssimazione sostituiamo il potenziale a delta con un potenziale a supporto compatto in $[-N,N]$ come in figura
\begin{figure}[H]
\centering
\begin{tikzpicture}
\draw [<->](-5,0) -- (5,0)node[below=1.5pt] {\color{black}$x$};
\draw[->](0,0)--(0,4)node[right=1.5pt] {\color{black}$V(x)$};

\draw [red,domain=-2:2,samples = 1000] plot ({\x}, {3*2.718^(-16*(\x)^2)});

\draw[dotted](.6,0)node[below]{\footnotesize$N$};
\draw[dotted](-.6,0)node[below]{\footnotesize$-N$};

\end{tikzpicture}
\caption{Potenziale a supporto compatto}
\end{figure}
definiamo la corrente di probabilità come 
\[J(x) = \frac{\hbar}{2im}(\overline{\psi}'\psi - \overline{\psi}\psi')\]
per leggi di conservazione la corrente di probabilità deve essere costante $\frac{dJ}{dx} = 0$ quindi deve valere che $J(x<0) = J(x>0)$. L'equazione di Schr{\"o}dinger nel caso stazionario si può scrivere come $-\psi'' = \frac{2mE}{\hbar^2}\psi$, se la risolviamo per il nostro sistema per le $x$ dove il potenziale è nullo otteniamo
\[\begin{cases}
\psi(x) = A(p)e^{ipx} +B(p) e^{-ipx} \qquad x<-N \\
\psi(x) = F(p)e^{ipx} +G(p) e^{-ipx} \qquad x>N
\end{cases}\]
dove $p^2 = \frac{2mE}{\hbar^2}$. Imponiano la condizione della conservazione della corrente di probabilità, ovvero $J(x<-N) = J(x>N)$, calcoliamo quindi:
\[J(x<-N) = \frac{\hbar}{2im}\left[(-ip\overline{A}e^{-ipx}+ip\overline{B}e^{ipx})( Ae^{ipx} +Be^{-ipx}) -(ipAe^{ipx} -ipBe^{-ipx})(\overline{A}e^{-ipx} +\overline{B}e^{ipx} )  \right]\]
\[= \frac{p\hbar}{2m}\left[-|A|^2 + |B|^2 -\overline{A}Be^{-2ipx} +\overline{B}Ae^{2ipx} - |A|^2 + |B|^2 - A\overline{B}e^{2ipx} + \overline{A}Be^{-2ipx}\right] = \frac{p\hbar}{m}(-|A|^2 + |B|^2)\]
analogamente viene
\[J(x<-N) = \frac{p\hbar}{m}(-|F|^2 + |G|^2)\]
da cui quindi arriviamo alla condizione
\[J(x<-N) = J(x>N)\implies|A|^2-|B|^2 = |F|^2 - |G|^2 \implies |A|^2+|G|^2 = |F|^2 + |B|^2\]
questa condizione può essere riscritta in forma matriciale utilizzando i vettori complessi $\begin{pmatrix}
  B\\
  F
 \end{pmatrix}\in\C^2$,$\begin{pmatrix}
  A\\
  G
 \end{pmatrix}\in\C^2$:
\[ \begin{pmatrix}
  B\\
  F
 \end{pmatrix} = S \begin{pmatrix}
  A\\
  G
 \end{pmatrix} \]
 con $S$ \textbf{matrice di scattering} $2\times2$. Si può dimostrare che $S$ è una matrice unitaria, ovvero $S^\dagger S = \mathbbm{1}$, che equivale alla seguente condizione:
 \[S^\dagger S = \begin{pmatrix}
  \overline{S_{11}} & \overline{S_{21}} \\
   \overline{S_{12}} & \overline{S_{22}} 
 \end{pmatrix}\begin{pmatrix}
  S_{11} & S_{12}\\
   S_{21} & S_{22} 
 \end{pmatrix} = \begin{pmatrix}
 1 & 0\\
 0 & 1\end{pmatrix} \implies |S_{11}|^2 + |S_{21}|^2 = 1\]
 Mettiamoci ora nel caso in cui l'onda arriva da sinstra e dopo il pontenziale va a destra, quindi $G = 0$, semplifichiamo ulteriormente il problema e poniamo $F=1$. A questo punto possiamo definire le seguenti quantità:
 \[T = \frac{|F|^2}{|A|^2} = \frac{1}{|A|^2} \qquad \textbf{coefficiente di trasmissione}\]
 \[R = \frac{|B|^2}{|A|^2}  \qquad \textbf{coefficiente di riflessione}\]
 che soddisfano la condizione $T+R = 1$, in questo caso è banalmente soddisfatta dato che	
 \[ \frac{1}{|A|^2} + \frac{|B|^2}{|A|^2} = \frac{|B|^2 + 1}{|A|^2} = \frac{|A|^2}{|A|^2} = 1\]
 dove abbiamo tenuto conto che vale $|A|^2+|G|^2 = |F|^2 + |B|^2$ che in questo caso è $|A|^2 = 1 + |B|^2$.\\
 Per ricavarci i valori di $A$ e $B$ dobbiamo studiare le condizioni al contorno della funzione d'onda, per farlo consideriamo il pontenziale deltiforme $V(x) = g\delta(x)$, nel paragrafo precedente abbiamo già studiato il caso di questo potenziale, imponiano quindi la condizione ottenuta $\psi'(0_+) - \psi'(0_-) = \frac{2mg}{\hbar^2}\psi(0)$ ed il fatto che $\psi$ deve essere continua in 0. La continuità ci da che $A+B=1$, mentre la condizione sulla derivata:
 \[\begin{cases}ip-ip(A-B) = \frac{2mg}{\hbar^2}(A+B) \\ A+B=1\end{cases} \implies A = \frac{ip\hbar^2 -mg}{ip\hbar^2} \quad B = \frac{mgA}{ip\hbar^2-mg} = \frac{mg}{ip\hbar^2} \]
 da cui è possibile anche calcolare i coefficienti di trasmissione e riflessione
 \[R = \frac{m^2g^2}{p^2\hbar+m^2g^2} \qquad T = \frac{p^2\hbar^4}{p^2\hbar^4+m^2g^2}\]
 la matrice di scattering ha come componenti $S_{11}= B/A$ e $S_{21} = 1/A$, i poli di questa matrice sono gli zeri di $A$ ovvero $ip\hbar^2= mg \implies p =\frac{mg}{\hbar^2i}$ che implica che $p^2$ è un numero negativo, ma $p^2$ è legato all'energia quindi questo caso potrebbe essere uno stato legato, nel caso in cui $g$ sia minore di 0 siamo nel caso di uno stato legato, invece nel caso $g>0$ abbiamo risonanza.\\
 I valori di $A$ e $B$ potevano essere calcolati in un modo diverso. Se formuliamo l'equazione di Schr{\"o}dinger in forma integrale e la risolviamo troviamo esattamente la stessa soluzione di prima.\\
 Iniziamo a trovare la formulazione integrale, nel caso generale in 3 dimensioni l'equazione di Schr{\"o}dinger può essere scritta come:
 \[\left(-\nabla^2 -\frac{2mE}{\hbar^2}\right)\psi(\x) = -\frac{2m}{\hbar^2}V(\x)\psi(\x)\] 
 se consideriamo il membro di destra come termine non omogeneo ci riconduciamo al caso di un equazione lineare $Lu = g$ che ha come soluzione $u = u_0 + g*\mathcal{E}$ dove $u_0$ è la soluzione dell'omogenea e $\mathcal{E}$ è la soluzione fondamentale dell'operatore $L$. Nel nostro caso quindi la soluzione sarà:
 \[\psi(\x) = \psi_0(\x) -\frac{2m}{\hbar^2}\int_{\R^3}\mathcal{E}(\x-\mathbf{y})V(\mathbf{y})\psi(\mathbf{y})\,d\mathbf{y} \]
 L'operatore in questo caso è del tipo $-\nabla^2 - \omega^2$ che è l'operatore di Helmholtz la cui soluzione fondamentale si può trovare con il metodo di Hormander, passando quindi nello spazio degli impulsi tramite trasformata di Fourier. La soluzione che si trova è:
 \[\mathcal{E}(\x) = \frac{e^{\pm i\omega|\x|}}{4\pi|\x|}\]
 che messa dentro alla formula per $\psi(\x)$, assieme alla soluzione dell'omogenea, ci da l'equazione integrale equivalente a quella di Schr{\"o}dinger:
 \[\psi(\x) = e^{\pm i \mathbf{p}\cdot \x} -\frac{2m}{4\pi\hbar^2}\int_{\R^3} \frac{e^{\pm i\omega|\x-\mathbf{y}|}}{|\x-\mathbf{y}|}V(\mathbf{y})\psi(\mathbf{y})\,d\mathbf{y} \]
 detta anche \textbf{equazione di Lippmann-Schwinger}.\\
 Ritorniamo nel nostro caso con potenziale unidimensionale $V(x) = g\delta(x)$, in questo caso l'operatore è $L = \frac{d^2}{dx^2} + p^2$ che ha come soluzione fondamentale anticipata $G_A(x) = -\theta(-x)Z(x)$ con $Z(x)$ soluzione dell'equazione omogenea con condizioni al contorno $Z(0) = 0$, $Z'(0)=1$. È facile vedere che $G_A(x) = -\theta(-x)\frac{\sin(px)}{p}$ da cui l'equazione di Lippmann-Schwinger del nostro sistema:
 \[\psi(x) = e^{ipx} -\frac{2m}{\hbar^2}\int_{\R} \theta(y-x)\frac{\sin(p(x-y))}{p} g\delta(y)\psi(y)\,dy = e^{ipx} -\frac{2mg}{\hbar^2}\theta(-x)\frac{\sin(px)}{p} \psi(0)\]
 notiamo che la condizione al contorno $\psi'(0_+) - \psi'(0_-) = \frac{2mg}{\hbar^2}\psi(0)$ è già soddisfatta dalla $\psi(x)$, infatti:
 \[\psi'(x) = ip e^{ipx} -\frac{2mg}{\hbar^2}\theta(-x)\cos(px)\psi(0)\]
 da cui:
 \[\psi'(0_+) - \psi'(0_-) = ip -ip +\frac{2mg}{\hbar^2}\psi(0) =  \frac{2mg}{\hbar^2}\psi(0)\]
inoltre per $x>0$ la funzione $\psi(x)$ possiamo scriverla come
\[\psi(x) = e^{ipx} \implies F=1\quad G=0\]
mentre per $x<0$ la funzione possiamo scriverla come (tenendo conto che $\psi(0) = 1$ per continuità della funziona d'onda):
\[\psi(x) = e^{ipx} -\frac{2mg}{\hbar^2}\left[\frac{e^{ipx} - e^{-ipx}}{2ip}\right] =e^{ipx} \left(1-\frac{mg}{ip\hbar^2}\right) + \frac{mg}{ip\hbar^2} e^{-ipx}   \]
che sono gli stessi $A$ e $B$ che abbiamo trovato prima.\\
Concentriamoci ora sugli stati legati del sistema, in precedenza li abbiamo indentificati come le singolarità della matrice di scattering un approccio alternativo al problema è quello di studiare la risolvente del sistema i cui poli ci danno gli stati legati. Questo approccio è preferibili in alcuni casi, ad esempio quello in cui nel potenziale compaiono due delte di Dirac che andremo a vedere in dettaglio a breve. Iniziamo però da alcuni fatti di carattere generale e consideriamo un Hamiltoniana $H =H_0 + V$ in $L_2(\R)$, la risolvente è l'operatore $G_z$ tale che 
\begin{equation}\label{risolvente} (H-z)G_z = \mathbbm{1}\end{equation}
se indichiamo con $G_z^0$ la risolvente dell'operatore $H_0$ allora vale l'identità del risolvente:
\[G_z = G_z^0 - G_z^0VG_z\]
è facile dimostrarlo, infatti mostriamo che vale la \eqref{risolvente} mettendoci dentro tale l'identità:
\[(H-z)G_z = (H_0-z+V)G_z = (H_0-z)G_z + VG_z = (H_0 -z)(G_z^0-G_z^0VG_z) + VG_z = \]
\[\mathbbm{1} - \mathbbm{1}VG_z + VG_z = \mathbbm{1}\]
per determinare la risolvente di un sistema, ovvero risolvere la \eqref{risolvente}, è più comodo passare dal nucleo dell'operatore, i termini di nuclei l'equazione per la risolvente diventa:
\[(H-z)G_z(x,y) = \delta(x-y)\]
che possiamo riscrivere come:
\[(H_0-z)G_z(x,y) = \delta(x-y) - V(x)G_z(x,y)\]
se consideriamo il termine a destra come termine noto abbiamo che la soluzione dell'equazione è la convoluzione tra la soluzione fondamentale e il termine noto, ma notiamo che la soluzione fondamentale di $H_0-z$ è proprio $G_z^0(x,y)$, quindi possiamo scrivere che:
\[G_z(x,y) = G_z^0(x,y)*[\delta(x-y) - V(x)G_z(x,y)] \]
tenendo conto che la convoluzione gode della proprietà distributiva e che la delta è l'elemento identità per tale operazione possiamo dire che
\begin{equation}\label{ris_nucleo}G_z(x,y) = G_z^0(x,y) - \int dy'\, G_z^0(x,y')V(y')G_z(y',y)\end{equation}
Applichiamo ora questi risultati al nostro sistema che ha $V(y') = g\delta(y')$, è facile allora trovare $G_z(x,y)$ con la formula appena trovata:
\begin{equation}\label{ris_sistema}G_z(x,y) = G_z^0(x,y) -gG_z^0(x,0)G_z(0,y)\end{equation}
si tratta adesso di trovare $G_z^0(x,y)$, per semplificare il problema cerchiamo $G_z^0(x)$ e poi reintroduciamo la variabile $y$ tenendo conto dell'invarianza per traslazione del problema
\[(H_0 -z)G_z^0(x) = \delta(x) \implies \left(-\frac{\hbar^2}{2m}\frac{d^2}{dx^2}-z\right)G_z^0(x) = \delta(x)\]
utilizziamo Fourier per risolvere l'equazione differenziale, passiamo quindi alla variabile $k = \frac{p}{\hbar}$
\[\left(\frac{\hbar^2}{2m}k^2 -z\right)\widetilde{G_z^0}(x) = 1 \implies \widetilde{G_z^0}(x) = \frac{1}{\frac{\hbar^2}{2m}k^2-z} = \frac{2m}{\hbar^2}\left(\frac{1}{k^2+a^2}\right) \] 
dove $a^2 = -\frac{2mz}{\hbar^2}$. Per tornare nello spazio delle configurazioni facciamo l'antitrasformata di Fourier
\[G_z^0(x) = \frac{m}{\pi\hbar^2}\int_{-\infty}^{+\infty}dk\, \frac{e^{ikx}}{k^2+a^2} = \frac{m}{\hbar^2a}e^{-a|x|}\]
dove l'integrale è stato fatto con il metodo dei residui, reintroduciamo adesso $y$ nell'unica maniera possibile per avere l'invarianza per traslazione
\[G_z^0(x,y)  = \frac{m}{\hbar^2a}e^{-a|x-y|}\]
che possiamo ora utilizzare nella \eqref{ris_sistema}, dove ci manca però $G_z(0,y)$ che però possiamo trovare valutando l'espressione \eqref{ris_sistema} in $x=0$:
\[G_z(0,y) = G_z^0(0,y) - gG_z^0(0,0)G_z(0,y) \implies G_z(0,y) = \frac{G_z^0(0,y)}{1+ gG_z^0(0,0)} = \frac{G_z^0(0,y)}{1+ \frac{gm}{\hbar^2a}}\]
con la quale possiamo finalmente arrivare alla soluzione per $G_z(x,y)$:
\[G_z(x,y) = G_z^0(x,y) - \frac{gG_z^0(x,0)G_z^0(0,y)}{1+ \frac{gm}{\hbar^2a}}\]
un polo di questo risolvente è $a = -\frac{gm}{\hbar^2}$ che si traduce nello stesso stato legato trovato in precedenza\footnote{Non è così banale, lo verificherò}.\\ 
Questo approccio per risolvere il problema degli stati legati è estremamente più comodo quando abbiamo a che fare con Hamiltoniane del tipo $H = -\frac{\hbar^2}{2m}\frac{d^2}{dx^2} + \sum_{k=1}^N g_k\delta(x-x_k)$, vediamo il caso $N=2$. Consideriamo quindi un sistema caratterizzato da un Hamiltoniana $H = -\frac{\hbar^2}{2m}\frac{d^2}{dx^2} +g_1\delta(x-x_1)+g_2\delta(x-x_2)$ con $x_1>0$, $x_2>0$ e $g_{1,2}\in\R$. Se utilizzassimo un approccio classico dovremmo determinare la funzione d'onda che descrive il sistema con le seguenti condizioni al contorno:
\[\begin{cases}
\psi'(x_{1+}) - \psi'(x_{1-}) = \frac{2mg_1}{\hbar^2}\psi(x_1)\\
\psi'(x_{2+}) - \psi'(x_{2-}) = \frac{2mg_2}{\hbar^2}\psi(x_2)
\end{cases}\]
assieme naturalmente alla condizioni di continuità di $\psi$. La mole di calcoli è relativamente minore se andiamo a studiare invece i poli della risolvente, sfruttando la \eqref{ris_nucleo} possiamo dire che:
\[G_z(x,y) = G_z^0(x,y) - g_1G_z^0(x,x_1)G_z(x_1,y) - g_2G_z^0(x,x_2)G_z(x_2,y) \]
dove la $G_z^0(x,y)$ l'abbiamo già calcolata, ci mancano quindi $G_z(x_1,y)$ e $G_z(x_2,y)$. Per ricavarle valutiamo questa espressione in $x_1$ e in $x_2$, arriviamo quindi al seguente sistema:
\[\begin{cases}
G_z(x_1,y) = G_z^0(x_1,y) - g_1G_z^0(x_1,x_1)G_z(x_1,y) - g_2G_z^0(x_1,x_2)G_z(x_2,y)\\
G_z(x_2,y) = G_z^0(x_2,y) - g_1G_z^0(x_2,x_1)G_z(x_1,y) - g_2G_z^0(x_2,x_2)G_z(x_2,y)
\end{cases}\]
notiamo che per la forma della $G_z^0(x,y)$ vale che $G_z^0(x_1,x_2) = G_z^0(x_2,x_1) =: B $ ed inoltre $G_z^0(x_1,x_1)=G_z^0(x_2,x_2)=:A$. Possiamo riscrivere allora il sistema nel seguente modo
\[\begin{cases}
G_z(x_1,y) = G_z^0(x_1,y) - g_1AG_z(x_1,y) - g_2BG_z(x_2,y)\\
G_z(x_2,y) = G_z^0(x_2,y) - g_1BG_z(x_1,y) - g_2AG_z(x_2,y)
\end{cases}\]
che in forma matriciale diventa:
\[\begin{pmatrix}
1+g_1A & g_2B\\
g_1B & 1+g_2A
\end{pmatrix}
\begin{pmatrix}
G_z(x_1,y)\\
G_z(x_2,y)
\end{pmatrix}=
\begin{pmatrix}
G_z^0(x_1,y)\\
G_z^0(x_2,y)
\end{pmatrix}
\]
se identifichiamo con $M$ la matrice più a sinistra la soluzione a problema è:
\[\begin{pmatrix}
G_z(x_1,y)\\
G_z(x_2,y)
\end{pmatrix}= M^{-1}\begin{pmatrix}
G_z^0(x_1,y)\\
G_z^0(x_2,y)
\end{pmatrix}\]
per scrivere la soluzione ci avvaliamo del metodo di cramer con il quale possiamo raccogliere il fattore $\frac{1}{\text{det}M}$ delle soluzioni e scrivere che:
\[G_z(x,y) = G_z^0(x,y) +\frac{1}{\text{det}M}(\dots \text{roba} \dots)\]
dove abbiamo tralasciato i termini in parentesi in quanto siamo interessati agli stati legati ovvero ai poli del risolvente che in questo caso corrispondo agli zero del determinante della matrice M
\[\text{det}M = 0 \implies (1+g_1A)(1+g_2A) - g_1g_2B^2 = 0\]
tenendo conto della definzione di $A$ e $B$ arriviamo all'equazione:
\[\left(1+\frac{g_1m}{a\hbar^2}\right)\left(1+\frac{g_2m}{a\hbar^2}\right) - \frac{g_1g_2}{\hbar^4a^2}e^{-2ad} = 0\]
dove abbiamo definito $d=|x_1-x_2|$. Si tratta di un equazione trascendentale che può essere risolta ``analiticamente'' sfruttando la funzione $W$ di Lambert, limitiamoci ad una trattazione più elementare di tipo geometrico e studiamo graficamente l'equazione che possiamo riscrivere come:
\[a^2 + \frac{m(g_1+g_2)}{\hbar^2}a= \frac{m^2g_1g_2}{\hbar^4}(e^{-2ad}-1)\]
a sinistra abbiamo una parabola nella variabile $a$ mentre a destra un esponenziale. La parabola ha sempre concavità verso l'alto, inoltre passa sempre per l'origine, l'esponenziale passa anch'esso per l'origine. Consideriamo che $a$ deve essere strettamente positivo in quanto è legato all'energia $a = \frac{\sqrt{2m}\sqrt{|E|}}{\hbar}$. L'altro zero della parabola è in $-\frac{m}{\hbar^2}(g_1+g_2)$ se $g_1+g_2>0$ allora lo zero sarà nelle $a$ negative e siamo nel caso della seguenti figure
\begin{figure}[H]
\centering
\begin{minipage}{.5\textwidth}
\centering
\begin{tikzpicture}
\draw [->](0,0) -- (5,0)node[below=1.5pt] {\color{black}$a$};
\draw[->](0,-1.5)--(0,2)node[right=1.5pt] {\color{black}$y$};
\draw[dotted](0,-1)node[left]{\scriptsize$-1$}--(5,-1);
\draw [red,domain=0:5,samples = 1000] plot ({\x}, {2.718^(-\x)-1});
\draw [blue,domain=0:4,samples = 100] plot ({\x}, {0.1*\x^2+0.1*\x});
\end{tikzpicture}
\caption{$g_1+g_2>0$ e $g_1g_2>0$}
\end{minipage}%
\begin{minipage}{.5\textwidth}
\centering
\begin{tikzpicture}
\draw [->](0,0) -- (5,0)node[below=1.5pt] {\color{black}$a$};
\draw[->](0,-1.5)--(0,2)node[right=1.5pt] {\color{black}$y$};
\draw[dotted](0,1)node[left]{\scriptsize$1$}--(5,1);
\draw [red,domain=0:5,samples = 1000] plot ({\x}, {-(2.718^(-\x)-1)});
\draw [blue,domain=0:4,samples = 100] plot ({\x}, {0.1*\x^2+0.1*\x});
\end{tikzpicture}
\caption{$g_1+g_2>0$ e $g_1g_2<0$}
\end{minipage}
\end{figure}
a seconda del segno di $g_1g_2$ avremo o no degli stati legati. Nel caso in cui $g_1+g_2<0$ siamo nelle seguente situazione
\begin{figure}[H]
\centering
\begin{minipage}{.5\textwidth}
\centering
\begin{tikzpicture}
\draw [->](0,0) -- (5,0)node[below=1.5pt] {\color{black}$a$};
\draw[->](0,-1.5)--(0,2)node[right=1.5pt] {\color{black}$y$};
\draw[dotted](0,-1)node[left]{\scriptsize$-1$}--(5,-1);
\draw [red,domain=0:5,samples = 1000] plot ({\x}, {2.718^(-\x)-1});
\draw [blue,domain=0:3.5,samples = 100] plot ({\x}, {\x^2-3*\x});
\end{tikzpicture}
\caption{$g_1+g_2<0$ e $g_1g_2>0$}
\end{minipage}%
\begin{minipage}{.5\textwidth}
\centering
\begin{tikzpicture}
\draw [->](0,0) -- (5,0)node[below=1.5pt] {\color{black}$a$};
\draw[->](0,-1.5)--(0,2)node[right=1.5pt] {\color{black}$y$};
\draw[dotted](0,1)node[left]{\scriptsize$1$}--(5,1);
\draw [red,domain=0:5,samples = 1000] plot ({\x}, {-(2.718^(-\x)-1)});
\draw [blue,domain=0:3.5,samples = 100] plot ({\x}, {\x^2-3*\x});
\end{tikzpicture}
\caption{$g_1+g_2<0$ e $g_1g_2<0$}
\end{minipage}
\end{figure}
dove nel caso $g_1g_2<0$ avremo sempre degli stati legati, mentre nel caso $g_1g_2>0$ avremo stati legati solo quando la parabola in 0 avrà derivata maggiore in modulo di quella dell'esponenziale, ovvero:
\[\frac{m|g_1+g_2|}{\hbar^2} > \frac{2m^2g_1g_2d}{\hbar^4}\]
che equivale alla condizione:
\[|g_1+g_2|> \frac{2mg_1g_2d}{\hbar^2}\]
\section{Momento angolare}
Nel paragrafo 2.3.2 abbiamo studiato l'hamiltoniana con simmetria centrale sfruttando delle coordinate sferiche, il laplaciano ci dava un termine $-\nabla_{S^{n-1}}^2$ che abbiamo trascurato di studiare, in realtà in fisica questo operatore è di grande importanza in quanto lo possiamo identificare fisicamente come il momento angolare. Richiamiamo alcuni risultati che avevamo ottenuto, l'hamiltoniana in coordinate sferiche valeva
\[H\psi = -\frac{\hbar^2}{2m}\left[\partial_r^2\psi + \frac{n-1}{r}\partial_r\psi + \frac{1}{r^2}\nabla_{S^{n-1}}^2\psi\right] +V(r)\psi\]
dove avevamo che:
\[\nabla_{S^{n-1}}^2 = \frac{1}{\sqrt{h}}\partial_\mu(\sqrt{h}h^{\mu\nu}\partial_\nu) \qquad \mu,\nu=1,2,\dots, n\]
fattorizzando la $\psi$ in modo che $\psi = \varphi(r)Y_\alpha(\Omega^\alpha)$ e tenendo conto che vale:
\[-\nabla_{S^{n-1}}^2 Y_\alpha=\lambda_\alpha^2Y_\alpha \]
possiamo scrivere l'hamiltoniana radiale
\[H_\alpha \varphi = \frac{\hbar^2}{2m}\left(-\varphi'' - \frac{n-1}{r}\varphi' + \frac{\lambda_\alpha^2}{r^2}\varphi\right) + V(r)\varphi(r)\]
notiamo che con questa fattorizzazione della funzione d'onda siamo passati ad uno spazio di Hilbert del tipo $L(\R^n) = L_2(\R^+,r^{n-1}dr)\times L_2(S_{n-1})$ dove abbiamo specificato la misura di Lebesque dello spazio.\\
Prima di arrivare al momento angolare soffermiamoci un attimo ancora sulla parte radiale e chiediamoci quale potrebbe essere l'operatore quantità di moto radiale $P_r$, è ovvio che deve soddisfare la regola di commutazione $[r,P_r] = i\hbar$, per le coordinate cartesiane è facile e l'abbiamo già visto che $[x_j,P_j] = i\hbar\delta_{ij} \implies P_j = -i\hbar\frac{\partial}{\partial x_j}$. La richiesta minimale su $P_r$ è che sia simmetrico, potrebbe venire naturale dire che vale $P_r = -i\hbar\partial_r$, ma è sbagliato per 2 ragioni principalmente. Mettiamoci per semplicità in $n=3$, si può dimostrare che $P_r$ così definito non è simmetrico (ricordiamo che nello spazio in cui stiamo lavorando la misura di Lebesque è $r^2dr$), inoltre noi vorremmo che $P_r^2 = -\partial_r^2 - \frac{2\partial_r}{r}$ in modo che messo nell'Hamiltoniana radiale ritorniamo all'Hamiltoniana classica. Il corretto operatore è $P_r = -i\hbar(\partial_r+\frac{1}{r})$, infatti con questa definizione $P_r^2$ è quello che vogliamo noi ed inoltre questo operatore è simmetrico:
\[(P_rf,g) - (f,P_rg) = i\int_{0}^{+\infty}dr\,\frac{d}{dr}(r^2\overline{f}g) = i[r^2\overline{f}g\Big|_0^{+\infty}\]
basta scegliere opportune condizioni al contorno per il dominio di $P_r$:
\[D_{P_r} = \{f,P_rf\in L_2(\R^+,r^{2}dr), \lim_{r\to0}rf=0\}\]
calcolando gli indici di difetto di questo operatore otteniamo $n_+=0$ e $n_-=1$ quindi per il teorema di Von-Neumann non esistono estensioni autoaggiunte, $P_r$ quindi è un operatore simmetrico ma non autoaggiunto. Si può dimostrare invece che $P_r^2$ è autoaggiunto.\\
Notiamo che nell'Hamiltoniana compare un termine con la derivata prima, inoltre la misura di Lebesque dell'insieme in cui stiamo lavorando non è banale, proprio per questi motivi è conveniente fare una trasformazione unitaria, così come l'avevamo fatto nel paragrafo 2.3.2 $u = r\varphi(r)$, con questa trasformazione arriviamo all'Hamiltoniana:
\[\widetilde{H_\alpha}u = \frac{\hbar^2}{2m}\left(-\frac{d^2}{dr^2} + \frac{\lambda_\alpha^2}{r^2}\right)u + V(r)u\]
anche l'operatore $P_r^2$ si trasforma tenendo conto per una trasformazione unitaria $U$ vale che $\widetilde{P_r} = UP_rU^{-1}$, applicando questa si arriva ad:
\[\widetilde{P_r} = -i\hbar\partial_r \qquad  \widetilde{P_r^2} = -\frac{\hbar^2}{2m}\frac{d^2}{dr^2} \] 
Concentriamoci adesso sul momento angolare, ovvero sulla parte con il termine $\nabla_{S^{n-1}}^2$, vediamo velocemente il caso $n=2$ ovvero il laplaciano sul cerchio $\nabla_{S^1}^2$ abbiamo che applicando la formula del laplaciano:
\[L^2 = -\hbar^2\nabla_{S^1}^2 = -\hbar^2\frac{d^2}{d\varphi^2}\]
e risolvendo l'equazione agli autovalori 
\[-\hbar^2 \frac{d^2}{d\varphi^2} Y_n(\varphi) = 4\pi^2n^2Y_n(\varphi) \qquad Y_n(\varphi) = \frac{e^{in\varphi}}{\sqrt{2\pi}} \quad n\in\mathbb{Z}\]
studiamo il caso $n=3$, in questo caso la matrice $h$ è:
\[h_{\mu\nu} = \begin{pmatrix}
1 & 0\\
0 & \sin^2\theta
\end{pmatrix} \qquad h^{\mu\nu} = \begin{pmatrix}
1 & 0\\
0 & \frac{1}{\sin^2\theta}
\end{pmatrix}\qquad \sqrt{h} = \sin\theta\]
si arriva allora a:
\[-\nabla_{S^2}^2 = -\frac{\partial^2}{\partial\theta^2} - \text{cotang}\theta\frac{\partial}{\partial\theta} - \frac{1}{\sin^2\theta}\frac{\partial^2}{\partial\varphi^2}\]
in questo caso abbiamo utilizzando come coordinate $\theta,\varphi$, è possibile utilizzare invece come coordinata al posto di $\theta$, $u = \cos\theta$ con $u\in[-1,1]$ in questo caso $\frac{du}{\sqrt{1-u^2}} = \sin\theta d\theta$, la metrica diventa $dh^2 =\frac{du^2}{1-u^2} + (1-u^2)d\varphi^2$ con il quale la matrice $h$:
\[h_{\mu\nu} = \begin{pmatrix}
\frac{1}{1-u^2}& 0\\
0 & 1-u^2\theta
\end{pmatrix} \qquad h^{\mu\nu} = \begin{pmatrix}
1-u^2 & 0\\
0 & \frac{1}{1-u^2}
\end{pmatrix}\qquad \sqrt{h} = 1\]
che ci portano ad un laplaciano sulla sfera del tipo:
\[-\nabla_{S^2}^2Y = -\partial_u((1-u^2)\partial_uY) - \frac{1}{1-u^2}\partial_\varphi^2Y\]
un operatore strettamente collegato a quello del momento angolare è il momento angolare proiettato lungo $z$ che vale $L_z = -i\hbar\frac{\partial}{\partial\varphi}$, si può mostrare che questi due operatore commutano $[L^2,L_z] = 0$, ma noi sappiamo che quando due operatori commutano allora hanno un S.O.N.C. in comune, che possiamo trovare risolvendo il seguente problema agli autovalori:
\begin{equation}\label{probautovalori}\begin{cases}
L^2Y = \hbar^2\lambda^2Y\\
L_zY = \hbar m Y
\end{cases}\end{equation}
per semplicità cerchiamo una soluzione fattorizzata nel seguente modo: $Y(u,\varphi) = p(u)e^{im\varphi}/\sqrt{2\pi}$ con $m\in\mathbb{Z}$, è immediato verificare che questa funzione soddisfa l'equazione agli autovalori per $L_z$, utilizziamo quindi l'equazione di $L^2$ per determinare $p(u)$, infatti la parte angolare si cancella:
\[L^2Y = -\hbar^2\nabla_{S^2}^2Y = -\hbar^2\partial_u((1-u^2)\partial_up(u))\frac{e^{im\varphi}}{\sqrt{2\pi}} + \frac{p(u)m^2\hbar^2}{1-u^2}\frac{e^{im\varphi}}{\sqrt{2\pi}} = \hbar^2\lambda^2p(u)\frac{e^{im\varphi}}{\sqrt{2\pi}}\]
che semplificando e svolgendo la derivata in $u$ diventa:
\[(1-u^2)\frac{d^2p(u)}{du^2} - 2u\frac{dp(u)}{du}+\left(\lambda^2 -\frac{m^2}{1-u^2}\right)p(u)=0 \qquad m\in\mathbb{Z}\]
questa si chiama \textbf{equazione generalizzata di Legendre}, l'equazione di Legendre si ottiene invece con $m=0$:
\[(1-u^2)\frac{d^2p(u)}{du^2} - 2u\frac{dp(u)}{du}+\lambda^2p(u)=0\]
la soluzione in $L_2(-1,1)$ regolare di quest'ultima equazione è:
\[p_l(u) = C_l\frac{d^2}{du^2}\left((u^2-1)^l\right)\]
con $C_l$ costante e autovalri che soddisfano $\lambda^2 = l(l+1)$, è facile verificarlo mettendo dentro questa soluzione nell'equazione e vedendo che è soddisfatto solo per tali valori di $\lambda$. Le soluzioni di quella generalizzata, quindi con $m\neq0$ sono le funzioni generalizzate di Legendre:
\[p_{l,m}(u) = K_{l,m}\frac{d^{l-m}}{du^{l-m}}\left((u^2-1)^l\right)\]
con $K_{l,m}$ costante e stessi autovalori di prima, ma con $|m|\leq l$. Le funzioni armoniche sferiche sono le $Y_{l,m} = p_{l,m}(u)\frac{e^{im\varphi}}{\sqrt{2\pi}} $, per determinare le costanti basta imporre che $Y$ sia un S.O.N.C, ovvero che valga:
\[\int_0^{2\pi}d\varphi\int_{-1}^1du\, \overline{Y_{l,m}}Y_{l',m'} = \delta_{ll'}\delta_{mm'}\]
Un approccio alternativo allo studio del momento angolare quantistico può essere affrontato prendendo il momento angolare classico e quantizzandolo sfruttando l'operatore quantità di moto quantistico:
\[\mathbf{L} = \mathbf{r}\times \mathbf{p} \implies \mathbf{L} = -i\hbar\mathbf{r}\times \nabla \]
che espresso in componenti è:
\[L_i = -i\hbar \varepsilon_{ijk}x_k\partial_j\]
dove $\varepsilon_{ijk}$ è il tensore di Levi-Civita. Le componenti del momento angolare soddisfano la seguente regola di commutazione:
\[[L_i,L_j] = i\hbar \varepsilon_{ijk}L_k\]
il quadrato del momento angolare sarà ovviamente $L^2 = L_x^2+L_y^2 + L_z^2$ e si può verificare che commuta con tutte le componenti del momento angolare: $[L^2,L_i]=0$, quindi non solo con $L_z$ come avevamo detto in precedenza, questo implica che $L^2$ ed $L_i$ condividono lo stesso sistema di autovettori.\\
Altri operatori legati al momento angolare sono i \textbf{ladder operator} definiti come:
\[L_+ = Lx + iL_y \qquad L_- = Lx-iL_y\]
questi operatori non sono autoaggiunti $L_+\neq L_+^\dagger$, $L_-\neq L_-^\dagger$, ma vale che $L_+^\dagger = L_-$ e $L_-^\dagger = L_+$. Possiamo esprimere le componenti $L_x$ ed $L_y$ del momento angolare in funzione di questi operatori:
\[L_x = \frac{1}{2}(L_+ +L_+) \qquad L_y = \frac{1}{2}(L_+-L_-)\] 
questi operatori sono utili per il problema agli autovalori di $L_z$, intanto la regola di commutazione che seguono questi operatori è: $[L_+,L_z] = -\hbar L_+$, inoltre vale che $L_zL_+ = \hbar L_+ +  L_+-z$ ed il problema agli autovalori diventa 
\[\begin{cases}
L_zL_\pm f_{l,m} = \hbar(m+1)L_+ f_{l,m\pm1}\\
L_\pm f_{l,m} = \hbar C^{\pm}_{l,m}f_{l,m\pm1}
\end{cases}
\]
possiamo quindi interpretare $L_+f_{l,m}$ come autostato di $L_z$. Altre formule utili che possiamo ricavare direttamente dalle definizioni sono:
\[L_x^2 = \frac{(L_++L_-)^2}{4} = \frac{L_+^2 + L_-^2 + L_+L_- + L_-L_+}{4}\quad L_y^2 = -\frac{(L_+-L_-)^2}{4} = \frac{L_+^2 + L_-^2 - L_+L_- - L_-L_+}{4}\]
\[L^2-L_z^2 = L_x^2 + L_y^2 = \frac{L_+L_- + L_-L_+}{2}\]
\[L_+L_- = (L_x+iL_y)(L_x-iL_y) = L_x^2 + iL_yL_x -iL_xL_y + L_y^2 = L_x^2+L_y^2 -i[L_x,L_y] = L^2-L_z^2 +\hbar L_z\]
e nello stesso modo $L_-L_+ = L^2-L_z^2-\hbar L_z$.
\\Poniamoci ora il problema di trovare le costanti $C^{\pm}_{l,m}$ del problema agli autovalori:
\[\|L_\pm f_{l,m}\|^2 = (L_\pm f_{l,m}, L_\pm f_{l,m}) = (f_{l,m},L_\mp L_\pm f_{l,m}) = (f_{l,m}, (L^2 -L_z^2\pm \hbar L_z)f_{l,m}) =\]
\[\hbar^2(l(l+1)-m(m+1))(f_{l,m},f_{l,m})=\hbar^2(l(l+1)-m(m+1)) \]
dove è bastato sfruttare il problema agli autovalori \eqref{probautovalori}, dato che:
\[\|L_\pm f_{l,m}\|^2 = \|\hbar C^{\pm}_{l,m} f_{l,m\pm1}\|^2 = \hbar^2 (C^{\pm}_{l,m})^2 \|f_{l,m\pm1}\|^2 = \hbar^2 (C^{\pm}_{l,m})^2\]
confrontando questi ultimi due risultati è evidente che $C^{\pm}_{l,m} = \sqrt{l(l+1)-m(m+1)}$.\\
Quando studiamo il porblema agli autovalori di $L^2$, fissato $l$ abbiamo una degenarizione $2l+1$, questo significa che possiamo dare a $L^2$ una rappresentazione matriciale con una matrice $2l+1\times 2l+1$:
\[(f_{l,m},L^2f_{l,m'}) = \hbar^2 l(l+1) \delta_{mm'}\]
nella stessa maniera possiamo dare una rappresentazione matriciale a $L_z$:
\[(f_{l,m},L_z f_{l,m'}) = \hbar m \delta_{mm'}\]
e ovviamente anche ad $L_\pm,L_x,L_y$:
\[(f_{l,m},L_\pm f_{l,m'}) = (f_{l,m},\hbar C^{\pm}_{l,m'}f_{l,m'\pm 1}) = \hbar C^\pm_{l,m'}\delta_{m,m'\pm1}\]
\[(f_{l,m},L_xf_{l,m'}) = \frac{\hbar}{2}[C^+_{l,m}\delta_{m,m'+1} + C^-_{l,m}\delta_{m,m'-1}]\]
\[ (f_{l,m},L_yf_{l,m'}) = \frac{\hbar}{2i}[C^+_{l,m}\delta_{m,m'+1} - C^-_{l,m}\delta_{m,m'-1}]\]
da notare che mentre $L^2,L_z$ sono matrici diagonali le restanti non lo sono. Grazie a queste rappresentazioni è immediato calcolare i valori medi degli operatori $L_x,L_y$ quando $m=m'$, infatti per le delte di kronecker $\braket{L_x} = 0, \braket{L_y} = 0$. Non è nullo però il valore medio di $L_x^2$, infatti:
\[(f_{l,m},L_x^2 f_{l,m'}) = \frac{\hbar^2}{2}(l(l+1)-m^2)\]
\subsection{Calcolo di alcune armoniche sferiche}
Calcoliamo esplicitamente alcune armoniche sferiche in $L_2(S^2)$, ricordiamo che gli operatori momento angolare si possono scrivere come:
\[L_z = -i\hbar \partial_\varphi \qquad L_x = -i\hbar(y\partial_z - z\partial_y) \qquad L_y = -i\hbar(z\partial_x - x\partial_z)\]
in coordinate polari:
\[L_x = -i\hbar (\sin\varphi \partial_\theta - \cos\varphi\text{cotan}\theta\partial_\varphi)\qquad L_y = -i\hbar(\cos\varphi\partial_\theta -\sin\varphi\text{cotan}\theta\partial_\varphi )\]
i ladder operator in coordinate polari:
\[L_\pm = L_x\pm iL_y = \hbar e^{\pm i\varphi}[\pm \partial_\theta + i \text{cotan}\theta\partial_\varphi]\]
le armoniche sferiche sono gli autovalori di $L^2$ ed $L_z$ quindi dobbiamo risolvere il problema:
\[\begin{cases}
L^2Y_{l,m}(\theta,\varphi) = \hbar^2l(l+1)Y_{l,m}(\theta,\varphi)\\
L_zY_{l,m}(\theta,\varphi) = \hbar m Y_{l,m}(\theta,\varphi)
\end{cases}\]
cerchiamo armoniche sferiche con la parte in $\varphi$ fattorizzata come $ Y_{l,m}(\theta,\varphi) = \Theta(\theta)e^{im\varphi}$. Al posto che risolvere il problema agli autovalori di $L^2$ possiamo sfruttare il fatto che $L_+f_{l,l} = \hbar C^+_{l,l}f_{l,m+1} = 0$ dato che $C^+_{l,l}=0$, se  $L_+f_{l,l}=0$ allora anche $L_+Y_{l,l}=0$, che riscriviamo come:
\[e^{i\varphi}\left[\frac{\partial}{\partial\theta} + i \frac{\cos\theta}{\sin\theta}\frac{\partial}{\partial \varphi}\right]e^{il\varphi}\Theta_{l,l} = 0 \implies \Theta_{l,l}' -l \Theta_{l,l} \frac{\cos\theta}{\sin\theta} = 0\]
possiamo risolvere l'equazione per separazione delle variabili:
\[\frac{d\Theta_{l,l}}{\Theta_{l,l}} = l\frac{\cos\theta}{\sin\theta}d\theta \implies \log(\Theta_{l,l}) = \log(\sin^l\theta) \]
da cui la soluzione $\Theta_{l,l} = N_l \sin^l\theta$. Calcoliamo la costante di normalizzazione per $l=1$:
\[Y_{1,1} = N_1 \sin\theta e^{i\varphi} \implies \|Y_{1,1}\|^2_{S^2}=\int_0^{2\pi}d\varphi\int_0^{\pi}d\theta\, \sin\theta |Y_{1,1}|^2 = 2\pi N_1^2\int_0^{\pi}d\theta\, \sin^3\theta \]
facendo un cambio di coordinate $u=\cos\theta$ l'integrale diventa:
\[2\pi N_1^2\int_0^{\pi}d\theta\, \sin^3\theta  = 2\pi N_1^2\int_{-1}^{1}du\,(1-u^2) = \frac{8}{3}\pi N_1^2 \implies N_1 = \pm\sqrt{\frac{3}{8\pi}} \]
di solito il sengo davanti alla radice si sceglie per convenzione, solitamente si sceglie il segno meno. In conclusione abbiamo trovato che $Y_{1,1} = -\sqrt{\frac{3}{8\pi}} e^{i\varphi}\sin\theta$, grazie a questo risultato è possibile trovare anche $Y_{1,0}$ grazie ad i ladder operator, infatti:
\[L_- Y_{1,1} = \hbar C^{-}_{1,1} Y_{1,0} \implies Y_{1,0} = \frac{1}{\hbar}L_-Y_{1,1}  = \sqrt{\frac{3}{4\pi}}\cos\theta\]
applicando di nuovo il ladder operator possiamo trovare $Y_{1,-1}$, ma possiamo anche sfruttare una proprietà delle armoniche sferiche:
\[\overline{Y_{l,m}} = (-1)^m Y_{l,-m}\]
grazie alla quale possiamo immediatamente concludere che $Y_{1,-1} = \sqrt{\frac{3}{8\pi}} e^{-i\varphi}\sin\theta$
\chapter{Esercizi da esame}
\section{Autostati di $L^2$}
Dimostare che $\psi = A_1 g(r)(x+z)$ è un autostato di $L^2\psi = \hbar^2 l(l+1)\psi$ con $l=1$ nello spazio di Hilbert $L_2(\R^3)$ e calcolare la probabilità di ottenere $m=0,\pm1$ in una misura.\\
\newline 
Per prima cosa scriviamo la funzione d'onda in coordinate sferiche: $\psi = A_1 g(r)r(\sin\theta\cos\varphi+\cos\theta)$, sfruttiamo le armoniche sferiche per riscrivere lo stato $\cos\theta = \sqrt{\frac{4\pi}{3}}Y_{1,0}$, $\sin\theta\cos\varphi = \sqrt{\frac{4\pi}{3}} \frac{1}{\sqrt{2}}(Y_{1,-1} -Y_{1,1})$, con questo $\psi$ diventa:
\[\psi =A_1 \sqrt{\frac{4\pi}{3}}g(r)r\left(Y_{1,0} +\frac{1}{\sqrt{2}}Y_{1,-1} -\frac{1}{\sqrt{2}}Y_{1,1}\right) \]
in generale vale che una funzione d'onda $\psi_l = A f(r) \sum_{m'=-l}^{l}C_{l,m'}Y_{l,m'}$ se $l$ è fissato è un autostato di $L^2$ quindi nella forma in cui abbiamo scritto la $\psi$ è un autostato.\\
Per il calcolo delle probabilità dobbiamo normalizzare la funzione d'onda, in generale prendiamo l'autostato di $L_z$ $\varphi_{l,m} = Bf(r)Y_{l,m}$ la normalizazzione vale:
\[1 = \|\varphi_{l,m}\|^2 = \int_0^{+\infty}dr\,r^2 \|\varphi_{l,m}\|^2_{S^2} = \int_0^{+\infty}dr\,r^2 B^2 f^2(r) \implies B^2 = \frac{1}{\int_0^{+\infty}dr\,r^2f^2(r)}\]
infatti le armoniche sferiche sono già normalizzate $\|Y_{l,m}\|^2=1$. Normalizziamo anche $\psi_l$:
\[1 = \|\psi_l\|^2 =  \int_0^{+\infty}dr\,r^2 \|Af \sum_{m'=-l}^{l}C_{l,m'}Y_{l,m'}\|^2_{S^2} = A^2 \int_0^{+\infty}dr\,r^2 f^2(r)\|\sum_{m'=-l}^{l}C_{l,m'}Y_{l,m'}\|^2_{S^2} \]
dato che le armoniche sferiche sono un S.O.N.C possiamo usare pitagora generalizzato ed arrivare ad:
\[1 = \frac{A^2}{B^2}\sum_{m'=-l}^{l}|C_{l,m'}|^2 =  \frac{A^2N^2}{B^2} \implies A = \frac{B}{N}\]
dove abbiamo anche definito $N= \sum_{m'=-l}^{l}|C_{l,m'}|^2$. Per calcolare le probabilità dobbiamo calcolare i coefficienti di Fourier generalizzati:
\[\mathcal{P}(m) = |(\psi_l , \varphi_{l,m})|^2 = \left|\int_0^{+\infty}dr\,r^2 (\psi_l,\varphi_{l,m})_{S^2} \right|^2 = \frac{B^2}{N^2}  \left|\int_0^{+\infty}dr\,r^2 f^2\left(\sum_{m'=-l}^{l}C_{l,m'}Y_{l,m'},Y_{l,m}\right) \right|^2\]
\[\mathcal{P}(m)= \frac{|C_{l,m}|^2}{N^2}\]
nel nostro caso $A = A_1 \sqrt{\frac{4\pi}{3}}$, $f(r) = rg(r)$, $N^2 = 2$ applicando la formula appena trovata otteniamo:
\[\mathcal{P}(m=1) = \frac{1}{2}\qquad \mathcal{P}(m=\pm1) = \frac{1}{4}\]
\section{Nuclei integrali}
Sia dato l'operatore $K(x,y) = -\theta(y-x)y(1+x)-\theta(x-y)x(1+y)$ in $L_2(0,1)$, dimostrare che tale operatore è compatto autoaggiunto ed ha traccia, inoltre discutere l'alternativa di Fredholm $f(x) -\mu Kf = 1$ e fare la verifica della traccia.\\
\newline
Per dimostare che è compatto basta calcolare la norma di Hilbert-Schmidt e vedere se converge:
\[\|K\|_{H-S}^2 = \int_0^1 dx\int_0^1 |K(x,y)|^2 <\infty\]
chiramente nel nucleo non abbiamo alcun elemento che diverge. Il nucleo è simmetrico, infatti $K(x,y) = \overline{K(y,x)}$ quindi l'operatore è autoaggiunto. La traccia vale:
\[\text{Tr}K = \int_0^1 K(x,x)\,dx = -\int_0^1 (x+x^2) = -\frac{5}{6}\]
Per discutere Fredholm dobbiamo studiare le proprietà spettrali di $K$, per farlo studiamo l'operatore inverse che è più facile da trattare in quanto è un operatore differenziale. Per trovarlo sappiamo che vale $LK(x,y) = \delta(x-y)$ dove $L=K^{-1}$ con opportune condizioni al contorno. Iniziamo a calcolare la derivata di $K$:
\[\partial_xK = -y(1+y)\delta(x-y) +\delta(x-y)y(1+y)-\theta(y-x)y - \theta(x-y)(1+y) = -\theta(y-x)y - \theta(x-y)(1+y)\]
dato che la delta si cancella, dobbiamo andare ad ordini superiori di derivazione:
\[\partial_x^2K = \delta(x-y)y - \delta(x-y)-y\delta(x-y) = -\delta(x-y)\]
quindi $L= -\frac{d^2}{dx^2}$, vediamo ora le condizioni al contorno calcolando le seguenti quantità:
\[K(0,y) = -y \quad \partial_x K(0,y) = -y \qquad K(1,y) = -(1+y) \quad \partial_x K(1,y) = -(1+y) \]
è evidente che quindi le condizioni al bordo di Robin sono $K(0,y) = K'(0,y),K(1,y)=K'(1,y)$ che si riflettono sulle condizioni della $\varphi$ che risolve il problema agli autovalori $L\varphi = \lambda \varphi$: $\varphi(0) = \varphi'(0),\varphi(1)=\varphi'(1)$, risolviamo il problema spettrale:
\[-\varphi'' = \lambda \varphi\]
possiamo avere 3 casi, $\lambda=0$ che ci sarà una soluzione $\varphi = 0$ in quanto l'operatore è invertibile, il caso $\lambda = -u^2<0$:
\[\varphi'' = u^2\varphi \implies \varphi = C_1 e^{-ux} + C_2 e^{-ux}\]
imponiamo le condizioni al contorno:
\[\begin{cases}
C_1 +C_2 - uC_1 + uC_2 = 0\\
C_1 e^u +C_2 e^{-u} -uC_1e^u+uC_2e^{-u}=0
\end{cases}\]
in forma matriciale:
\[\begin{pmatrix}
1-u & 1+u \\
(1-u)e^u & (u+1)e^{-u}
\end{pmatrix}\begin{pmatrix}
C_1\\
C_2
\end{pmatrix} = 0\]
se vogliamo soluzioni non banali quindi $C_1 \neq 0,C_2\neq 0$ dobbiamo imporre il determinante nullo:
\[(1-u^2)e^{-u} - (1-u^2)e^u = 0 \implies (1-u^2)(e^{-u}-e^{-u})= 0 \implies u^2=1 \implies \lambda_1 = -1 \]
vediamo ora il caso di $\lambda = \omega^2 >0$, senza rifare i conti sostituiamo in quelli precedenti $u = i\omega$ e otteniamo quindi la condizione sul determinante:
\[(1+\omega^2)(e^{i\omega} - e^{-i\omega})=0\]
in questo caso la soluzione è $\sin\omega = 0$ ovvero $\omega_n = n\pi$ con $n=1,2,3,\dots$ quindi $\lambda_n = \omega^2 = n^2\pi^2$. Lo spettro di di $L$ quindi si può riassumere come $\sigma(L) = \{\lambda_1 = -1,\lambda_n =n^2\pi^2 \}$ gli autovalori di $K$ sono l'inverso di quello di $L$, possiamo quindi fare la verifica della traccia:
\[\text{Tr}K = \sum_n \frac{1}{\lambda_n} = -1 + \frac{1}{\pi^2}\sum_n\frac{1}{n^2} = -1 + \frac{1}{6} =  -\frac{5}{6}\]
%\xrightarrow[n\to \infty]{\H} 
\end{document}
