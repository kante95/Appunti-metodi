\documentclass[12pt]{article}
\usepackage{mathtools}
\usepackage[italian]{babel}
\usepackage{graphicx}
\usepackage[utf8]{inputenc}
\usepackage{float}
\usepackage{tabularx}
\usepackage{chngpage}
\usepackage{amsthm}
\usepackage{amsfonts}
\usepackage{gensymb}
\usepackage{tikz}

\begin{document}
\newcommand{\R}{\mathbb{R}}
\newcommand{\C}{\mathbb{C}}
\newcommand{\Sum}{\sum_{n=0}^\infty}
\newtheorem{prop}{Proposizione}
\newtheorem{thm}{Teorema}[section]
\newtheorem{lem}[thm]{Lemma}
\theoremstyle{definition}
\newtheorem{dfn}{Definizione}
\theoremstyle{remark}
\newtheorem*{rmk}{Remark}
\renewcommand{\d}[2]{\frac{d #1}{d #2}} % for derivatives
\newcommand{\dd}[2]{\frac{d^2 #1}{d #2^2}} % for double derivatives
\newcommand{\pd}[2]{\frac{\partial #1}{\partial #2}} 
% for partial derivatives
\newcommand{\pdd}[2]{\frac{\partial^2 #1}{\partial #2^2}} 
% for double partial derivatives
\section{Equazione di propagazione in una dimensione}
Si tratta di risolvere la seguente equazione differenziale
\begin{equation}
\label{eq}
\begin{cases}
    \pdd{u(x,t)}{t}-c^2\pdd{u(x,t)}{x}=0  \\
   u(x,0) = 0\\
   \pd{u}{t}(x,0) = g(x)
  \end{cases} 
\end{equation}
Applichiamo la trasformata di Fourier all'equazione per lo spazio e otteniamo
\[\mathcal{F}\left(\pdd{u(x,t)}{t}-c^2\pdd{u(x,t)}{x}\right) = 0\]
\[\pdd{\mathcal{F}(u(k,t)}{t}+c^2k^2\mathcal{F}(u(k,t)) =0\]
infatti la derivata rispetto al tempo passa fuori dalla trasformata di Fourier in quanto fatta solo rispetto allo spazio e la derivata rispetto allo spazio per proprietà della trasformata diventa un fattore $-k^2$ davanti. Scriviamo la trasformata cosi: $\mathcal{F}(u(k,t)) \equiv \tilde{u}(k,t)$ e otteniamo l'equazione
\[\pdd{\tilde{u}(k,t)}{t}+c^2k^2\tilde{u}(k,t)=0\]
definito $\omega = kc$ abbiamo
\[\pdd{\tilde{u}(k,t)}{t}+\omega^2\tilde{u}(k,t)=0\]
che è l'equazione armonica con soluzione nota:
\begin{equation}
\label{sol}
\tilde{u}(k,t) = A(k)\sin(\omega t+\phi)
\end{equation}
con $A(k)$ da determinare secondo le condizioni iniziali e $\phi=0$ per la prima condizione inziale. Per trovare $A(k)$ prendiamo la seconda condizione dell'equazione \eqref{eq} e applichiamo la trasformata di Fourier in modo da avere la condizione iniziale nello spazio di Fourier
\[\mathcal{F}\left(\pd{u}{t}(k,0)\right) = \mathcal{F}(g(k)) \qquad \pd{\tilde{u}}{t}(k,0) = \tilde{g}(k)\]
quindi prendiamo la soluzione \eqref{sol} deriviamola rispetto al tempo e imponiamo la condizione in $t=0$
\[\pd{\tilde{u}}{t}(k,t) = A(k)\omega \cos(\omega t)\]
\[\pd{\tilde{u}}{t}(k,0) =A(k)\omega = \tilde{g}(k) \]
da cui $A(k) = \tilde{g}(k)/\omega$
La nostra soluzione finale quindi è:
\[\tilde{u}(k,t) = \frac{\tilde{g}(k)}{\omega}\sin(\omega t)\]
Ora l'unica cosa che ci rimane è antitrasformare la $\tilde{u}(k,t)$ in modo da avere l'originale $u(x,t)$
\begin{equation}\label{f}
\mathcal{F}^{-1}(\tilde{u}(k,t)) = u(x,t) = \mathcal{F}^{-1}\left(\frac{\tilde{g}(k)}{\omega}\sin(\omega t)\right)
\end{equation}
Per il secondo membro utilizziamo il teorema della convuluzione, ovvero:
\[\mathcal{F}(f*\chi) = \mathcal{F}(f)\mathcal{F}(\chi)\]
infatti se nel nostro caso troviamo una $f$ e una $\chi$ in modo che: $\mathcal{F}(f) \equiv \tilde{g}(k)$ e $\mathcal{F}(\chi) \equiv \sin(\omega t)/\omega$, possiamo scrivere l'equazione \eqref{f} come:
\[u(x,t) = \mathcal{F}^{-1}\left(\mathcal{F}(f*\chi)\right) = f*\chi\]
trovare $\mathcal{F}(f) \equiv \tilde{g}(k)$ è facile, per definizione di trasformata di Fourier $f\equiv g$. Dobbiamo ora trovare una $\chi$ tale che 
\[\mathcal{F}(\chi)=\tilde{\chi} = \frac{\sin(\omega t)}{\omega}\]
Facciamo che magicamente sappiamo che è: \[\chi = \frac{1}{2c}\theta(ct-|x|)\] e verifichiamo che effettivamente la sua trasformata sia $\sin(\omega t)/\omega$
Consideriamo che la funzione $\theta(x)$ è definita come:
\[\theta(x) =\begin{cases}
    1 \qquad x>0  \\
   0\qquad x<0\\
  \end{cases}\]
quindi $\theta(ct-|x|) = 1$ quando $ct-|x|>0$ e $\theta(ct-|x|) = 0$ quando $ct-|x|<0$. Abbiamo quindi:
\[\theta(ct-|x|) = 1 \implies ct-|x|>0 \implies |x|<ct \implies -ct<x<ct\]
ovviamente allora:
\[\theta(ct-|x|) = 0 \implies x<-ct \quad\&\quad x>ct\]
Graficamente abbiamo quindi:
\begin{figure}[H]
\centering
\begin{tikzpicture}
\draw [->](-5,0) -- (5,0)node[below=1.5pt] {\color{black}$x$};
\draw[->](0,0)--(0,5)node[right=1.5pt] {\color{black}$\theta(ct-|x|)$};
\draw[blue](-5,0)--(-2.5,0)node[below=1.5pt] {\color{black}$-ct$};
\draw[blue](2.5,0)node[below=1.5pt,yshift=-3pt] {\color{black}$ct$}--(5,0);
\draw[blue](-2.5,0)--(-2.5,2.5);
\draw[blue](2.5,0)--(2.5,2.5);
\draw[blue](-2.5,2.5)--(0,2.5)node[above=1.5pt,xshift=-3pt] {\color{black}$1$}--(2.5,2.5);
\end{tikzpicture}
\end{figure}
Per definizione di trasformata abbiamo:
\[\tilde{\chi}(k) = \int_{-\infty}^{+\infty}e^{-ikx}\chi(x)\,dx =\frac{1}{2c}\int_{-\infty}^{+\infty}e^{-ikx}\theta(ct-|x|)\,dx \]
ma per come abbiamo appena mostrato essere $\theta$, l'integrando è nullo per $x<-ct$ e $x>ct$, mentre vale $e^{-ikx}$ altrimenti, ci riduciamo quindi a:
\[\frac{1}{2c}\int_{-\infty}^{+\infty}e^{-ikx}\theta(ct-|x|)\,dx = \frac{1}{2c}\int_{-ct}^{+ct}e^{-ikx}\,dx = \frac{1}{c}\left(\frac{e^{-ikx}}{-ik}\right)\bigg\vert_{-ct}^{ct}\]
\[= \frac{1}{2c}\left(\frac{e^{-ikct}-e^{ikct}}{-ik} \right) =\frac{1}{ck}\left(\frac{e^{ikct}-e^{-ikct}}{2i} \right) \]
tenendo conto che:
\[\sin(x) = \frac{e^{ix}-e^{-ix}}{2i}\]
sapendo che $\omega=ct$ otteniamo come ci aspettavamo:
\[\tilde{\chi}(k) =\frac{\sin(\omega t)} {\omega}\]
Torniamo all'equazione \eqref{f} abbiamo ottenuto che:
\[u(x,t) = g*\chi\]
che per definizione di prodotto di convuluzione:
\[u(x,t) = \int_{-\infty}^{+\infty}g(x-y)\chi(y)\,dy = \frac{1}{2c}\int_{-\infty}^{+\infty}g(x-y)\theta(ct-|y|)\,dy\]
che per le considerazioni già fatte prima:
\[\frac{1}{2c}\int_{-\infty}^{+\infty}g(x-y)\theta(ct-|y|)\,dy=\frac{1}{2c}\int_{-ct}^{+ct}g(x-y)\,dy = u(x,t)\]
\end{document}
